\documentclass[twoside]{thesis}

\usepackage{/home/greg/thesis/thesis}
\usepackage{/home/greg/thesis/definitions}
\usepackage{/home/greg/thesis/tikz}


\begin{document}
\def\biblio{}

\begin{titlepage}
\newcommand{\HRule}{\rule{\linewidth}{0.5mm}}
\begin{center}

\textsc{\LARGE University of Southampton}\\[0.2cm]
\textsc{\Large Faculty Of Social And Human Sciences}\\[0.2cm]
\textsc{\Large Mathematical Sciences}\\[1.5cm]

\HRule \\[0.4cm]
{ \huge \bfseries Timing noise in pulsars \\[0.4cm] }

\HRule \\[1.5cm]

% Author and supervisor
\noindent
\begin{minipage}{0.4\textwidth}
\begin{flushleft} \large
\emph{Author:}\\
Gregory \textsc{Ashton}
\end{flushleft}
\end{minipage}%
\begin{minipage}{0.4\textwidth}
\begin{flushright} \large
\emph{Supervisors:} \\
Dr.~David~Ian \textsc{Jones} \\
Dr.~Reinhard \textsc{Prix}
\end{flushright}
\end{minipage}

\vfill

% Bottom of the page
{\large \today}
\end{center}
\end{titlepage}

\tableofcontents

\chapter{Introduction to the physics of neutron stars}
\label{sec: neutron star physics}
\subfile{../introduction/introduction}

\chapter{Neutron star precession in the rotating frame}
\label{sec: rotating frame}
\subfile{../rotating_frame/rotating_frame}

\chapter{Neutron star precession in the inertial frame}
\label{sec: inertial frame}
\subfile{../inertial_frame/inertial_frame}

\chapter{Testing models of periodic modulations for B1828-11}
\label{sec: }
\subfile{../comparing_periodic_modulations/comparing_periodic_modulations}

\chapter{Extending the precession model for B1828-11}
\chapter{Introduction to continuous gravitational waves: source modelling and detection}

\chapter{Timing noise in continuous gravitational waves}
\label{sec: timing noise in cgw}
\subfile{../timing_noise_in_CGW/timing_noise_in_CGW}

\chapter{Glitches in continuous gravitational waves}


\bibliography{../bibliography}


\end{document}
