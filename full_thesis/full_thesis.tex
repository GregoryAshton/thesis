\documentclass[twoside, 11pt]{thesis}

\usepackage{/home/greg/thesis/thesis}
\usepackage{/home/greg/thesis/definitions}
\usepackage{/home/greg/thesis/tikz}


\begin{document}
\def\biblio{}

\ifdefined\nocolouredlinks
  \removecolourlinks
\fi

\frontmatter
%\title      {Timing variations in neutron stars}
%\title      {A waste of timing variations}
\title      {Timing variations in neutron stars: models, inference, and their
             implications for gravitational waves}
\authors    {Gregory Ashton}
\faculty  {Faculty of Social, Human and Mathematical Sciences}
\department    {Gravity Group}
%\addresses  {\groupname\\\deptname\\\univname}
\date       {\today}
\subject    {}
\keywords   {}
\maketitle

\begin{abstract}

Timing variations in pulsars, low frequency ubiquitous structure known as
timing noise and sudden increases in the rotational frequency which we call
glitches, provide a means to study neutron stars. Since the first observations,
many models have been proposed, yet no definitive explanation has arisen.

In this thesis, we aim to improve this situation by developing models of timing
noise. We focus chiefly on precession models which explain periodic modulation
seen in radio pulsar data.  Developing models and testing them provides an
opportunity to infer the elemental properties of neutron stars: evidence for
long period precession has implications for the superfluid component predicted
by models used to explain glitches. However, often more than one model can
qualitatively explain the data, therefore we need a method to decide which
model best fits the data. This is precisely the case for PSR~B1828-11 which has
been used as evidence for both precession and so-called magnetospheric
switching. We address this confusion by applying the tools of probability
theory to develop a Bayesian model comparison and find that the evidence is in
favour of precession.

In the second part of this thesis, we will discuss the implications of timing
variations for the detection of continuous gravitational waves from neutron
stars. To search for these signals, matched filtering methods are used which
require a template, a guess for what the signal `looks like'. Timing
variations, as seen in the electromagnetic signal, may also exist in the
gravitational wave signal. If detected, these could provide an invaluable
source of information about neutron stars. However, if not included in the
template, they may mean that the gravitational wave signal is not detected in
the first place. We investigate this issue for both timing noise and glitches,
using electromagnetic observations to predict for what types of gravitational
wave searches this may be an issue. We find that while timing noise is unlikely
to be an issue for current gravitational wave searches, glitches may cause a
significant problem in all-sky searches for gravitational waves from neutron
stars.

\end{abstract}

{
\newlength\longest
\thispagestyle{empty}
\null\vfill

\settowidth\longest{\huge\itshape xxxxxxxxxxxxxxxxxxxxxxxx}
\centering
\parbox{\longest}{%
  \raggedright{\huge\itshape%
  One man's `noise' is another man's `signal'.
  \par\bigskip
  }
  \raggedleft\Large\MakeUppercase{Edwin Thompson Jaynes}\par%
}
\vfill\vfill
}

\tableofcontents
%\listoffigures
%\listoftables

%% -----------------------
%% Authorship declaration
%% -----------------------
\authorshipdeclaration{\citep{Ashton2015} and \citep{Ashton2016}.}
%% -----------------------

\acknowledgements{

I would like to thank Ian Jones, my supervisor in Southampton, for his tireless
efforts to guide my research in a meaningful direction, his clear and honest
evaluation, but most importantly for being a friendly and enthusiastic mentor.
Reinhard Prix, my supervisor in Hannover, also deserves a hearty thanks for his
enduring commitment, many useful insights, and for introducing me to Bayesian
probability theory, without which much of this thesis would not exist.  It has
been a pleasure to work with both Ian and Reinhard.  Thanks also to my advisor,
Nils Andersson, for his useful input early on and his continued humour
throughout such as his brilliant suggestion to title the work `A waste of
timing noise'. To my examiners, Graham Woan and Wynn Ho, I extend my thanks
for taking the time to provide detailed and insightful comments which I
thoroughly appreciate.

For the work presented in Chapter~\ref{sec: testing models}, I am grateful to
Will Farr, Danai Antonopoulou, and Ben Stappers for valuable discussions and
comments, Dan Foreman-Mackay  for the software used in generating posterior
probability distributions \citep{dan_foreman_mackey_2014_11020}, and Andrew
Lyne \citep{Lyne2010} for generously sharing the data for PSR~B1828-11.  For
the work presented in Chapter~\ref{sec: glitches in cgw}, I kindly thank
Christobal Espinoza for maintaining the glitch catalogue and for helping to
extract the required data. I would also like to express my gratitude to Matthew
Pitkin whose advice on MCMC simulation software was of great help.

All of my friends in the Maths departments deserve my praise both for their
moral support and making Southampton an enjoyable place to live and work.  To
my final year companions, Vanessa Graber, Marta Colleoni, Liana Kontogeorgaki,
and Yafet Sanchez, I have thoroughly enjoyed the adventure we had together and I
wish you all the best of luck in your future pursuits.  I particularly would
like to acknowledge Yafet for his willingness to question everything and
reminding me that while life may be difficult, consider the alternative.
Matthew Cobain, Marc Scott and Andrew Meadowcroft, I thank for their kind help
in editing this thesis, many hours of badminton, many enjoyable discussions,
and many gin \& tonics.

My heartfelt thanks to Emily Lawrance for her solid support, wonderful cooking,
and for keeping me sane; to Christine Ashton for her everlasting
enthusiasm and words of wisdom; and finally to the man who inspired me to such
heights, Mark Ashton.

}

%\listofsymbols{ll}{
%$\nu$ & The observed spin frequency of a pulsar
%}

\mainmatter

\chapter{Introduction}
\label{sec: neutron star physics}
\subfile{../introduction/introduction}

\chapter{Timing variations}
\label{sec: timing variations}
\subfile{../timing_variations/timing_variations}

\chapter{Action of the electromagnetic torque on a precessing neutron star}
\label{sec: rotating frame}
\subfile{../rotating_frame/rotating_frame}

\chapter{Modelling observations of precessing pulsars}
\label{sec: inertial frame}
\subfile{../inertial_frame/inertial_frame}

\chapter{Comparing models of the periodic variations in spin-down
and beam-width for PSR B1828-11}
\label{sec: testing models}
\subfile{../comparing_periodic_modulations/comparing_periodic_modulations}

\chapter{Continuous gravitational waves: calculating the mismatch}
\label{sec: intro to cgw}
\subfile{../detecting_cgw/detecting_cgw}

\chapter{Glitches in continuous gravitational waves}
\label{sec: glitches in cgw}
\subfile{../glitches_in_CGW/glitches_in_CGW}

\chapter{Timing noise in continuous gravitational waves: a numerical study}
\label{sec: timing noise in cgw}
\subfile{../timing_noise_in_CGW/timing_noise_in_CGW}

\chapter{Timing noise in continuous gravitational waves: random walk models}
\label{sec: timing noise in cgw analytic}
\subfile{../analytic_timing_noise_cgw/analytic_timing_noise_cgw}

\chapter{Conclusion and outlook}
\label{sec: final conclusion}

Timing variations in pulsars have long been used as a way to infer the
elemental properties of neutron stars. However, while in isolated cases models
for glitches and timing noise have successfully explained the observations, we
are some way from having a complete and universal understanding. For timing
noise, this may be due to the varied ways in which it manifests, but in the
comprehensive review by \citet{Hobbs2010} the authors suggest that over
sufficiently long timescales (which may be much longer than we are able to
observe) timing residuals show quasi-periodic features. Moreover, some pulsars
(see \citet{Lyne2010} for examples) have pronounced periodic modulations in not
only their timing residuals, but also the shape of pulsations, PSR~B1828-11 is
the best example of this.  With this in mind, in this thesis we investigated
two mechanisms of strictly periodic variations in detail: magnetospheric
switching and precession.

We began in Chapter~\ref{sec: rotating frame} by studying the action of the
electromagnetic torque on precessing pulsars using numerical solutions to the
Euler rigid-body equations coupled to the \citet{Deutsch1955} torque. This
allowed us to investigate the role of the anomalous component of the torque and
conclude that, for realistic neutron stars, this can safely be ignored. We
cleared up some confusion in the literature, by showing that the `persistent
precession' solutions found by \citet{Melatos2000} were in fact not
precessing, but aligned with the principal axes of the effective body frame
arising the inclusion of the anomalous torque.

In order to test models they need to be predictive, in the sense that they
model the features of a neutron star observed by pulsar astronomers. To address
this in the context of precession, in Chapter~\ref{sec: inertial frame} we
developed numerical solutions which, together with solving the Euler rigid-body
equations, solve for the Euler rotation angles which transform from the
rotating frame to the inertial frame. This allowed us to directly model the
phase residual, spin-down rate, and pulse profile of a precessing pulsar.
We compared the numerical model against analytic solutions from the literature
and against our derivation of the spin-down rate which we used later in
Chapter~\ref{sec: testing models}. We finished this chapter by discussing some
preliminary results from a hybrid model which couples precession to
magnetospheric torque switching. In the future, it would be interesting to explore
this further by understanding better how reconfiguring the magnetosphere will
effect the \citet{Deutsch1955} torque and then testing different mechanisms to
bias, in a probabilistic way, the switching using the precessional cycle. The
numerical model is perfectly suited to this task as it captures the complicated
feedback between the torque switching  and precession.

PSR~B1828-11 stands out in the literature for the strong $\sim 500$~day
periodic modulations observed in its timing properties. As a result, it has
been used as evidence for both precession \citep{Stairs2000} and periodic
magnetospheric torque switching \citep{Lyne2010}. \citet{Jones2012} and
\citet{Cordes2013} suggested that these used models may not be mutually exclusive,
which prompted our hybrid model discussed in Chapter~\ref{sec: inertial frame}.
Neglecting these hybrid models, precession and magnetospheric
switching are mutually exclusive, but it is important to know which is favoured
since both have important implications for neutron star physics. To decide
this, in Chapter~\ref{sec: testing models}, we applied a Bayesian model
comparison. We found an odds-ratio of $10^{2.7\pm 0.5}$ in favour of the
precession model, a key result of this thesis which we published in
\citet{Ashton2016}. This does not rule out the switching interpretation
entirely as we have not tested an exhaustive set of models, but it does provide
a quantitative framework to evaluate models. In this chapter  we focus
primarily on the methods used to ensure that we make an unbiased comparison, although
there is some development for the modulations of the beam-width due to
precession. In the future, we intend to use these tools to test further
modifications of the models. Furthermore, we would like to combine the data
from other pulsars with long period modulations, such as PSR~0919+06
\citep{Perera2015}, and repeat the model comparison to see how this changes the
odds-ratio.

Detecting gravitational waves from an isolated neutron star would provide a
unique opportunity to learn about them, especially if we additionally saw the
object electromagnetically as well. However, signals which are subject to
timing variations may be difficult to detect if, as in the case of many current
gravitational wave searches, we use matched filtering templates which do not
include timing variations. This issue has not been properly tackled in the
literature and so in the last chapters of this thesis we begin the process of
quantifying the risks.

In Chapter~\ref{sec: intro to cgw} we set out the tools which we will use to
calculate the mismatch, defined to be the loss of signal to noise ratio. In
doing so we defined a new approach, the generalised metric-mismatch, which can
calculate the mismatch for any arbitrary signal by approximating it by a
piecewise Taylor expansion.

Searches for gravitational waves from known pulsars can handle glitches in the
signal since they can see that they have occurred. In Chapter~\ref{sec:
glitches in cgw} we showed that for blind searches, where we have no
electromagnetic signal, glitches pose a substantial risk, especially since many
searches target young rapidly spinning-down pulsars which we found to be
likely to have larger and more frequent glitches. Many of these searches use
an initial semi-coherent stage and then follow-up candidates with
fully-coherent searches.  We showed how a signal can be identified as a
candidate in the semi-coherent stage and then subsequently lost in the
follow-up. In the future, we would like to develop search methods which are
robust to glitches. These could then be applied to past data to ensure that we
have not already missed a continuous gravitational wave signal.

Quantifying the effect of timing noise on gravitational wave searches is a more
difficult task due to the fact that we have no universal empirical description
of what constitutes timing noise; therefore, it is difficult to predict what
timing noise may exist in gravitational wave signals from isolated neutron
stars.  One way to approach this, discussed in Chapter~\ref{sec: timing noise
in cgw}, is to use data on the frequency and spin-down rate evolution of the
Crab pulsar to generate `noisy gravitational wave signals'.  We searched for
these using standard matched filtering tools and calculated the minimum
mismatch having searched in a narrow band of frequency and spin-down rate. The
conventional wisdom for continuous wave searches contends that, since the SNR
scales with the square-root of the observation time, provided we look for a
sufficiently long time the signal will eventually become detectable.  However,
by investigating how the minimum mismatch scaled with the observation time for
noisy signals, we found that $\langle \mutilde_{\textrm{min}} \rangle \propto
\Tobs^{2.88}$.  This result, published in \citet{Ashton2015}, means that for
noisy signals the conventional wisdom does not hold, there is in fact an
observation time after which the SNR decreases with observation time; for the
Crab pulsar this was found to be $ \approx 600$~days.

In Chapter~\ref{sec: timing noise in cgw analytic} we approached the issue of
timing noise in continuous gravitational waves from a different angle. Namely,
by modelling timing noise as a random walk in the phase, frequency, or
spin-down rate: a phenomenological description dating back to
\citet{Boynton1972}. Recently, this model has been disfavoured
\citep{Hobbs2010} as a substantive explanation of timing noise, but it
nevertheless provides a simple empirical description which is consistent with
the timing residuals of many pulsars. Moreover, while we apply it to isolated
neutron stars, it can also be applied to low-mass X-ray binary systems where
spin-wandering due to fluctuations in the torque can be modelled by a random
walk. In this chapter, we calculated the mismatch due to random walks and
showed how to include the minimisation step modelling a narrow-band search over
the template frequency and spin-down rate.  We go on to use the Crab ephemeris
to predict the strength of frequency noise in the Crab pulsar and hence the
dependence of the mismatch on observation time found in Chapter~\ref{sec:
timing noise in cgw}. This will be useful in developing models that we would
like to use in the future to estimate the levels of mismatch for other searches
which may be effected by random walk models of timing noise.

%Finally, we present some preliminary results predicting the mismatch due to
%random walk models of timing noise in blind gravitational wave searches by
%using the \citet{Cordes1989} fitting formulae.

In this thesis, we hope to have developed the understanding of timing noise in
neutron stars. Most significantly, by providing a framework with which to debate
the merits of models explaining periodic signals and quantify their comparison.
We also hope that by understanding the role of timing variations in continuous
gravitational waves, we help to guide efforts to detect these elusive signals
in the future.






\bibliography{../bibliography}


\end{document}
