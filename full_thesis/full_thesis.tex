\documentclass[twoside]{thesis}

\usepackage{/home/greg/thesis/thesis}
\usepackage{/home/greg/thesis/definitions}
\usepackage{/home/greg/thesis/tikz}


\begin{document}
\def\biblio{}

\frontmatter
%\title      {Timing variations in neutron stars}
%\title      {A waste of timing variations}
\title      {Timing variations in neutron stars: models, inference, and their
             implications for gravitational waves}
\authors    {Gregory Ashton}
\faculty  {Faculty of Social, Human and Mathematical Sciences}
\department    {Gravity Group}
%\addresses  {\groupname\\\deptname\\\univname}
\date       {\today}
\subject    {}
\keywords   {}
\maketitle

\begin{abstract}

Timing variations in pulsars, either the low frequency ubiquitous structure
known as timing noise or the sudden increase in rotation frequency which we
call glitches, provide a means to study neutron stars. Since the first
observations were made of these phenomena, many models have been proposed to
explain them, yet no definitive explanation has arisen. In particular timing
noise, perhaps due to the variety of phenomena it pertains too, lacks any
universal description.

In this thesis we aim to improve this situation by developing models of timing
noise, we focus chiefly on models which explain periodic modulation. Developing
such models is important as it provides an oppotunity to understand more about
how neutron stars work. However, often more than one model can qualitatively
explain the data, therefore we need a method to decide which model best fits
the data. To do this we apply the tools of probability theory and inference.

Our knowledge of neutron stars is exclusively derived from our electromagnetic
observations of them. In the future we may detect gravitational waves from neutron
stars. To do this, we need to use matched filtering methods which assume a
search for the signal using a template. Timing variations as seen in the electromagnetic
signal may also exist in the gravitational wave signal. If detected these could
provide an invaluable source of information about neutron stars. However, if not
included in the template, they may mean that the signal is not detected in the
first place. We will investigate this issue for both timing noise and glitches
to predict for what types of searches this may be an issue.



\end{abstract}

\tableofcontents
%\listoffigures
%\listoftables

%% -----------------------
%% Authorship declaration
%% -----------------------
\authorshipdeclaration{\citep{Ashton2015} and \citep{Ashton2016}.}
%% -----------------------

\acknowledgements{
For the work presented in Chapter.~\ref{sec: testing models} we especially
thank Will Farr, Danai Antonopoulou, and Ben Stappers for valuable discussions
and comments, \citet{dan_foreman_mackey_2014_11020} for the software used in
generating posterior probability distributions, and \citet{Lyne2010} for
generously sharing the data for PSR~B1828-11.

For the work presented in Chapter.~\ref{sec: glitches in cgw} we kindly thank
Christobal Espinoza for advice on working with the glitch catalogue.

}


\listofsymbols{ll}{}
\mainmatter

\chapter{Introduction}
\label{sec: neutron star physics}
\subfile{../introduction/introduction}

\chapter{Timing variations}
\label{sec: timing variations}
\subfile{../timing_variations/timing_variations}

\chapter{Neutron star precession in the rotating frame}
\label{sec: rotating frame}
\subfile{../rotating_frame/rotating_frame}

\chapter{Neutron star precession in the inertial frame}
\label{sec: inertial frame}
\subfile{../inertial_frame/inertial_frame}

\chapter{Comparing models of the periodic variations in spin-down
and beam-width for PSR B1828-11}
\label{sec: testing models}
\subfile{../comparing_periodic_modulations/comparing_periodic_modulations}

\chapter{Extending the precession model for B1828-11}
\label{sec: extending precession models}

\chapter{Introduction to continuous gravitational waves: source modelling and detection}
\label{sec: intro to cgw}
\subfile{../detecting_cgw/detecting_cgw}

\chapter{Glitches in continuous gravitational waves}
\label{sec: glitches in cgw}
\subfile{../glitches_in_CGW/glitches_in_CGW}

\chapter{Timing noise in continuous gravitational waves: a numerical study}
\label{sec: timing noise in cgw}
\subfile{../timing_noise_in_CGW/timing_noise_in_CGW}

\chapter{Timing noise in continuous gravitational waves: analytic models}
\label{sec: timing in cgw analytic}
\subfile{../analytic_timing_noise_cgw/analytic_timing_noise_cgw}



\bibliography{../bibliography}


\end{document}
