\section{Goldreich's result}
\label{sec: goldreich}
Starting with 
\begin{equation}
A=\frac{M^{2}}{2EI_{0}},
\end{equation}
and differentiating
\begin{equation}
\dot{A}=\frac{1}{(2EI_{0})^{2}}
        \left(\frac{d}{dt}(M^{2})2EI_{0}-\frac{d}{dt}(2EI_{0})M^{2}\right),
\end{equation}
simplifying and writing in index notation we have
\begin{equation}
\dot{A}=\frac{1}{E^{2}}\left(2M^{i}\dot{M}_{i}E-\dot{E}M^{i}J_{i}\right).
\end{equation}
Now $E=\frac{1}{2}I_{ij}\omega^{i}\omega^{j}$,  $\dot{E}=T^{i}\omega_{i}$ and
$\dot{M}^{i}=T^{i}$, and so
\begin{equation}
\dot{A}=\frac{1}{E^{2}}\left(M^{i}T_{i}I_{jk}\omega^{j}\omega^{k}
                             -T^{i}\omega_{i}M^{j}J_{j}\right).
\end{equation}
For a diagonalised matrix, $I_{ij}$ has only 3 non zero components and so
$I_{jk}\omega^{j}\omega^{k}=(I\omega)_{j}\omega^{j}$. In addition
$M^{i}=(I\omega)^{i}$ and so writing
\begin{equation}
\dot{A}=\frac{1}{E^{2}}T^{i}\left((I\omega)_{i}\omega_{j}
        -\omega_{i}(I\omega)_{j}\right)(I\omega)^{j},
\end{equation}
\begin{equation}
\dot{A}=\frac{1}{E^{2}}T^{i}\omega_{i}\omega_{j}
        (I\omega)^{j}\left(I^{i}_{\;i}-I^{j}_{\;j}\right).
\end{equation}
Considering for the time being only the last few terms and expanding the
summation it is clear all $i=j$ terms will be zero. Working with a biaxial mass
distribution where $I_{1}=I_{2}$, then the only non-zero terms are given by 
\begin{eqnarray*}
T^{i}\omega_{i}\omega_{j}(I\omega)^{j}\left(I^{i}_{\;i}-I^{j}_{\;j}\right)&=&
T_{1}\omega_{1}I_{0}(1-\epsilon_{i})\omega^{2}_{3}
(I_{0}-I_{0}(1+\epsilon_{I}))+
T_{2}\omega_{2}I_{0}(1-\epsilon_{i})\omega^{2}_{3}(I_{0}-I_{0}(1+\epsilon_{I})) 
\\ && + 
T_{3}\omega_{3}(I_{0}(1+\epsilon_{I})-I_{0})
               (I_{0}\omega_{1}^{2}+I_{0}\omega_{2}^{2}), \\
&=& I_{0}^{2}\epsilon_{I}\omega_{3}\left(T_{3}(\omega_{1}^{2}+\omega_{2}^{2}) 
    - \omega_{3}(1+\epsilon_{I})(T_{1}\omega_{1}+T_{2}\omega_{2})\right).
\end{eqnarray*}
Working up to 1st order in $\epsilon_{I}$ and inserting the spherical polar
coordinates and then the spindown torque (we neglect the anomalous torque here)
we have
\begin{eqnarray*}
&=& I_{0}^{2}\epsilon_{I}\omega^{3}
    \left(\frac{2R}{3c}\epsilon{A}\omega^{3}\right)
    \cos a \left([\sin \chi \cos \chi \sin a \cos \phi 
                 - \sin^{2}\chi \cos a]\sin^{2} a \right. \\
&& \left.- \cos a ([\sin \chi \cos \chi \sin a \cos \phi 
                   - \sin^{2}\chi \cos a]\cos \phi \sin a 
                   - \sin^{2}a \sin^{2}\phi)\right).
\end{eqnarray*}
Of course to know the exact behaviour, one would need to fully solve the
original ODEs. To show agreement with Goldreich's result, we can assume that
$\phi$ varies much faster than $a$ and $\omega$ such that $\tau_{P}<<\tau_{S}$.
Then averaging over a free nutation period and neglecting the coefficients for
the time being
\begin{eqnarray*}
&\sim& \cos a \left(-\sin^{2}a\cos a \sin^{2}\chi + 
       \frac{1}{2}\left(\sin^{2}a\cos a \cos^{2}\chi + 
       \sin^{2}a \cos a \right) \right) \\
& \sim& \cos^{2} a \sin^{2}a\left(1-\frac{3}{2}\sin^{2}\chi\right).
\end{eqnarray*}
Finally putting all the terms together we have an expression for the averaged
rate of change of $A$ with respecting Goldreich's assumptions:
\begin{equation}
\langle \dot{A} \rangle =\frac{1}{\langle E\rangle ^{2}}I_{0}^{2}
                         \epsilon_{I}\omega^{6}\frac{2R}{3c}\epsilon{A} 
                         \cos^{2} a \sin^{2}a
                         \left(1-\frac{3}{2}\sin^{2}\chi\right) 
\end{equation}

\section{Considerations for the timescales}\label{sec: timescales}
We have used the ordering of the three timescale at $\omega_{0}=\omega(t=0)$
to categorise the results. However, clearly the magnitude of the spin vector
will decrease and the different dependencies on the spin vector could cause a
`crossing' of the timescales producing unexpected results. Writing the
timescales then as 
\begin{equation}
\tau_{P}(t)=\frac{2\pi}{\epsilon_{I}\omega(t)}, \;\;\;\;\; 
\tau_{A}(t)=\frac{2\pi}{\epsilon_{A}\omega(t)},  \;\;\;\;\; 
\tau_{S}(t)=\frac{3c}{2R}\frac{2\pi}{\epsilon_{A}\omega(t)^{2}}.
\end{equation}
Only $\tau_{S}$ could cross with the other two timescales since the spin down
and anomalous timescales obey $\tau_{S}>\tau_{A}$ provided that 
\begin{equation*}
\omega(t)>\frac{3c}{2R}.
\end{equation*}•
This condition is satisfied by setting the initial rotational spin frequency at
less than $\omega_{0} \sim 10^{4}$ [Hz rad], that is the star should not break
special relativity. At later times the spin frequency will decay and so this
condition is still satisfied.

For the anomalous torque and precession timescale crossings two cases exists:
either the deformations are such that the spin down timescale is larger than
the precession initially (region A and B) or, as in region C, the precession
timescale is larger than then spin down timescale. Considering the first case:
\begin{equation}
\tau_{S}>\tau_{P} \;\;\; 
\Rightarrow \omega(t)<\frac{3c}{2R}\frac{\epsilon_{I}}{\epsilon_{A}}.
\end{equation}•
In this particular state $\epsilon_{I}>\epsilon_{A}$ and so as in the previous
case while the spin vector decays this inequality is always satisfied and the
orderings remain the same. In the other case we have:
\begin{equation}
\tau_{P}>\tau_{S} \;\;\; 
\Rightarrow \omega(t)>\frac{3c}{2R}\frac{\epsilon_{I}}{\epsilon_{A}}.
\end{equation}
In this case it is possible for the time scales to cross at
$\omega_{cr}=\frac{3c}{2R}\frac{\epsilon_{I}}{\epsilon_{A}}$. For pulsar C this
corresponds to a rotational spin frequency of $0.9$ [Hz rad] which is four
orders of magnitude smaller than the initial frequency. For this reason we can
rule out this crossing of the time scales as an important factor in
calculations. 


