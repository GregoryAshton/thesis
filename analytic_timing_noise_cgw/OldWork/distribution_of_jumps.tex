\subsubsection{Distribution of jumps}
In Eqn.~\eqref{eqn: delta fdot n}  we defined the steps in spin-down to be
normally distributed. Therefore the jumps between spin-down values will also be
normally distributed, that is $\fdot_{i} - \fdot_{i-1} \sim N(0, \sigS)$. In
the lower order terms, for example the frequency, we can have a mix of the
random walk in frequency and the induced effect of the random walk in spin-down;
the same is true for the phase but with an induced effect from the frequency
and spin-down.  We now investigate how the induced effect of higher order random
walks effects on the distributed of jumps in frequency and phase. To check this
we  look at the distribution at the $i^{th}$ step using equation 
\eqref{eqn: delta f n}
\begin{align}
\Delta \f_{i} - \Delta \f_{i-1} & = \s{j=1}{i}\tn \f_{j}
-  \s{j=1}{i-1}\tn \f_{j} + \left(\s{j=1}{i-1}(i-j)\tn \fdot_{j}  
-  \s{j=1}{i-2}(i-1-j)\tn \fdot_{k} \right)\dT  \\
& = \tn \f_{i} + \left(\tn \fdot_{i-1} 
+ \s{j=1}{i-2}(i-j-i+1+j)\tn \fdot_{j}\right)\dT \\
& = \tn \f_{i} + \s{j=1}{i-1}\tn \fdot_{j} \dT \\
& \sim N(0, \sigF) + \s{j=1}{i-1}N(0, \sigS) \dT 
\end{align}
To combine the distributions we use the summation property of normal
distributions. For a central normal distribution $X_{i}\sim N(0,
\sigma_{i}^{2})$, the sum of $N$ terms with real coefficients $a_{i}$ is given
by 
\begin{equation}
\s{i=1}{N}a_{i}X_{i} \sim N\left(0, \s{i=1}{N}(a_{i}\sigma_{i})^{2}\right).
\end{equation}
Then we have
\begin{align}
\Delta \f_{i} - \Delta \f_{i-1} & \sim N(0, \sigF) 
+ N\left(0, \s{j=1}{i-1}(\sigS\dT)^{2}\right) \\
& \sim N(0, \sigF) + N(0, (i-1)\sigS\dT^{2}) \\
& \sim N(0, \sigF + (i-1)\sigS\dT^2)  
\end{align}
Similarly for the phase 
\begin{align}
\Delta \phi_{i} - \Delta \phi_{i-1} = &  \left(\s{j=1}{i}\tn \phi_{j} 
-  \s{j=1}{i-1}\tn \phi_{j}\right) + 2\pi\left(\s{j=1}{i-1}(i-j)\tn \f_{j}  
-  \s{j=1}{i-2}(i-1-j)\tn \f_{k} \right)\dT \notag \\
& + \pi\left(\s{j=1}{i-1}(i-j)^{2}\tn \fdot_{j} 
- \s{j=1}{i-2}(i-1-j)^{2}\tn \fdot_{j}\right)\dT^{2} \\
= &  \tn \phi_{i} +  2\pi \s{j=1}{i-1}\tn\f_{j} 
+ \pi \s{j=1}{i-1}(2i - 2j -1)\fdot_{j}\dT^{2} \\
\sim &  N(0, \sigS) + \s{j=1}{i-1}N(0, \sigF) \dT 
+ \s{j=1}{i-1}(2i - 2j -1)N(0, \sigS) \dT^{2}  \\
\sim &  N\left(0, \sigS\right) + N\left(0, (i-1)\sigF\dT^{2}\right) 
+ N\left(0, \s{j=1}{i-1}(2i - 2j -1)\sigS\dT^{4}\right)  \\
\sim &  N\left(0, \sigS\right) + N\left(0, (i-1)\sigF\dT^{2}\right) 
+ N\left(0, (i-1)^{2}\sigS\dT^{4}\right)  \\
\Delta \phi_{i} - \Delta \phi_{i-1}  \sim & N\left(0, \sigS 
+ (i-1)\sigF\dT^{2} + (i-1)^{2}\sigS\dT^{4}\right)
\end{align}

This result indicates that the integrated effect of random walks in higher
order terms will create a normal distribution in the monthly ephemeris. The
standard deviation will
depend on the observation time. 


