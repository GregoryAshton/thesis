\documentclass[../full_thesis/full_thesis.tex]{subfiles}

% Default image directory
\newcommand{\thisdir}{../analytic_timing_noise_cgw}
\graphicspath{{\thisdir/img/}}

\begin{document}

Continuous gravitational waves may be emitted from a variety of sources and
often different sources require different types of searches.  Already we have
discussed the search for GWs from an isolated neutron star which emits due to a
'mountain' in its crust. In addition to these searches, efforts are made to
detect CWs from LMXBs \citep{ligo2015scox1} and supernova remnants
\citep{ligo_SNR2015}. All of these searches (except some targetted searches)
assume the GW signal is smooth, either as a smooth spin-down Taylor expansion,
or in the case of LMXBs a constant frequency signal. However, timing-variations
may violate this assumption and lead to a loss of SNR in the detection process.
In Chapter.~\ref{sec: timing noise in cgw} we used a realisation of
timing-noise from the Crab to estimate the effect of timing noise in
fully-coherent searches. In this chapter, we will calculate the mismatch
analytically for random walk models of timing noise. We will do this using the
generalised metric mismatch defined in Sec.~\ref{sec: generalising the
metric-mismatch}. Already we have used this technique to calculate the mismatch
due to glitches in Sec.~\eqref{sec: mismatch due to glitches}. 

In Sec.~\ref{sec: TN interpretations random walk models} we described the
random walk model for timing noise. Recent observations \citep{Hobbs2010}
suggest that such a description does not capture the physics of timing noise in
isolated radio pulsars and hence is not a useful way to infer neutron star
physics. However, empirically the timing residuals of many pulsars are
indistinct from a random walk, so while it cannot help with inference, it is a
useful description if we only need to know what timing noise `looks like'. With
this in mind, in this section we consider a simplistic random walk model of
timing noise. This can be applied to searches for GWs from isolated neutron
stars, but also to to searches for GWs from LMXBs where fluctuations in the
spin-down torque may cause a random walk around some fixed frequency.

To calculate the mismatch, we will model the random walk as a zero-mean
gaussian walk in the phase, frequency, and spin-down which occurs at $N$ fixed
time intervals, $\Delta T$. Choosing fixed time intervals appears to introduce
an additional timescale not usually present in random walk models. However, as
discussed in Sec.~\ref{sec: physical interpretation of the monthly
ephemeris} this is consistent with a large number of unresolved events which
are measured over a fixed time scale.

\section{Defining a random walk}
\label{sec: Defining a random walk}
We allow the spin-down, frequency, and phase to undergo a random walk such that
each step proceeds from the previous value and the steps themselves are
normally distributed. Constraining the spin-down to be the highest order term
allowed to vary we have
\begin{equation}
\Delta \fdot_{i} - \Delta\fdot_{i-1} = \tn \fdot_{i} \sim N(0, \sigS)
\end{equation}•
Notice that the residual between parameter space offsets is denoted by $\tn$
which is always normally distributed. Rearranging this gives an expression for
the offset in the $i^{th}$ segment, by induction we can also write down the
$i-1$ term
\begin{align}
\Delta\fdot_{i} &  = \tn\fdot_{i} + \Delta\fdot_{i-1}  \\
\Delta\fdot_{i-1} &  = \tn\fdot_{i-1} + \Delta\fdot_{i-2}  .
\end{align}
As each step proceeds from the previous step, and the random walk begins at the
origin, we can write
\begin{equation} 
\Delta\fdot_{i} = \s{j=1}{i}\tn\fdot_{j}.
\label{eqn: fdot offset} 
\end{equation}

If we want to build a realistic model, then we must consider the effect that a
random walk in spin-down may have on the frequency and phase. For example if we
increase the spin-down for a period of time, then we would expect the frequency
to decrease at a greater rate during this period. In our discreet model it is
not possible to dynamically change the frequency during a single segment.
However, we can approximate this by updating the frequency in the next
segment with the induced frequency offset due to the spin-down in previous
segments. This must be done for lower order terms for each random walk in the
frequency and spin-down. For the phase no lower order terms exist so there is no
induced effect.

Because the random walk is discreet and constant in any given segment, we can
calculate the offset in the lower order terms from a Taylor expansion. The
total offset at the  $i^{th}$ reference time is then given by the summation of
the offset caused by higher order terms up to that reference time. The reference times
can be arbitrarily chosen, but setting each to start at the beginning of the segment
simplifies the calculation.
In figure~\ref{fig: Illustration fdot int} we plot the random walk in spin-down as given by 
\eqref{eqn: fdot offset}. The frequency offset induced by the spin-down can be
calculated using a Taylor expansion
\begin{equation}
\Delta \f_{i} = \s{j=1}{i-1}\Delta\fdot_{j} \dT 
\label{eqn: f offset induced} 
\end{equation}

This can be though of as the integration of the spin-down up to the $i^{th}$
reference time and is illustrated by the shaded region in figure~\ref{fig:
Illustration fdot int}. 

\begin{figure}[ht]
\centering
\includegraphics[width=0.7\textwidth]{Illustration_F1_int}
\caption{Illustration of equation~\eqref{eqn: f offset induced} demonstrating
a random walk in the spin-down and the filled in region which we
integrate over to find the frequency offset at $t_{i}$.}
\label{fig: Illustration fdot int}
\end{figure}

Since we want to consider random walks in all three parameters we now add in a
random walk in frequency. Each step is independent of the induced effect from
the spin-down and is given by \mbox{$\tn \f_{i} \sim N(0, \sigF)$}. The two
effects will sum linearly such that the frequency offset is
\begin{equation}
\Delta \f_{i} = \s{j=1}{i}\tn \f_{j} + \s{j=1}{i-1}\Delta\fdot_{j} \dT.
\label{eqn: f offset} 
\end{equation}•

By a similar process we can calculate the induced effect of the frequency and
spin-down on the phase. In full the phase offset is given by
\begin{equation}
\Delta\phi_{i}  =  \s{j=1}{i}\tn \phi_{j} 
+ 2\pi\left(\s{j=1}{i-1}\Delta \f_{j}\dT 
+ \frac{1}{2}\s{j=1}{i-1}\Delta \fdot_{j}\dT^{2}\right) \label{eqn: phi offset} 
\end{equation}

%Equations \eqref{eqn: f offset} and \eqref{eqn: phi offset} will allows us to
%construct the parameter space offsets in all three terms from the distibutions
%$\tn \phi_i$, $\tn \phi_i$, and $\tn \phi_i$.




\section{Random walk models part I}
\label{sec: Random walk models part I}
We will now calculat the mismatch for a fully-coherent search given the
random walk in phase, frequency, and spin-down rate defined in the previous
section.

Let us begin by expanding the metric-mismatch summation from Eqn.~\eqref{eqn:
mismatch}.  Writing the summations explicitly, we have
\begin{align}
m & = g_{\alpha\beta i j}\dl^{\alpha i}\dl^{\beta j}  \\
&=\s{i=1}{N}\s{j=1}{N}g_{\alpha\beta i j}\dl^{\alpha i}\dl^{\beta j}  \\
&= \s{i=1}{N}g_{\alpha\beta i i}\dl^{\alpha i}\dl^{\beta i}
+ \s{i=1}{N} \s{\substack{j=1\\ j \ne i}}{N} g_{\alpha \beta ij}\dl^{\alpha i}\dl^{\beta j}.
\end{align}
The summation has been intentionally split into terms for which the two
segments are the same and those for which they are different. The metric when
the reference time is at the beginning of each segment is given by equation
\eqref{eqn: metric equal segments tref 0}. By considering the metric for the
two cases, we can write the two distinct components as
\begin{equation}
g_{\alpha\beta ij} = \left\{
\begin{array}{cc}
g_{\alpha\beta}^{\mathrm{E}} & \textrm{ if } i =j \\
g_{\alpha\beta}^{\mathrm{NE}} & \textrm{ if } i  \ne j
\end{array}\right.  .
\end{equation}
Then the mismatch can be calculated from
\begin{align}
m &= \s{i=1}{N}g_{\alpha\beta}^{\mathrm{E}}\dl^{\alpha i}\dl^{\beta i}
+ 2\s{i=1}{N} \s{j=1}{i-1} g_{\alpha \beta}^{\mathrm{NE}}\dl^{\alpha i}\dl^{\beta j} .
\label{eqn: mismatch sep}
\end{align}

\subsubsection{Writing the parameter offsets in terms of normal distributions}
Equations~\eqref{eqn: f offset} and \eqref{eqn: phi
offset} give the offsets as functions of the offsets in higher order
parameters. In order to calculate statistical values, we now write these in terms of
the normal distributions from which the random walks are constructed.
Substituting Eqn.~\eqref{eqn: fdot
offset} into Eqn.~\eqref{eqn: f offset} and using the summation properties defined
in Appendix~\ref{sec: summation identities}, we have
\begin{align}
\Delta \f_{i}  & = \s{j=1}{i}\tn \f_{j}
+ \s{j=1}{i-1}\s{k=1}{j}\tn \fdot_{k} \dT ,  \\
& = \s{j=1}{i}\tn \f_{j}
+ \s{j=1}{i-1}(i-j)\tn \fdot_{j} \dT .
\label{eqn: delta f n}
\end{align}
Similarly, substituting this equation into Eqn.~\eqref{eqn: phi offset} we
have
\begin{align}
\begin{split}
\Delta\phi_{i} & = \s{j=1}{i}\tn \phi_{j}
+ 2\pi \left(\s{j=1}{i-1}\Delta\f_{j}\dT
+ \frac{1}{2}\s{j=1}{i-1}\Delta\fdot_{j}\dT^{2}\right) \\
& = \s{j=1}{i}\tn \phi_{j} + 2\pi\left(\s{j=1}{i-1}\left(\s{k=1}{j}\tn\f_{k}
+ \s{k=1}{j-1}(j-k)\tn\fdot_{k}\dT\right)\dT
 + \frac{1}{2}\s{j=1}{i-1}\s{k=1}{j}\Delta\fdot_{k}\dT^{2} \right)  \\
& = \s{j=1}{i}\tn \phi_{j} + 2\pi\left(\s{j=1}{i-1}(i-j)\tn\f_{j}\dT
 + \s{j=1}{i-1}\s{k=1}{j-1}(j-k)\tn\fdot_{k}\dT^{2}
 + \frac{1}{2}\s{j=1}{i-1}(i-j)\Delta\fdot_{j}\dT^{2} \right)  \\
& = \s{j=1}{i}\tn \phi_{j} + 2\pi\left(\s{j=1}{i-1}(i-j)\tn\f_{j}\dT
 + \frac{1}{2}\s{j=1}{i-1}\left(\left(i-j\right)\left(i-j-1)\right)
 + (i-j)\right)\tn\fdot_{j}\dT^{2}\right)  \\
& = \s{j=1}{i}\tn \phi_{j} + 2\pi\left(\s{j=1}{i-1}(i-j)\tn\f_{j}\dT
 + \frac{1}{2}\s{j=1}{i-1}(i-j)^{2}\tn\fdot_{j}\dT^{2}\right)
\end{split}
\label{eqn: delta phi n}
\end{align}

\subsubsection{Taking the expectation}

In Eqn.~\eqref{eqn: delta fdot n}, Eqn.~\eqref{eqn: delta f n},
Eqn.~\eqref{eqn: delta phi n} we have written the parameter space offsets
(which are to be used in calculating the mismatch) purely in terms of the
random walk distributions $\tn \phi_i$, $\tn \f_i$, and $\tn \fdot_i$. We can
calculate the mismatch exactly given a set of random walk jumps by inserting
these into Eqn.~\eqref{eqn: mismatch sep}. However, since we are dealing with
statistical quantities, we can instead infer the behaviour of the mismatch
under the random walk by taking an expectation.

Inseting Eqn.~\eqref{eqn: delta fdot n}, Eqn.~\eqref{eqn: delta f n},
Eqn.~\eqref{eqn: delta phi n}  in Eqn.~\eqref{eqn: mismatch sep} yields a
number of terms with all the permutations of two terms from $[\tn \phi, \tn \f,
\tn \fdot]$. Taking the expection, all the cross-correlated terms, such as
$\tn\phi_{i}\tn \fdot$, will have an expectation of zero since the steps of the
random walk are independent. The only non-vanishing terms are given by
\begin{align}
E[\tn\phi_{i}\tn\phi_{j}] &= \delta_{ij}\sigP, &
E[\tn\f_{i}\tn\f_{j}] &= \delta_{ij}\sigF,&
E[\tn\fdot_{i}\tn\fdot_{j}] &= \delta_{ij}\sigS,
\end{align}
After some simplification we find that the mismatch is given by
\begin{align}
\begin{split}
E[m]   = &  \frac{A_{\phi}}{6} \left(N - \frac{1}{N}\right)
+ \frac{\pi^{2} A_{{f}}}{30}\left(4 N^{3} + 5 N^{2} + \frac{1}{N}\right)\\
 & +  \frac{\pi^{2} A_{{\dot{f}}}}{3780} \left(66 N^{5} - 21 N^{3} + 105 N^{2}
 + 217 N + 63 - \frac{94}{N}\right),
\end{split}
\label{eqn: expectation}
\end{align}
where
\begin{equation}
	A_{\phi} = \sigP \;\;\;\;\;
    A_{\f} = \sigF\Delta T^{2} \;\;\;\;\;
    A_{\fdot} = \sigF\Delta T^{4},
\end{equation}
define three `activity parameters'.

\subsubsection{Verifying the results}
Recalling that $N=\Tobs/\Delta T$, Eqn.~\eqref{eqn: expectation} makes
predictions for the leading order
scaling of the three random walks with observation period
\begin{equation}
E[m]_{PN} \sim \sigP \Tobs, \hspace{10mm}
E[m]_{FN} \sim \sigF \Tobs^{3}, \hspace{10mm}
E[m]_{FN} \sim \sigS \Tobs^{5}.
\end{equation}

We can observe this behaviour directly and verify the predictions made by
Eqn.~\eqref{eqn: expectation} by comparing with exact numerical results. That
is, using the signal injection and recovery tools developed in Sec.~\ref{sec:
narrow-band method} of Chapter.~\ref{sec: timing noise in cgw} we simulate
signals undergoing a random walk and calculate the corresponding mismatch (no
minimisation step is done here, this is discussed in the next section). In
particular, we perform three Monte Carlo studies for a random walk in the
phase, frequency, and spin-down rate and in each case compre the simulated
results with the analytic prediction. The results are shown in Fig.~\ref{fig: rw I}
and demonstrate good agreement between the simulation means and the prediction
of Eqn.~\eqref{eqn: expectation}.

\begin{figure}[ht]
\centering
\subfloat[Random walk in phase]{\includegraphics[width=0.5\textwidth]{ExpectationPhase}}
\subfloat[Random walk in frequency]{\includegraphics[width=0.5\textwidth]{ExpectationFrequency}}\\ \subfloat[Random walk in spin-down]{\includegraphics[width=0.5\textwidth]{ExpectationSpindown}}
\caption{A comparison of Monte Carlo numerical simulated mismatch with the prediction
of Eqn.~\eqref{eqn: expectation} for a random walk in the phase, frequency,
and spin-down rate.}
\label{fig: rw I}
\end{figure}

\subsubsection{Implied scaling}

Eqn.~\eqref{eqn: expectation} predicts the variation in mismatch with
observation time due to a random walk in the parameters of the signal. The
random walk occurs at fixed intervals of $\dT$ and in three parameters, as such
the mismatch is a function of all these variables. This method differs from
results in the literature studying timing noise \citep{Cordes1981} where the
steps are Poisson distributed in time with an average rate. The choice of a
fixed $\dT$ is necceciated by the ease of calculation for regular intervals,
but is consisent with the description of timing noise provided by the Crab
ephemeris in Sec.~\ref{sec: timing noise in cgw}.  For the expectation of the
mismatch to be physical, we should expect that, at least to leading order, it
is invariant to a suitable combination of $\dT$ and $\sigma$. Equivalently we
imagine that if the Crab ephemeris was updated every half a month instead of
once a month, we should measure the same mismatch for the same observation
period.

To leading order, for the expected mismatch to be invariant to changes in $\dT$
we require
\begin{equation}
\sigP \sim \dT \;\;\;\;\; \sigF \sim \dT \;\;\;\;\; \sigS \sim \dT
\end{equation}
Importantly this result agrees with the description of a Compound Poisson
process in the limit where the many events occur during the observation period
$\dT$.
\meta{Greg: not sure about this claim}


\section{Random walk models part II} 
\label{sec: Random walk models part II}
%Searches for CW require a choice of parameters for the global template. For
%narrow band searches we can use EM data to estimate the best choice of
%parameters.  These are found by least-squares fitting a Taylor expansion to the
%observed pulse TOAs, the resulting coefficients are the best fit parameters.
%Peforming a narrow band search for a signal using matched filtering, we
%identify a candidate as a local minimum in mismatch in parameter space. Finding
%such a local minimum can be thought of as a minimisation problem of the
%mismatch with respect to the parameters.  Using the $\mathcal{F}$-statistic
%matched filtering algorithm the phase is analytically minimised by definition.
%Each  parameter that is searched over adds an additional term in the Taylor
%expansion to minimise over.  So searching over $\f$ and $\fdot$ is equivalent
%to analytically minimising with a second order polynomial. To explain this
%better we will use a first look at a generic toy model.

We have a description of a random walk from which we can calculate the mismatch
as in section \ref{sec: Random walk models part I}. However, this is a special
case in which the random walk for each parameter offsets begins at the origin
and then grows with time. The parameter offsets, defined in equation
\eqref{eqn: parameter space offsets}, depend both on the signal and the
template $\bdl_{0}$.  It is the signal which undergoes a random walk, we choose
the template such that the parameter offset was zero at the origin. We can
imagine instead choosing a different template, this will result in a different
set of parameter offsets and hence mismatch. Our original choice of template 
does not attempt to minimise the mismatch, it is therefore probable that
smaller mismatched can be achieved by varying the template. This is precisely
what is done in a narrow band search. We can think of such a search as a 
minimisation of the mismatch for a random walk.

To understand this concept better we now consider the residual from a least squares
minimisation of a toy model random walk starting at the origin. The system
described in section \ref{sec: Random walk models part I} is the random walk
starting at the origin - subtracting the least squares minimisation corresponds
to finding the minimum mismatch between a global Taylor expansion and the
random walk signal.


\subsubsection{Least-squares minimisation of a random walk}
\label{sec: Least-squares minimisation of a random walk}
In this appendix, we will describe the process of fitting and
removing a polynomial from $N$ data points $(x_i, y_i)$ which undergoes a
random walk. The polynomial will be fitted using a least squares minimisation.
The $x_i$ are the independent points at which $y_i$ (which undergoes a random
walk) is measured. We begin by defining the least-squares fitting method then
go on to calculate the residual for several different degrees of polynomial.
This introduces the method in a generic setting which is then applied in
Section~\ref{sec: random walk models part II} to calculate the mismatch for a
GW signal which undergoes a random walk, but in which the search minimises the
mismatch over the search frequency and frequency derivative.

\subsection{Least squares fitting of a polynomial}
Given $N$ data points $x_{i}$, $y_{i}$, we define the residual from a least-squares
polynomial fit of order $k$, as
\begin{equation}
    r_i^{(k)} = y_{i} - y^{\textrm{(k)}}_{i},
\end{equation}
where
\begin{equation}
y^{\textrm{(k)}}_{i} = a_{0} + a_{1}x_{i} + a_{2}x_{i}^{2} + \dots
                                                           + a_{k} x_{i}^{k},
\end{equation}
is a polynomial of degree $k$.

Then the residual which we want to minimise is
\begin{equation}
R^{2} = \s{i=1}{N}\left(r_i^{(k)}\right)^{2}
      = \s{i=1}{N}\left(y_{i} - \left(a_{0} + a_{1}x_{i} + a_{2}x_{i}^{2} +
        \dots + a_{k} x_{i}^{k}\right)\right)^{2}.
\end{equation}
Partial differentiation with respect to the parameters $a_{i}$, yields $k$
simultaneous equations. Writing these as a matrix and then solving
for the best fit, $\hat{y}^(k)_i$, it can be shown \citep{WolframLeastSquares} that
\begin{align}
\hat{y}^{\textrm{(k)}}_{i} & = X \left(X^{T}X\right)^{-1} X^{T} y_{i} & \textrm{where} & &
X & = \left[\begin{array}{ccccc}
1 & x_{1} & x_{1}^{2} & \dots & x_{1}^{k} \\
1 & x_{2} & x_{2}^{2} & \dots & x_{2}^{k} \\
\vdots & \vdots & \vdots & \vdots & \vdots \\
1 & x_{n} & x_{n}^{2} & \dots & x_{n}^{k} \\
\end{array}\right]
\end{align}
Here $X$ is an example of a \emph{Vandermonde} matrix in which the terms follow
a geometric progression. It is useful to note that
\begin{equation}
XX^{T} = \left[\begin{array}{cccc}
N & \s{i=1}{N}x_{i} & \cdots &  \s{i=1}{N}x_{i}^{k} \\
\s{i=1}{N}x_{i} & \s{i=1}{N}x_{i}^{2} & \cdots &  \s{i=1}{N}x_{i}^{k+1} \\
\vdots & \vdots & \ddots & \vdots \\
\s{i=1}{N}x_{i}^{k} & \s{i=1}{N}x_{i}^{k+1} & \cdots &  \s{i=1}{N}x_{i}^{2k}
\end{array}\right].
\end{equation}

Provided that the $x_{i}$ are suitably defined, then an analytic fit can be
found for any $k$, the difficulty lies in inverting the matrix.

\subsection{Least squares fitting a polynomial to a random walk} We now take
the $x_i, y_i$ to be a Gaussian random walk beginning at the origin. To define
this, let $\delta y_{i} \sim N(0, \sigma^{2})$ be independent and identitically
distributed random variables for which their sum generates the random walk:
\begin{equation}
y_{i} = \sum_{j=1}^{i}\delta y_{i}.
\label{eqn: ToyModel RW definition}
\end{equation}
We also set each random walk event to occur according to $x_{i} = i \Delta x$.
Then the residual after fitting and removing a $k^{th}$ order polynomial to the
random walk $y_i$, is
\begin{equation}
r_i^{(k)} = y_{i} - \hat{y}_{i}^{(k)} = y_{i} - X \left(X^{T}X\right)^{-1} X^{T} y_{i}.
\label{eqn: fitted residual}
\end{equation}
This suggests the residual will be similar to the random walk, but modified by
the least squares fitting.  To illustrate this, in Figure~\ref{fig: ToyModelRW}
we plot a simulated random walk along with several fits.
\begin{figure}[htb]
\centering
\includegraphics[width=.9\textwidth]{ToyModelRW}
\caption{Example of a random walk on the left along with three polynomial fits
of varying order. On the right is the corresponding residual after subtracting
these fits. A dotted line marks the origin in both plots.}
\label{fig: ToyModelRW}
\end{figure}

\subsection{Zeroth order fitting}

We begin with the case of $k=0$ in which $X^{T} = [1, 1, \dots 1]$ such
that
\begin{equation}
X \left(X^{T}X\right)^{-1} X^{T} = \frac{1}{N} J_{N}
\end{equation}
where $J_{N}$ is the $N\times N$ matrix of ones.  Inserting this into
Eqn.~\eqref{eqn: fitted residual}, the residual from a zeroth order fit is
given by
\begin{equation}
r_i^{(0)}= y_{i} - \frac{1}{N} \s{j=1}{N}y_{j}.
\end{equation}
The zeroth order residual can be interpetted as the removing the
average value $\langle y_i \rangle$ from the random walk: this was illustrated
in Figure~\ref{fig: ToyModelRW}.

%For example the
%expectation after $i$ steps of the original random walk can be shown to be
%zero, therefore the expectation for the zeroth order residual after $i$ steps
%will also be zero.

%This can intuitively be understood from the fact that we started out RW at the origin,
%a zeroth order fit shifts the origin but a random walk should

We can now take expectations to understand the behaviour of the residual when
compared to the original definition of the random
walk in Eqn.~\ref{eqn: ToyModel RW definition}. For example, consider
the mean square translation distance from the origin of a random walk after $i$
steps. For a normal random walk, this has the well known result
\begin{equation}
E[y_{i}^{2}] = i \sigma^{2}.
\label{eqn: RW classic}
\end{equation}
We can calculate the corresponding quantity of the $k=0$ residual by first
noting that
\begin{align}
E\left[y_{i}y_{j}\right] & = E\left[\s{k=1}{i}\delta y_{k} \s{l=1}{j}\delta y_{l} \right] \\
& = \s{k=1}{i}\s{l=1}{j}E\left[\delta y_{k} \delta y_{l}\right] \\
& = \s{k=1}{i}\s{l=1}{j} \delta_{kl} \sigma^{2} \\
& = \sigma^{2}\min(i, j),
\label{eqn: E yiyi}
\end{align}
where $\delta_{kl}$ is the Kronecker delta. Then we have
\begin{align}
\left(r^{(0)}_{i}\right)^{2} & = y_{i}^{2} - \frac{2}{N}\s{k=1}{N}y_{i}y_{k} + N^{-2}\s{k=1}{N}\s{l=1}{N}y_{k}y_{l} \\
& =  y_{i}^{2} - 2 N^{-1} \left(\s{k=1}{i}y_{i}y_{k} + \s{k=i+1}{N}y_{i}y_{k} \right)+ N^{-2}\s{k=1}{N}\left(\s{l=1}{k}y_{k}y_{l} + \s{l=k+1}{N}y_{k}y_{l} \right).
\end{align}
Taking the expectation we have
\begin{align}
E\left[\left(r^{(0)}_{i}\right)^{2} \right] & = \sigma^{2}\left(i - \frac{2}{N}\left(\s{k=1}{i}k + \s{k=i+1}{N}i \right)+ \frac{1}{N^{2}}\s{k=1}{N}\left(\s{l=1}{k}l+ \s{l=k+1}{N}k \right) \right) \\
& = \sigma^{2}\left(\frac{N}{3} - i + \frac{1}{2} + \frac{i^{2}}{N} - \frac{i}{N} + \frac{1}{6 N}\right).
\end{align}
This result can be compared with Eqn.~\eqref{eqn: RW classic}, the expectation
of the squared value for a random walk.  In contrast, the expectation after $i$
steps for the residual random walk depends on the length of data $N$ that was
fitted. It can be shown the expectation has a minimum at $i=N/2$.

To further understand the difference between the random walk and
the residual random walk, let us consider the sum of squares after $N$ steps for
the random walk
\begin{equation}
E\left[\s{i=1}{N} y_{i}^{2}\right] = \s{i=1}{N} i \sigma^{2} =
                               \frac{1}{2}\left(N^{2} + N\right)\sigma^{2}.
\label{eqn: sum of squares}
\end{equation}
On the other hand, the sum of squares for the residual random walk is given by
\begin{equation}
E\left[\s{i=1}{N} \left(r^{(0)}_{i}\right)^{2}\right] = 
\frac{1}{6}\left(N^{2} -1\right)\sigma^{2}.
\label{eqn: sum of squares k0}
\end{equation}
Comparing equations \eqref{eqn: sum of squares} and \eqref{eqn: sum of squares
k0} we note that, for the leading order term, the coefficient is reduced, but
the power remains the same.

%We verify this behaviour with a simple script that produces a random walk of
%length $N$ then fits and subtracts a $0^{th}$ order polynimial; we then
%calculate the sum of the square residual. In Figure~\ref{fig:
%sum_of_squares_res_oth_order} we repeat this operation multiple times then plot
%the average of the sum of squares for the residual while varying $N$, the
%prediction of Eqn.~\eqref{eqn: sum of squares k0} is also plotted showing
%agreement.
%
%\begin{figure}[ht]
%\centering
%\includegraphics[width=.6\textwidth]{sum_of_squares_res_oth_order}
%\caption{Comparing Eqn.~\eqref{eqn: sum of squares k0} with the averaged sum of squares for a simulated random walk }
%\label{fig: sum_of_squares_res_oth_order}
%\end{figure}

\subsection{First order fitting}
We now consider a first order fitting for which
\begin{align}
\hat{y}^{(1)}_{i} & = X\left(X^{T}X\right)^{-1} X^{T} y_{i} & \textrm{with} &&
X & = \left[\begin{array}{cc}
1 & \Delta x \\
1 & 2 \Delta x  \\
\vdots & \vdots  \\
1 & N \Delta x  \\
\end{array}\right].
\end{align}
Inserting the definitions of $x_{i}$ we can write
\begin{equation}
\left(X^{T}X\right)^{-1} = \frac{1}{N(N-1)}\left[
\begin{array}{cc}
4N+2 & -\frac{6}{\Delta x} \\
 -\frac{6}{\Delta x} & \frac{12}{\Delta x^{2} (N+1)}
\end{array}•
\right] = \Cone.
\end{equation}
For convenience we have defined a symmetric matrix $\Cone$. We then proceed to
define another matrix
\begin{align}
    \mathcal{C}_{ij}^{(1)} & := X\Cone X^{T} \\  & =
\left[\begin{array}{cc}
1 & \Delta x \\
1 & 2\Delta x  \\
\vdots & \vdots  \\
1 & N \Delta x \\
\end{array}\right]
\left[\begin{array}{cc} \Cone_{11} & \Cone_{12} \\ \Cone_{21} & \Cone_{22} \end{array}\right]
\left[\begin{array}{cccc}
1 & 1 & \dots & 1 \\
\Delta x & 2\Delta x & \dots  & N \Delta x
\end{array}\right] \\
& =
\Cone_{11} J_{N} +
\Cone_{12} \Delta x \left[ \begin{array}{cccc}
2 & 3 & \dots & N+1 \\ 3 & 4 & \dots & \vdots \\ \vdots & & & \\  N+1& \dots & \dots & 2N
\end{array}\right] +
\Cone_{22} \Delta x^{2} \left[ \begin{array}{cccc}
1 & 2 & \dots & N \\ 2 & 4 & \dots & \vdots \\ \vdots & & & \\  N& \dots & \dots & N^{2}
\end{array}\right]
\end{align}
We can write $r^{(1)}_{i}$ as a summation by inferring the dependence of the
$i^{th}$ row of each matrix on the $j^{th}$ column
\begin{align}
r^{(1)}_{i} & = y_{i} - \s{j=1}{N}\mathcal{C}_{ij}^{(1)} y_{j}
& \textrm{ where} &&
\mathcal{C}_{ij}^{(1)} & = \Cone_{11} + \Cone_{12}\Delta x (i+j) + \Cone_{22}\Delta x^{2} ij
\end{align}

We have now defined the first order residual. To understand that fitting an
removing a first order polynomial has, we compute the expectation of the
square for the $i^{th}$ term
\begin{align}
\begin{split}
E\left[\left(r^{(1)}_{i}\right)^{2}\right]  =
\frac{1}{15 N \left(N^{2} - 1\right)} &
\left(2 N^{4} - 18 N^{3} i + 9 N^{3} + 78 N^{2} i^{2} - 78 N^{2} i \right.\\
      &\hspace{3mm} + 14 N^{2} - 120 N i^{3} + 180 N i^{2} - 78 N i \\
      &\hspace{3mm} \left. + 9 N + 60 i^{4} - 120 i^{3} + 78 i^{2} - 18 i + 2\right),
\end{split}
\end{align}
which can be compared with the classic result for a random walk given in
Eqn.~\eqref{eqn: RW classic}. Alternatively, comparing with Eqn.~\eqref{eqn:
sum of squares}, the expected sum of squares for the residual random walk is
\begin{equation}
E\left[\s{i=1}{N} \left(r^{(1)}_{i}\right)^{2}\right] 
= \frac{1}{15}\left(N^{2} -4\right)\sigma^{2},
\label{eqn: sum of squares k1}
\end{equation}
as in the zeroth order fit, the power of the leading order term remains the
same, but the coefficient decreases.

\subsection{Second order fitting}
For the residual left after removing a quadratic, the  argument proceeds in much
the same way with
\begin{align}
r^{(2)}_{i} & = y_{i} - \s{j=1}{N}\mathcal{C}_{ij}^{(2)} y_{j},
\end{align}
where
\begin{align}
\mathcal{C}_{ij}^{(2)}  = &\Ctwo_{11} + \Ctwo_{22}\Delta x^{2}ij +
\Ctwo_{33}\Delta x^{4}i^{2}j^{2} \nonumber + \Ctwo_{12}\Delta x(i+j) \nonumber \\
& + \Ctwo_{13}\Delta x (i^{2} + j^2) + \Ctwo_{23}\Delta x^{3}(ij^{2} + i^{2}j),
\label{eqn: MC_2}
\end{align}
and
\begin{equation}
C^{(2)} = \frac{1}{N(N-1)(N-2)}
\left[\begin{matrix}9 N^{2} + 9 N + 6 & - \frac{1}{\Delta{x}} \left(36 N + 18\right) & \frac{30}{\Delta{x}^{2}}\\- \frac{1}{\Delta{x}} \left(36 N + 18\right) & \frac{12 \left(2 N + 1\right) \left(8 N + 11\right)}{\Delta{x}^{2} \left(N + 1\right) \left(N + 2\right)} & - \frac{180}{\Delta{x}^{3} \left(N + 2\right)}\\\frac{30}{\Delta{x}^{2}} & - \frac{180}{\Delta{x}^{3} \left(N + 2\right)} & \frac{180}{\Delta{x}^{4} \left(N + 1\right) \left(N + 2\right)}\end{matrix}\right].
\label{eqn: C_2}
\end{equation}

The expression for the expected square value is too long to write out in full,
but the expected sum of squares for the residual random walk is
\begin{equation}
E\left[\s{i=1}{N} \left(r^{(2)}_{i}\right)^{2}\right] =
 \frac{1}{70}\left(3N^{2} -27\right)\sigma^{2},
\label{eqn: sum of squares k2}
\end{equation}
for which as in the case of the zeroth order and first order residuals, the
power of the leading order term remains unchanged, but the coefficient
decreases.

\subsection{Conclusions}
\label{sec: appendix conclusions}
We now have a method to calculate statistical quantities from the residual
left over after subtracting a $k^{th}$ order polynomial from a random walk.
Considering the sum of squares for a random walk and the residuals in equations
\eqref{eqn: sum of squares}, \eqref{eqn: sum of squares k0},
\eqref{eqn: sum of squares k1}, and \eqref{eqn: sum of squares k2} we find that
the leading order term retains the same
power of $N$ with increasing $k$ but the coefficient of this power gets
smaller. This reflects the improved fitting with the polynomial degrees. We
also note that with each increase in the order of fit we get a limit
on $N$ for which the sum of squares is positive. For zeroth order fitting this
is $N>1$, for first order $N>2$ and for second order $N>3$. This is because in
order to perform a least squares fit, we need at least $k+1$ points to fit.


\subsubsection{Least squares minimisation for a CW random walk}
\label{sec: Least squares minimisation for a CW random walk}

In the random walk studied in section \ref{sec: Random walk models part I} we
set the initial offset in all three parameters to zero and then allow the
random walk to wander from this point. In terms of the random walk signal, this
is the equivalent of selecting a global template such that the mismatch is
initially zero but grows with time. Clearly this is a poor choice for the
global template if you want to minimise the mismatch. The results in section
\ref{sec: Random walk models part I} are a worst case scenario where the we
minimised the mismatch only for the first segment.  This is far from optimal,
we now investigate how we can improve this.

We will model the parameter offsets $\Delta\phi_{i}, \Delta \f_{i}$ 
following the definitions in section \ref{sec: Defining a random walk}. 
Searching over a narrow band in $\f$ and $\fdot$
is equivalent to minimising the mismatch up to a quadratic term in the Taylor
expansion. So the parameter space offsets with which we should calculate the
mismatch from, are given by
\begin{equation}
\Delta^{(2)}\lambda_{\alpha i} = \Delta\lambda_{\alpha i} - \Delta\lambda_{\alpha i}^{2}
\end{equation}•

The toy model random walk considered in section \ref{sec: Least-squares
minimisation of a random walk} was minimised with respect to the sum of squared
differences between the fit and the random walk. Ideally in calculating the mismatch
we should minimise $\Delta\lambda_{\alpha i}^{2}$ with respect to the
mismatch. This however adds an additional level of complexity. The minimisation
with respect to the sum of squared differences will not too different from the
minimisation with respect to the mismatch and so we will use this as an
approximation.

\paragraph{Random walk in the phase}
We begin with the simplest case of a random walk in phase for which we have
\begin{equation}
\Delta\phi_{i} = \s{j=1}{i}N(0, \sigP).
\end{equation}
Then we note that
\begin{equation}
E[\Delta\phi_{i} \Delta\phi_{j}] = \sigP \min(i, j)
\end{equation}
The mismatch for a random walk in the phase having removed a second order
fit is given by
\begin{align}
m & = g_{\alpha \beta i j} \Delta^{(2)}\lambda^{\alpha i}\Delta^{(2)}\lambda^{\beta j} \\ 
& = \s{i=1}{N}g_{00}^{E} \Delta^{(2)}\phi_{i}\Delta^{(2)}\phi_{i} 
+ 2 \s{i=1}{N}\s{j=1}{i-1}g_{00}^{NE}\Delta^{(2)}\phi_{i}\Delta^{(2)}\phi_{j}
\label{eqn: 4202540871}
\end{align}
To calculate the expectation of the mismatch, we need to evaluate the
expectation of
\begin{align}
\Delta^{(2)}\phi_{i}\Delta^{(2)}\phi_{i}& = \left(\Delta\phi_{i} 
- \s{k=1}{N}\CT_{ik}\Delta\phi_{k}\right)
 \left(\Delta\phi_{j} - \s{l=1}{N}\CT_{jl}\Delta\phi_{l}\right) \\
& = \Delta\phi_{i}\Delta\phi_{j} - 
\left(\s{k=1}{N}\CT_{ik} \Delta\phi_{j}\Delta\phi_{k} 
+ \s{l=1}{N}\CT_{jl}\Delta\phi_{i}\Delta\phi_{l}\right) + 
\s{k=1}{N}\s{l=1}{N}\CT_{ik}\CT_{jl} \Delta\phi_{k}\Delta\phi_{l}.
\end{align}•
Where we defined $\CT_{ij}$ in equations \ref{eqn: C_2} and \ref{eqn:  MCC_2}
and have replaced $\Delta x$ with the time $\dT$. Then taking the expectation
\begin{align}
\E{\Delta^{(2)}\phi_{i}\Delta^{(2)}\phi_{i}} & = 
E\left[\Delta\phi_{i}\Delta\phi_{j}\right] - 
\left(\s{k=1}{N}E[\Delta\phi_{j}\Delta\phi_{k}] 
+ \s{l=1}{N}E[\Delta\phi_{i}\Delta\phi_{l}]\right) + 
\s{k=1}{N}\s{l=1}{N}E[\Delta\phi_{k}\Delta\phi_{l}] 
    \label{eqn: expected mismatch k2}\\
& = \sigma^{2}_{\phi} \left(\min(i, j) - \left(\s{k=1}{N}\CT_{ik} \min(j, k) 
+ \s{l=1}{N}\CT_{jl}\min(i, l) \right)\right. \notag \\
& \hspace{50mm} \left. + \s{k=1}{N}\s{l=1}{N}\CT_{ik}\CT_{jl}\min(k, l)\right)
\label{eqn: expected mismatch dP0idP0j_k2}
\end{align}• 
Using symbolic mathematics packages we 
calculate an analytic expression which is a function of $\dT, i, j$ and $N$.
Inserting this into equation \eqref{eqn: 4202540871} and simplifying
\begin{align} 
E[m]  & = \s{i=1}{N}g_{00}^{E} E\left[\Delta^{(2)}\phi_{i}\Delta^{(2)}\phi_{i}\right] 
+ 2 \s{i=1}{N}\s{j=1}{i-1}g_{00}^{NE}E\left[\Delta^{(2)}\phi_{i}\Delta^{(2)}\phi_{j}\right]  \\
& = \frac{1}{70}\sigP\left(3N - \frac{27}{N}\right)
\label{eqn: Expected mismatch RW in phase k2}
\end{align}
This expression can be compared to equation \eqref{eqn: expectation} without
the higher order random walks. Notably we retain the same leading order scaling
of $N$ but the overall coefficient is decreased. Rearranging the expression in
the bracket demonstrates the mismatch is negative or zero for $1 \ge N \ge 3$;
This is a reflection of the minimum number of points needed in order to perform
the quadratic fit.

\paragraph{Random walk in the frequency}

For a random walk in the frequency we have an added complexity caused by the
effect the frequency offsets induces in the phase. For the frequency offset we 
have
\begin{align}
\Delta f_{i} &= \s{j=1}{i}N(0, \sigF).
\end{align}
Recalling that we set the reference time at the beginning of each segment, 
then as in section \ref{sec: Defining a random walk}, the induced phase offset is
\begin{align}
\Delta\phi_{i} &=2\pi \s{j=1}{i-1}f_{j}\dT \\
 & = 2\pi\dT \s{j=1}{i-1}\s{k=1}{j}N(0, \sigF) \\
& = 2\pi\dT \s{j=1}{i}(i-j)N(0, \sigF).
\end{align}
The expected values of combinations of these quantities we find to be
\begin{align}
E[\Delta\f_{i}\Delta\f_{j}] & = \sigF \min(i, j) \\
E[\Delta\phi_{i}\Delta\f_{j}] & = 2 \pi \dT \sigF \s{k=1}{\min(i, j)}(i-k)\\
E[\Delta\phi_{i}\Delta\phi_{j}] & = 
\left(2\pi\dT\right)^{2}\sigF \s{k=1}{\min(i, j)}(i-k)(j-k)
\label{eqn: Expectations raw}
\end{align}
Then calculating the expected mismatch due to a random walk in frequency 
\begin{align}
E[m] = & 
\s{i=1}{N}\left(g_{00}^{E}E\left[\Delta^{(2)}\phi_{i}\Delta^{(2)}\phi_{i}\right] 
+ 2 g_{01}^{E}E\left[\Delta^{(2)}\phi_{i}\Delta^{(2)}\f_{i}\right] 
+  g_{11}^{E} E\left[\Delta^{(2)}\f_{i}\Delta^{(2)}\f_{i}\right] \right) \\
& + 2\s{i=1}{N}\s{j=1}{i-1}\left(\right.
g_{00}^{NE}E\left[\Delta^{(2)}\phi_{i}\Delta^{(2)}\phi_{j}\right] + 
g_{01}^{NE}E\left[\Delta^{(2)}\phi_{j}\Delta^{(2)}\f_{i}\right] +  \\ 
&\hspace{20mm}\left. g_{10}^{NE}E\left[\Delta^{(2)}\phi_{i}\Delta^{(2)}\f_{j}\right] + 
g_{11}^{NE} E\left[\Delta^{(2)}\f_{i}\Delta^{(2)}\f_{j}\right] \right)
\end{align}
We calculate each of these expressions in a similar manner to equation
\eqref{eqn: expected mismatch dP0idP0j_k2} replacing the relevant expectations
with those given in equations \eqref{eqn: Expectations raw}. This yields an
expected mismatch given by 
\begin{equation}
E[m] = \frac{\pi^{2} }{630} \sigF \dT^{2}  \left(N^{3} - 7N- \frac{18}{N} \right).
\label{eqn: Expected mismatch RW in frequency k2}
\end{equation}
Again this can be compared with equation \eqref{eqn: mismatch} without the 
random walk in phase or spin-down. 
\paragraph{Verification}

We now verify equation~\eqref{eqn: Expected mismatch RW in frequency k2} and
\eqref{eqn: Expected mismatch RW in phase k2} by comparing with signal
injection recoveries. The injected signals undergo a random walk as in section
\ref{sec: Random walk models part I}, however, when searching for the signals we 
use a narrow band effectively minimising over the frequency and spin-down. The 
results are plotted in figure \ref{fig: verification of minimised RW} and 
demonstrate good agreement between the analytic and simulated mismatches.

\begin{figure}[ht]
\centering
\subfloat[Random walk in phase]{\includegraphics[width=0.5\textwidth]{ExpectationPhase_NarrowBand}} 
\subfloat[Random walk in frequency]{\includegraphics[width=0.5\textwidth]{ExpectationFrequency_NarrowBand}}\\
%\subfloat[Random walk in spin-down]{\includegraphics[width=0.5\textwidth]{ExpectationSpindown}}
\caption{Verification of equations \eqref{eqn: Expected mismatch RW in
frequency k2} and \eqref{eqn: Expected mismatch RW in phase k2} using signal
injection and recovery.}
\label{fig: verification of minimised RW}
\end{figure}•
\FloatBarrier


\begin{subappendices}
\section{Summation identities}\label{sec: summation identities}
\begin{align}
\s{b=1}{c}\s{c=1}{b} X_{c} & = \left( X_{1}\right) 
 + \left( X_{1} + X_{2} \right) + \ldots  +\left( X_{1} + X_{2} 
 + \ldots + X_{c-1} + X_{c}\right)\\
& = c X_{1} + (c-1)X_{2} + \ldots + 2 X_{c-1} + X_{c} \\
& = \s{b=1}{c}(c+1-b)X_{b}
\end{align}•

\begin{align}
\s{j=1}{i-1}\s{k=1}{j-1}(j-k)X_{k} = & [0] + \left[ X_{1}\right] 
+ \left[2X_{1} + X_{2} \right] + \left[3X_{1} + 2X_{2} +X_{3}\right] 
+ \ldots  \\
& + \left[\left(i-2\right)X_{1} + \left(i-3\right)X_{2} 
+ \left(i-4\right)X_{3} + \ldots + 3X_{i-4} + 2X_{i-3} + X_{i-1}\right]  \\
= & \left(1 + 2 + 3 + \ldots +  (i-4) + (i-3) + (i-2)\right)X_{1}   \\ 
& + \left(1 + 2 + 3 + \ldots +(i-4) + (i-3) \right)X_{2}   \\ 
& + \left(1 + 2 + 3 + \ldots + (i-4) \right)X_{3} + \ldots   \\ 
& + (1 + 2 + 3)X_{i-4} + (1+2)X_{i-3} + X_{i-2}  \\
= & \s{k=1}{i-2}k X_{1} + \s{k=1}{i-3}k X_{2}  + \ldots + \s{k=1}{2}k X_{i-3} 
+ \s{k=1}{1}kX_{i-2}  \\
= & \s{j=1}{i-2}\left(\s{k=1}{i-1-j}k\right)X_{j} = 
\frac{1}{2}\s{j=1}{i-2}(i-j)(i-j-1)X_{j}
\end{align}


\section{Least-squares minimisation of a random walk}
\label{sec: least squares minimisation of a random walk}
In this appendix, we will describe the process of fitting and
removing a polynomial from $N$ data points $(x_i, y_i)$ which undergoes a
random walk. The polynomial will be fitted using a least squares minimisation.
The $x_i$ are the independent points at which $y_i$ (which undergoes a random
walk) is measured. We begin by defining the least-squares fitting method then
go on to calculate the residual for several different degrees of polynomial.
This introduces the method in a generic setting which is then applied in
Section~\ref{sec: random walk models part II} to calculate the mismatch for a
GW signal which undergoes a random walk, but in which the search minimises the
mismatch over the search frequency and frequency derivative.

\subsection{Least squares fitting of a polynomial}
Given $N$ data points $x_{i}$, $y_{i}$, we define the residual from a least-squares
polynomial fit of order $k$, as
\begin{equation}
    r_i^{(k)} = y_{i} - y^{\textrm{(k)}}_{i},
\end{equation}
where
\begin{equation}
y^{\textrm{(k)}}_{i} = a_{0} + a_{1}x_{i} + a_{2}x_{i}^{2} + \dots
                                                           + a_{k} x_{i}^{k},
\end{equation}
is a polynomial of degree $k$.

Then the residual which we want to minimise is
\begin{equation}
R^{2} = \s{i=1}{N}\left(r_i^{(k)}\right)^{2}
      = \s{i=1}{N}\left(y_{i} - \left(a_{0} + a_{1}x_{i} + a_{2}x_{i}^{2} +
        \dots + a_{k} x_{i}^{k}\right)\right)^{2}.
\end{equation}
Partial differentiation with respect to the parameters $a_{i}$, yields $k$
simultaneous equations. Writing these as a matrix and then solving
for the best fit, $\hat{y}^(k)_i$, it can be shown \citep{WolframLeastSquares} that
\begin{align}
\hat{y}^{\textrm{(k)}}_{i} & = X \left(X^{T}X\right)^{-1} X^{T} y_{i} & \textrm{where} & &
X & = \left[\begin{array}{ccccc}
1 & x_{1} & x_{1}^{2} & \dots & x_{1}^{k} \\
1 & x_{2} & x_{2}^{2} & \dots & x_{2}^{k} \\
\vdots & \vdots & \vdots & \vdots & \vdots \\
1 & x_{n} & x_{n}^{2} & \dots & x_{n}^{k} \\
\end{array}\right]
\end{align}
Here $X$ is an example of a \emph{Vandermonde} matrix in which the terms follow
a geometric progression. It is useful to note that
\begin{equation}
XX^{T} = \left[\begin{array}{cccc}
N & \s{i=1}{N}x_{i} & \cdots &  \s{i=1}{N}x_{i}^{k} \\
\s{i=1}{N}x_{i} & \s{i=1}{N}x_{i}^{2} & \cdots &  \s{i=1}{N}x_{i}^{k+1} \\
\vdots & \vdots & \ddots & \vdots \\
\s{i=1}{N}x_{i}^{k} & \s{i=1}{N}x_{i}^{k+1} & \cdots &  \s{i=1}{N}x_{i}^{2k}
\end{array}\right].
\end{equation}

Provided that the $x_{i}$ are suitably defined, then an analytic fit can be
found for any $k$, the difficulty lies in inverting the matrix.

\subsection{Least squares fitting a polynomial to a random walk} We now take
the $x_i, y_i$ to be a Gaussian random walk beginning at the origin. To define
this, let $\delta y_{i} \sim N(0, \sigma^{2})$ be independent and identitically
distributed random variables for which their sum generates the random walk:
\begin{equation}
y_{i} = \sum_{j=1}^{i}\delta y_{i}.
\label{eqn: ToyModel RW definition}
\end{equation}
We also set each random walk event to occur according to $x_{i} = i \Delta x$.
Then the residual after fitting and removing a $k^{th}$ order polynomial to the
random walk $y_i$, is
\begin{equation}
r_i^{(k)} = y_{i} - \hat{y}_{i}^{(k)} = y_{i} - X \left(X^{T}X\right)^{-1} X^{T} y_{i}.
\label{eqn: fitted residual}
\end{equation}
This suggests the residual will be similar to the random walk, but modified by
the least squares fitting.  To illustrate this, in Figure~\ref{fig: ToyModelRW}
we plot a simulated random walk along with several fits.
\begin{figure}[htb]
\centering
\includegraphics[width=.9\textwidth]{ToyModelRW}
\caption{Example of a random walk on the left along with three polynomial fits
of varying order. On the right is the corresponding residual after subtracting
these fits. A dotted line marks the origin in both plots.}
\label{fig: ToyModelRW}
\end{figure}

\subsection{Zeroth order fitting}

We begin with the case of $k=0$ in which $X^{T} = [1, 1, \dots 1]$ such
that
\begin{equation}
X \left(X^{T}X\right)^{-1} X^{T} = \frac{1}{N} J_{N}
\end{equation}
where $J_{N}$ is the $N\times N$ matrix of ones.  Inserting this into
Eqn.~\eqref{eqn: fitted residual}, the residual from a zeroth order fit is
given by
\begin{equation}
r_i^{(0)}= y_{i} - \frac{1}{N} \s{j=1}{N}y_{j}.
\end{equation}
The zeroth order residual can be interpetted as the removing the
average value $\langle y_i \rangle$ from the random walk: this was illustrated
in Figure~\ref{fig: ToyModelRW}.

%For example the
%expectation after $i$ steps of the original random walk can be shown to be
%zero, therefore the expectation for the zeroth order residual after $i$ steps
%will also be zero.

%This can intuitively be understood from the fact that we started out RW at the origin,
%a zeroth order fit shifts the origin but a random walk should

We can now take expectations to understand the behaviour of the residual when
compared to the original definition of the random
walk in Eqn.~\ref{eqn: ToyModel RW definition}. For example, consider
the mean square translation distance from the origin of a random walk after $i$
steps. For a normal random walk, this has the well known result
\begin{equation}
E[y_{i}^{2}] = i \sigma^{2}.
\label{eqn: RW classic}
\end{equation}
We can calculate the corresponding quantity of the $k=0$ residual by first
noting that
\begin{align}
E\left[y_{i}y_{j}\right] & = E\left[\s{k=1}{i}\delta y_{k} \s{l=1}{j}\delta y_{l} \right] \\
& = \s{k=1}{i}\s{l=1}{j}E\left[\delta y_{k} \delta y_{l}\right] \\
& = \s{k=1}{i}\s{l=1}{j} \delta_{kl} \sigma^{2} \\
& = \sigma^{2}\min(i, j),
\label{eqn: E yiyi}
\end{align}
where $\delta_{kl}$ is the Kronecker delta. Then we have
\begin{align}
\left(r^{(0)}_{i}\right)^{2} & = y_{i}^{2} - \frac{2}{N}\s{k=1}{N}y_{i}y_{k} + N^{-2}\s{k=1}{N}\s{l=1}{N}y_{k}y_{l} \\
& =  y_{i}^{2} - 2 N^{-1} \left(\s{k=1}{i}y_{i}y_{k} + \s{k=i+1}{N}y_{i}y_{k} \right)+ N^{-2}\s{k=1}{N}\left(\s{l=1}{k}y_{k}y_{l} + \s{l=k+1}{N}y_{k}y_{l} \right).
\end{align}
Taking the expectation we have
\begin{align}
E\left[\left(r^{(0)}_{i}\right)^{2} \right] & = \sigma^{2}\left(i - \frac{2}{N}\left(\s{k=1}{i}k + \s{k=i+1}{N}i \right)+ \frac{1}{N^{2}}\s{k=1}{N}\left(\s{l=1}{k}l+ \s{l=k+1}{N}k \right) \right) \\
& = \sigma^{2}\left(\frac{N}{3} - i + \frac{1}{2} + \frac{i^{2}}{N} - \frac{i}{N} + \frac{1}{6 N}\right).
\end{align}
This result can be compared with Eqn.~\eqref{eqn: RW classic}, the expectation
of the squared value for a random walk.  In contrast, the expectation after $i$
steps for the residual random walk depends on the length of data $N$ that was
fitted. It can be shown the expectation has a minimum at $i=N/2$.

To further understand the difference between the random walk and
the residual random walk, let us consider the sum of squares after $N$ steps for
the random walk
\begin{equation}
E\left[\s{i=1}{N} y_{i}^{2}\right] = \s{i=1}{N} i \sigma^{2} =
                               \frac{1}{2}\left(N^{2} + N\right)\sigma^{2}.
\label{eqn: sum of squares}
\end{equation}
On the other hand, the sum of squares for the residual random walk is given by
\begin{equation}
E\left[\s{i=1}{N} \left(r^{(0)}_{i}\right)^{2}\right] = 
\frac{1}{6}\left(N^{2} -1\right)\sigma^{2}.
\label{eqn: sum of squares k0}
\end{equation}
Comparing equations \eqref{eqn: sum of squares} and \eqref{eqn: sum of squares
k0} we note that, for the leading order term, the coefficient is reduced, but
the power remains the same.

%We verify this behaviour with a simple script that produces a random walk of
%length $N$ then fits and subtracts a $0^{th}$ order polynimial; we then
%calculate the sum of the square residual. In Figure~\ref{fig:
%sum_of_squares_res_oth_order} we repeat this operation multiple times then plot
%the average of the sum of squares for the residual while varying $N$, the
%prediction of Eqn.~\eqref{eqn: sum of squares k0} is also plotted showing
%agreement.
%
%\begin{figure}[ht]
%\centering
%\includegraphics[width=.6\textwidth]{sum_of_squares_res_oth_order}
%\caption{Comparing Eqn.~\eqref{eqn: sum of squares k0} with the averaged sum of squares for a simulated random walk }
%\label{fig: sum_of_squares_res_oth_order}
%\end{figure}

\subsection{First order fitting}
We now consider a first order fitting for which
\begin{align}
\hat{y}^{(1)}_{i} & = X\left(X^{T}X\right)^{-1} X^{T} y_{i} & \textrm{with} &&
X & = \left[\begin{array}{cc}
1 & \Delta x \\
1 & 2 \Delta x  \\
\vdots & \vdots  \\
1 & N \Delta x  \\
\end{array}\right].
\end{align}
Inserting the definitions of $x_{i}$ we can write
\begin{equation}
\left(X^{T}X\right)^{-1} = \frac{1}{N(N-1)}\left[
\begin{array}{cc}
4N+2 & -\frac{6}{\Delta x} \\
 -\frac{6}{\Delta x} & \frac{12}{\Delta x^{2} (N+1)}
\end{array}•
\right] = \Cone.
\end{equation}
For convenience we have defined a symmetric matrix $\Cone$. We then proceed to
define another matrix
\begin{align}
    \mathcal{C}_{ij}^{(1)} & := X\Cone X^{T} \\  & =
\left[\begin{array}{cc}
1 & \Delta x \\
1 & 2\Delta x  \\
\vdots & \vdots  \\
1 & N \Delta x \\
\end{array}\right]
\left[\begin{array}{cc} \Cone_{11} & \Cone_{12} \\ \Cone_{21} & \Cone_{22} \end{array}\right]
\left[\begin{array}{cccc}
1 & 1 & \dots & 1 \\
\Delta x & 2\Delta x & \dots  & N \Delta x
\end{array}\right] \\
& =
\Cone_{11} J_{N} +
\Cone_{12} \Delta x \left[ \begin{array}{cccc}
2 & 3 & \dots & N+1 \\ 3 & 4 & \dots & \vdots \\ \vdots & & & \\  N+1& \dots & \dots & 2N
\end{array}\right] +
\Cone_{22} \Delta x^{2} \left[ \begin{array}{cccc}
1 & 2 & \dots & N \\ 2 & 4 & \dots & \vdots \\ \vdots & & & \\  N& \dots & \dots & N^{2}
\end{array}\right]
\end{align}
We can write $r^{(1)}_{i}$ as a summation by inferring the dependence of the
$i^{th}$ row of each matrix on the $j^{th}$ column
\begin{align}
r^{(1)}_{i} & = y_{i} - \s{j=1}{N}\mathcal{C}_{ij}^{(1)} y_{j}
& \textrm{ where} &&
\mathcal{C}_{ij}^{(1)} & = \Cone_{11} + \Cone_{12}\Delta x (i+j) + \Cone_{22}\Delta x^{2} ij
\end{align}

We have now defined the first order residual. To understand that fitting an
removing a first order polynomial has, we compute the expectation of the
square for the $i^{th}$ term
\begin{align}
\begin{split}
E\left[\left(r^{(1)}_{i}\right)^{2}\right]  =
\frac{1}{15 N \left(N^{2} - 1\right)} &
\left(2 N^{4} - 18 N^{3} i + 9 N^{3} + 78 N^{2} i^{2} - 78 N^{2} i \right.\\
      &\hspace{3mm} + 14 N^{2} - 120 N i^{3} + 180 N i^{2} - 78 N i \\
      &\hspace{3mm} \left. + 9 N + 60 i^{4} - 120 i^{3} + 78 i^{2} - 18 i + 2\right),
\end{split}
\end{align}
which can be compared with the classic result for a random walk given in
Eqn.~\eqref{eqn: RW classic}. Alternatively, comparing with Eqn.~\eqref{eqn:
sum of squares}, the expected sum of squares for the residual random walk is
\begin{equation}
E\left[\s{i=1}{N} \left(r^{(1)}_{i}\right)^{2}\right] 
= \frac{1}{15}\left(N^{2} -4\right)\sigma^{2},
\label{eqn: sum of squares k1}
\end{equation}
as in the zeroth order fit, the power of the leading order term remains the
same, but the coefficient decreases.

\subsection{Second order fitting}
For the residual left after removing a quadratic, the  argument proceeds in much
the same way with
\begin{align}
r^{(2)}_{i} & = y_{i} - \s{j=1}{N}\mathcal{C}_{ij}^{(2)} y_{j},
\end{align}
where
\begin{align}
\mathcal{C}_{ij}^{(2)}  = &\Ctwo_{11} + \Ctwo_{22}\Delta x^{2}ij +
\Ctwo_{33}\Delta x^{4}i^{2}j^{2} \nonumber + \Ctwo_{12}\Delta x(i+j) \nonumber \\
& + \Ctwo_{13}\Delta x (i^{2} + j^2) + \Ctwo_{23}\Delta x^{3}(ij^{2} + i^{2}j),
\label{eqn: MC_2}
\end{align}
and
\begin{equation}
C^{(2)} = \frac{1}{N(N-1)(N-2)}
\left[\begin{matrix}9 N^{2} + 9 N + 6 & - \frac{1}{\Delta{x}} \left(36 N + 18\right) & \frac{30}{\Delta{x}^{2}}\\- \frac{1}{\Delta{x}} \left(36 N + 18\right) & \frac{12 \left(2 N + 1\right) \left(8 N + 11\right)}{\Delta{x}^{2} \left(N + 1\right) \left(N + 2\right)} & - \frac{180}{\Delta{x}^{3} \left(N + 2\right)}\\\frac{30}{\Delta{x}^{2}} & - \frac{180}{\Delta{x}^{3} \left(N + 2\right)} & \frac{180}{\Delta{x}^{4} \left(N + 1\right) \left(N + 2\right)}\end{matrix}\right].
\label{eqn: C_2}
\end{equation}

The expression for the expected square value is too long to write out in full,
but the expected sum of squares for the residual random walk is
\begin{equation}
E\left[\s{i=1}{N} \left(r^{(2)}_{i}\right)^{2}\right] =
 \frac{1}{70}\left(3N^{2} -27\right)\sigma^{2},
\label{eqn: sum of squares k2}
\end{equation}
for which as in the case of the zeroth order and first order residuals, the
power of the leading order term remains unchanged, but the coefficient
decreases.

\subsection{Conclusions}
\label{sec: appendix conclusions}
We now have a method to calculate statistical quantities from the residual
left over after subtracting a $k^{th}$ order polynomial from a random walk.
Considering the sum of squares for a random walk and the residuals in equations
\eqref{eqn: sum of squares}, \eqref{eqn: sum of squares k0},
\eqref{eqn: sum of squares k1}, and \eqref{eqn: sum of squares k2} we find that
the leading order term retains the same
power of $N$ with increasing $k$ but the coefficient of this power gets
smaller. This reflects the improved fitting with the polynomial degrees. We
also note that with each increase in the order of fit we get a limit
on $N$ for which the sum of squares is positive. For zeroth order fitting this
is $N>1$, for first order $N>2$ and for second order $N>3$. This is because in
order to perform a least squares fit, we need at least $k+1$ points to fit.



\end{subappendices}


\biblio

\end{document}
