\documentclass[../full_thesis/full_thesis.tex]{subfiles}

% Default image directory
\newcommand{\thisdir}{../timing_variations}
\graphicspath{{\thisdir/img/}}

\begin{document}

Timing variations are, broadly speaking, any time when the usual Taylor
expansion in the phase Eqn.~\eqref{eqn: Taylor compact} does not accurately
describe the phase evolution of the pulsations. Two distinct types of variations
exist: the sporadic event-like \emph{glitches} seen in some pulsars
and the ubiquitous \emph{timing-noise} present, at some level, in all pulsars.
In this chapter, we will discuss the observations of these variations and the
models proposed to understand them.

\section{Glitches}
\label{sec: glitches}
In addition to the regular spin-down of radio pulsars due to magnetic braking,
some pulsars undergo anomalies in their timing solutions known as
\emph{glitches}.  These are sudden rapid increases in the pulsation frequency
which were first observed in the Crab \citep{Boynton1969, Richards1969} and
Vela pulsars \citep{RadhakrishnanManchester1969, Reichley1969}. Pulsar
timing methods model this as a permanent increase in the phase, frequency, and
first frequency derivative in addition to a frequency increment that
subsequently decays exponentially to zero \citep{Edwards2006}.  To model this,
for each glitch pulsar astronomers add on an additional term to
Eqn.~\eqref{eqn: Taylor compact}:
\begin{align}
\begin{split}
\phi_{\textrm{g}} = H(\tTOA-\tg)& {}\left(
\Delta\phi + \Delta\nu(\tTOA - \tg) + \frac{\Delta\dot{\nu}}{2}(\tTOA - \tg)^{2} \right.\\
& \hspace{3mm}+ \left.\left[1-\exp\left(-\frac{\tTOA - \tg}{\tau}\right)\right]\Delta\nu_{\textrm{t}}(\tTOA - \tg)
\right),
\end{split}
\label{eqn: glitch timing model}
\end{align}
where $H(t)$ is the Heaviside step function. The first
three terms are the permanent increase in phase, frequency, and spin-down,
while the last term gives the transient increase in the frequency
$\Delta\nu_{\textrm{t}}$ which decays exponentially with a timescale $\tau$.
To illustrate this, in Fig.~\ref{fig: glitch sketch} we show the spin-frequency
model of a glitch including a permanent increase in frequency $\Delta\nu$ and
a component $\Delta\nu_{\textrm{t}}$ which is `recovered'.
\begin{figure}[htb]
\centering
\includegraphics[]{glitch_sketch}
\caption{Illustration of the glitch model fitted by pulsar astronomers.}
\label{glitch sketch}
\end{figure}

In effect, pulsar astronomers fit separate Taylor expansions either side of the
glitch. This is a good model when the rise-time of the glitch, during which the
frequency increases, is short compared to the duration between observations.
This evolution of the frequency during a glitch has yet to be observed, but
high time resolution monitoring of the Vela pulsar placed an upper limit of
40~s for the rise-time between the original and the new period
\citep{Dodson2001}. Since we cannot resolve the glitch itself,
Eqn.~\eqref{eqn: glitch timing model} is appropriate and used for all known
glitches.

A comprehensive review of glitches was carried out by
\citet{Espinoza2011}; to illustrate a typical glitch, in Fig.~\ref{fig: glitch}
we reproduce data from this review on a glitch in the Crab pulsar.
\begin{figure}[htb]
    \centering
    \includegraphics[width=0.5\textwidth]{GlitchExample_jbman}
    \caption{
A glitch in the PSR B0531+21, the Crab pulsar. It occurred around MJD
53067 and had a fractional frequency jump of $\Delta\nu/\nu = 5.33 \pm 0.05
\times 10^{−9}$. (a) The timing residuals relative to a
slowdown model with two frequency derivatives when fitting data only up to the
glitch date. (b) Timing residuals after fitting all data in the plot; note
that the glitch feature is still visible. Both these panels have the same
scale, covering 500 ms. (c) Frequency residuals, obtained by subtracting the
main slope given by an average $\dot\nu$. (d) The behaviour of $\dot\nu$
through the glitch. This figure and caption are taken from Fig.~1 of
\citet{Espinoza2011}}
    \label{fig: glitch}
\end{figure}

Over 165 of the 2000 observed pulsars have been seen to glitch, often multiple
times. Typical values of the instantaneous frequency change range form
$10^{-9}$~Hz to $10^{-4}$~Hz For some pulsars this is accompanied by a change,
with either sign, in the spin-down rate $\Delta\dot{\nu}$ with absolute
magnitudes between $10^{-19}$~Hz/s to $10^{-12}$~Hz/s.
In most pulsars, the fraction of the change in frequency
\begin{align}
Q = \frac{\Delta\nu_\textrm{t}}{\Delta\nu + \Delta\nu_\textrm{t}},
\end{align}
can not be measured. This may be due either to infrequent observations of the
pulsar, or simply that $Q\ll1$.  A review of pulsars with measured values of
$Q$ was conducted by \citet{Lyne2000}; they found that in glitches from 18
pulsars, $Q$ correlates with $|\dot{\nu}|$ reaching values as large as
$\sim0.9$ for the youngest pulsar with the highest absolute spin-down rate,
the Crab pulsar.

Many pulsars have been observed to glitch several times, \citet{Melatos2008}
considered the waiting times between glitches and concluded that in most
glitching pulsars the glitches happen randomly with waiting times consistent
with a Poisson process, except in PSR J0537-6910 and PSR J0835-4510 which
displayed quasi-periodicity in the waiting times.

In Chapter~\ref{sec: glitches in cgw}, we perform our own investigation into the
population statistics of glitches with an aim to understand their implication
for gravitational wave searches. We find, in agreement with
\citet{Espinoza2011} and references therein, that the distribution of glitch
magnitudes has multiple modes which suggests that glitches may come from more
than one mechanism. We go on to apply a statistical model and determine, in an
empirical fashion, the properties of the underlying source populations.

Glitches provide a unique opportunity to investigate the physics of neutron
stars and many of the leading insights have been gained by their study. Two
leading models exist known as the \emph{superfluid unpinning} model and the
\emph{starquake} model.

In the superfluid unpinning model proposed by \citet{Anderson1975}, the star
contains a superfluid component in which the angular momentum is stored in an
array of vortices which are `pinned' to the crust. The magnetic dipole, rigidly
fixed to the crust, exerts a torque on the crust gradually spinning it down.
The superfluid component cannot decrease its angular momentum without
destroying the vortices into which it stores angular momentum and so does not
spin-down at the same rate.  A lag in frequency between the superfluid
component and rest of the crust develops until the forces are sufficiently
large to cause an avalanche of unpinning events rapidly transferring the stored
angular momentum in the superfluid component to the crust. The observed
pulsations measure the rotation rate of the crust into which the dipole is
frozen, so when this unpinning occurs, we see a rapid increase in the
frequency.

The second model, starquakes, follows from the observation that a rapidly
spinning fluid body has an oblate `rest shape' with a bulge about its equator
due to the centrifugal force. The crust of a star spinning at a frequency $\nu_0$
will solidify with a corresponding oblateness. Subsequently, as the star spins down,
it will have a different rest shape due to its decreased frequency, but the crust
will retain a memory of the earlier shape at which it solidified. This will cause
strains in the crust which eventually cause a starquake relieving the strain,
resetting the reference rest shape, and producing glitch like features. This
model was first proposed by \citet{Ruderman1969} and later built upon by
\citet{Baym1971}.

Both of these models have support in the literature and have been developed
significantly to explain the variety of observed glitches. However, there are
observations which cause difficulties for both models: glitches seen in the
Vela pulsar are too large and too often to be consistent with a starquakes
model, while the unpinning model requires a superfluid component which is at
odds with observation of precession (we discuss this further in
Chapter~\ref{sec: testing models}). In this thesis, we will not use glitches
as a tool for inferring neutron star physics, but any predictions we do make
must be compatible with what has already been learnt from glitches.




\section{Timing noise: observations}
\label{sec: timing noise observations}
Timing noise refers to small-scale structure in the timing residual which
cannot be attributed to any other source and hence cannot be modelled and
included in the timing model. The presence of timing noise indicates that we do
not have a complete picture of the neutron star: there is unmodelled physics.

Characterising timing noise is a difficult task: the exact form it takes will
depend on the order of Taylor expansion used to fit the timing parameters.
Typically pulsar astronomers truncate at $\ddot{\nu}$, but fitting to higher
orders is possible and will tend to decrease the `level' of the resulting
structure understood as timing noise. Of course using a sufficiently large
number of terms in the Taylor expansion, eventually one will fit out all of the
structure. However, this does not provide any additional insight into the cause
of timing noise. In this thesis, we will define timing noise as the remainder
having fitted and subtracted a second order Taylor expansion. In the
literature, a second order fit is most commonly used (see for example
\citet{Hobbs2010}), but examples exist of fitting and removing higher order
Taylor expansions (e.g. B0919-06, B1540-06, and B1828-11 in \citet{Lyne2010}).

To illustrate timing noise and how it can depend on the order of Taylor
expansion used, in Figure~\ref{fig: timing noise example} we show the phase
residual remaining after fitting and removing a $3^{rd}$, $4^{th}$, and
$5^{th}$ order Taylor expansion to the Crab pulsar. In all three instances we
see a quasi-periodic structure reaching residuals up to half a cycle; higher
order have lower residuals, but the form of the structure remains consistent
between orders.  \begin{figure}[htb] \centering
\includegraphics[width=0.75\textwidth]{PhaseResidual_45000_47000.pdf}
\caption{A phase residual demonstrating the structure which is named timing
noise. This is generated from data on the Crab pulsar (see section \ref{sec:
timing noise as described by the crab ephemeris} for details)} \label{fig:
timing noise example} \end{figure}

Several methods exist in the literature to quantify the strength of timing
noise such as the $\Delta_{8}$ value introduced by \citet{Arzoumanian1994}, the
generalisation of the Allan variance \citep{Matsakis1997}, the covariance
function of the residuals \citep{Coles2011}, and fitting for timing noise as
part of the pulsar timing model \citep{Lentati2014}.  The most comprehensive
and recent analysis was performed by \citet{Hobbs2010} who considered 366
pulsars over timescales $\gtrsim10$~years.  We summarise their conclusions
here: \begin{enumerate}

    \item Timing noise is widespread in pulsars

    \item Timing noise is inversely correlated the characteristic age defined
in equation \ref{eqn: characteristic age}

    \item The structures seen in the timing residual vary with data span: as
more data is collected, more quasi-periodic features are observed.

    \item The dominant contribution to timing noise for young pulsars with
$\tau_{c}<10^{5}$~years can be explained as being caused by the recovery from
previous glitches.

    \item A handful of pulsars exhibit significant periodicity's while
quasi-periodicity's are observed in many pulsars

\end{enumerate}

These general features give a broad picture, but there is great variation in
the form of timing noise between pulsars; this is illustrated by the variety of
timing residuals reported in \citet{Hobbs2010}. To understand the variety of
observation, in the next section we will discuss some of the models for timing
noise existing in the literature and which observations they are able to
explain.


\section{Timing noise: interpretations}
\label{sec: timing noise interpretations}
The underlying mechanism which causes timing noise is not understood. Since the
first discussions in \citet{Boynton1972} multiple models have been proposed
which are able to describe some of the features. However, the variety of ways
timing noise manifests and the uncertainty in the mechanisms at work have made
it difficult for any conclusive statements to be made about the models. A
complete undestanding of timing noise must not only explain the observed
variations, but also remain consistent with our understanding of neutron stars
derived from other observations such as glitches. A complicating factor in
understanding timing noise is that the observed features have timescales
similar to the duration that we have been able to observe pulsars; it is
therefore possible that observations which look like a random walk over a short
timescale, may in fact be periodic or something else entirely over longer
timescales.  Observations of new features prompt new models for timing noise;
as a result the timing noise interpretations have evolved with the
observations. In this section we will present an overview of these concepts and
the existing evidence which supports them.

\subsection{Random walk models}
\label{sec: TN interpretations random walk models}

Timing noise was first quantified and interpreted by \citet{Boynton1972} as a
Poisson like random walk in one of the phase, frequency, or spindown. The
pulsar spins down according to the power law spindown of equation \eqref{eqn:
power law spindown} except that at random times the pulse phase, frequency or
spindown jumps, or changes discontinuously. The waiting times between events
are Poisson distributed with a rate R such that over a period $T$ the number of
events follows a Poisson distribution with mean RT. All the jumps are
independent and the magnitudes are randomly distributed with means given by
$\langle\dP\rangle$, $\langle\dF\rangle$, and $\langle\dS\rangle$ for the phase
frequency and spindown. We can investigate the statistical properties of such
random walks by splitting the phase into contributions from the secular
spindown $\phi_{\mathrm{S}}$ (as given by equation \eqref{eqn: Taylor compact})
and contributions from the random walks
\begin{equation}
    \phi = \phi_{\mathrm{S}} + \Delta\phi_{\mathrm{R}}
\end{equation}
The three random walks can be written as the sum of $N$ individual events
occurring at times $t_{i}$
\begin{align}
    \Delta\phi_{\mathrm{R}} & = \s{i=1}{N}\Delta\phi_{i} H(t - t_{i}) 
     && \mathrm{(phase),} \\
    \Delta\phi_{\mathrm{R}} & = \s{i=1}{N}\Delta\f_{i} (t-t_{i})H(t - t_{i}) 
     && \mathrm{(frequency),} \\
    \Delta\phi_{\mathrm{R}} & = \s{i=1}{N}\frac{1}{2}\Delta\fdot_{i} (t-t_{i})^{2}H(t - t_{i}) 
     && \mathrm{(spindown),}
\end{align}
where $H(t)$ is unit step function at $t=0$. It should be noted that here we are
treating the three types of noise separately such that timing noise residuals 
from either a random walk in phase, frequency, or spindown; work by \citet{Cordes1980}
extended this model to handle mixing between the types of noise.

Timing noise is the remainder having
fitted and subtracted a second order Taylor expansion. Provided the perturbations
of $\phi_{\mathrm{s}}$ are small then the remainder will be exactly given by 
$\phi_{\mathrm{s}}$. Then, as described by \citet{Boynton1972} we can then 
average over: the $\dP_{i}, \dF_{i}$ or $\dS_{i}$
distributions, the $t_{i}$ distribution, and the $N$ distributions to give
\begin{align}
    \langle \Delta\phi_{R} \rangle & = \langle \dP \rangle R T 
    = S_{\mathrm{PN}}T && \mathrm{(phase),} \\
    \langle \Delta\phi_{R} \rangle & = \frac{1}{2}\langle \dF \rangle R T^{2} 
    = \frac{1}{2}S_{\mathrm{FN}}T^{2} && \mathrm{(frequency),} \\
    \langle \Delta\phi_{R} \rangle & = \frac{1}{6}\langle \dS \rangle R T^{3} 
    = \frac{1}{6}S_{\mathrm{SN}}T^{3} && \mathrm{(spindown).} \\
\end{align}
Here we have implicitly defined the strength parameters which combine the rate 
and averaged magnitude of jumps into a single quantity. Measurements of timing 
noise, if they are discreet events, will observe the accumulation of many 
individual events. As a result only the strength can be
measured, not the rate and average magnitude.

The three types of noise are distinguishable by their dependence on the
observation time $T$. The type of noise can be measured by
translating this into the dispersion measure of
$\Delta\fddot$ after fitting a cubic, or by inspection of the power spectrum.
\citet{Boynton1972} was able to categorise the Crab pulsar as frequency like noise.
They found that over a $5$~year period the noise process was stationary
and consistent with the frequency noise hypothesis. No deterministic process
could account for the timing residuals strengthening their conviction that some
random process was taking place.  However, they suggested that over longer periods
the random walk will be non-stationary due to either mixing with other types of
walks, or decay of the strength parameter with time.

The interpretation of timing noise as a Poisson random walk is a purely
statistical model. It is however backed up for a rich variety of physical
models.  A key feature of any physical
random walk model is that it must be able to produce both increases and
decreases in the relevant parameter. For this reason it is felt unlikely that
the timing noise mechanism is the same as the glitch mechanism, although they
must be related.  In addition it is unclear if timing noise is a continuous or
discreet process, certainly if it is discreet the waiting time between events
must be shorter than the shortest observation periods $\sim days$. 

The first physical model was proposed by \citet{Boynton1972}, the noise process
consisted of the as accretion of small lumps of matter onto the NS from the
interstellar medium. Lumps of matter fall randomly onto the surface of the star
causing either a spin-up or slowdown through the transfer of angular momentum.
After this many models were proposed such as starquakes and the random pinning
and unpinning of vortex lines; these were reviewed by \citet{Cordes1981} and
evaluated against observational constraints. Of these only three mechanisms
where found to be consistent with observations: crust breaking by vortex pinning, a
response to heat pulses, and luminosity related torque fluctuations. Since this
review, new random walk mechanisms have been proposed such as: variations in
the magnetospheric gap size \citep{Cheng1987}; the interference by debris entering
the magnetosphere \citep{Cordes2008}; and the accumulation of multiple micro-glitches
\citep{Janssen2006}. It would be a useful exercise to review both the new and 
old mechanisms against the current observational catalogue.

The first measurement of individual events was made by \citet{Cordes1985} who
identified $\sim20$ events in both frequency and spindown which could not be
explained by a glitch. In the same work, considering 24 pulsars over a period
of~$\sim13$~years, the authors concluded that: the timing noise seen in the
data could not be explained solely by an idealised random walk processes in the
phase, or its derivatives. They suggested that most of the activity is due to a
mixture of events in the phase, frequency and/or frequency derivative.

A recent review, and the most comprehensive by far was performed by
\citet{Hobbs2010} for 366 pulsars. They found that, timing residuals tend to
admit quasi-periodic features when observed on sufficiently long time scales
$\gtrsim 2$~years. As such, the method of measuring the type and strength of
timing noise depends on the length and epoch of observation. This suggests a
pure random walk hypothesis is not entirely consistent with observations.
Nevertheless, it is unclear what repercussions the conclusions of
\citet{Hobbs2010} has for the physical origin of timing noise.

\subsection{Free precession}
\label{sec: free precession}

A mechanism which could quite naturally produce strictly periodic variations in the
observable features of a pulsar is \emph{free precession}. This occurs in any
non-spherical body for which the angular momentum is not aligned with a principle
axis of the moment of inertia. Such a circumstance could arise given the
chaotic birth of NSs. However, we must be clear that the timing noise induced by 
precession alone would be strictly deterministic; this is something which we do
not observe.
It is instructive however to consider the mechanics of precession since it will
be visited later on.

In the simplest case imagine an biaxial body, 
rotating about an axis $\Omega$. with a moment of inertia given by 
\begin{equation}
    I = \left[\begin{array}{ccc}
            I_{0} & 0 & 0 \\
            0 & I_{0} & 0 \\
            0 & 0 & I_{0}(1 + \epsilon)
            \end{array}\right],
\end{equation}
where $\epsilon \ll 1$ is the measure of oblateness or prolateness.  If the
body is free from torques, then in the rotating frame of the body, Euler's
equations of motion (see section \ref{sec: neutron star dynamics in the
rotating frame}) are given by
\begin{equation}
    I\dot{\bm{\Omega}} + \bm{\Omega} \times \left(I\bm{\Omega}\right)=0.
\end{equation}
This is a system of three coupled ODEs. Writing the components of the spin
vector as $\bm{\Omega} = [\Omega_{x}, \Omega_{y}, \Omega{z}]$, we have the
set of equations:
\begin{align}
\dot{\Omega}_x = -\Omega_y\Omega_z, &&
\dot{\Omega}_y = \Omega_x \Omega_z, &&
\dot{\Omega}_z = 0
\end{align}
We can find a solution by first realising that $\Omega_{z}=\mathrm{const}$.
We are then left with a set of
two coupled ODEs, solving these with appropriate intial conditions 
the solutions take the form
\begin{align}
    \Omega_{x} & = \Omega_{0}\sin(a_0)\sin\left(\Omega_{0}\cos(a_0)\epsilon t\right), \\
    \Omega_{y} & = \Omega_{0}\sin(a_0)\cos\left(\Omega_{0}\cos(a_0)\epsilon t\right),\\
    \Omega_{z} & = \Omega_0 \cos(a_0),
\end{align}
where $a_0$ is the angle between the spin-vector and the body frame $z$ axis and
$\Omega_0$ is the magnitude of the spin-vector. 

We observe that the spin axis of the body will trace out a cone about the $z$
principle axis of the moment of inertia with a period of
$\frac{1}{\Omega_{z}\epsilon}$.  The half-angle of the cone is set by the
initial conditions and will not evolve. This is the motion of free precession
and is illustrated in figure \ref{fig: precession}. 
\begin{figure}[htb]
\centering
\includegraphics[scale=0.2]{Precession.png}
\caption{Illustration of free precession for a simple biaxial body. The spin
    axis $\spin$ traces out a cone about the angular momentum vector $\mathbf{J}$.}
\label{fig: precession}
\end{figure}•

Neutron stars are assume to have a rigid crust, as such they may be non-axially
symmetric. Precession as a candidate to explain timing noise fluctuations was
first discussed by \citet{Ruderman1970}. He found that the free precession
period was, for reasonable values of the  ellipticity $\epsilon$, able to
explain periodic fluctuations in the Crab pulsar. Of the known Neutron star 
physics, one of the few mechanism that could operate over the time-scales observed
in timing residuals is free precession. 

The favoured interpretation for glitches poses a problem for sustained free
precession as an interpretation of timing noise. Theoretical models suggest the
interior of a neutron star is a superfluid; most of the moment of inertia is
contained in an array of vortices which are pinned to the crust.  Glitch events
correspond to the sudden unpinning of these vortices. It was shown by
\citet{Shaham1977} that for perfect pinning the free precession frequency and
geometry were modified resulting in no slowly oscillating long-lived modes. In
the case of imperfect pinning \citet{Sedrakian1999} found that long-lived modes
existed but where damped.

Despite the inconsistency with the superfluid pinning model for glitches,
evidence was presented by \citet{Stairs2000} of free precession in pulsar
B1828-11. They found the phase residuals and variations in the pulse profile
could be accounted for by precession.  This was followed up by detailed
modelling of the effects by \citet{Akgun2006}.

Including the spindown from an applied torque \citet{Cordes1993} noted that
free precession may be driven by fluctuations that counter the damping process;
in turn, the precession can drive torque fluctuations. The effect is most
noticeable in young pulsars. 

Work by \citet{Jones2001} compared a model of free precession against the handful
of proposed observations of free precession. Their model included the feedback
between the torque and precession and required only the crust to undergo precession.
In all but one case such a model was found to be consistent with the observations.


\subsection{Two state switching}
\label{sec: two state switching}

Recently a new model has been proposed by \citet{Lyne2010} to explain the
observation that, over long time periods, the timing noise structure is
quasi-periodic. This began with the observation by \citet{Kramer2006} that the
pulses from PSR B1931+24 where intermittent. The pulsar acts as a normal pulsar
for $\sim10$~days and then switches off, being undetectable for $\sim25$~days,
and then switching on again. Analysing the spindown rate between the on and off
states, they determined the spindown rate $\dot{\f}$ was $\sim50\%$ faster in
the on state. This figure illustrating this is reproduced in figure \ref{fig:
kramer 2006 fig2}.
\begin{figure}
    \centering
    \includegraphics[width=.5\textwidth]{Kramer_2006_fig2}
    \caption{Figure taken from \citet{Kramer2006} showing the switched spindown
             of pulsar PSR B1931+24}
    \label{fig: kramer 2006 fig2}
\end{figure}
In the  upper panel (\textbf{A}) the authors show the evolution of the
rotational frequency over a 160 day period encompassing several switching
events. The line shows the long-term spindown of the pulsar while the dots show
individual measurements made during the on state. During these on states the
gradient of the reduction in frequency is increased, that is the spindown has
increased. It is thought that measurements of the frequency in the off state
would produce a line with decreased spindown connecting the dots. This is
accompanied by the timing residual measured over the same period in the lower
panel (\textbf{B}). This shows significant quasi-periodic modulations in sync
with the switching. This work proposed that the switching was a magnetospheric
phenomenon. The sharpness of the switches certainly requires something acting
on a short timescale and it intuitively makes sense that the greater spindown
results from greater torque produced by the emissions observed during the on
state.

The authors of \citet{Lyne2010} then tested a range of other pulsars and
presented a study of 17 pulsars for which they claim evidence for two-state
switching. Unlike B1931+24 these pulsars are not intermittent but 
continuously pulse. Measuring the spindown as a function of time over a
$\sim20$~year period they demonstrated fluctuations as reproduced in figure
\ref{fig: lyne 2010 fig2}

\begin{figure}
    \centering
    \includegraphics[width=.5\textwidth]{Lyne_2010_fig2}
    \caption{Figure taken from \citet{Lyne2010} showing the spindown rate
             of 17 pulsars over a $\sim20$~year period.}
    \label{fig: lyne 2010 fig2}
\end{figure}

This plot shows smooth variations in the spindown with some pulsars better
behaved than others. The method used to calculate the spindown required
averaging over a $\sim100$~day period; the authors argue that, if the spindown
undergoes a sharp switch between two values, this will be smoothed out by the
averaging.  Therefore it is argued in \citet{Lyne2010} that figure \ref{fig:
lyne 2010 fig2} show the $\dot{nu}$ moving between a few (typically 2) well
defined values.  The authors noted a gradual long-term change in the spindown
for all pulsars across the data set, indicating a non-zero second order
spindown. 

In order to quantify the claim of \citet{Kramer2006} that the switching was
magnetospheric (e.g. the results of enhanced particle flow), \citet{Lyne2010}
looked for correlations between the pulse width, measuring the amount of
emission, with the spindown rate.  They found that for 6 pulsars the pulse
width was indeed wider during the higher spindown. However, these variations
are also smooth and subject to the same averaging process. To improve the
resolution they then show the individual measurements of pulse width for two of
the pulsars; these appear to demonstrate the switching happening
instantaneously between the two values. The authors argue this confirms that
the two-state switching is magnetospheric since this is the only mechanism able
to act on such short timescales. Intuitively it makes sense that changes in
the pulse shape, will be correlated with changed in the amount of emission and 
hence the spindown rate. 

It has not been established how these magnetospheric will manifest themselves.
In particular this interpretation lacks an explanation of how the magnetosphere is
regulated to stay in a stable state for long periods ($1-10$~years) but then
switch over $\lesssim100$~days.

\citet{Lyne2010} proposes that the quasi-periodic structure observed in many
timing residuals (e.g. see \citet{Hobbs2010}) could be explain by this
switching process. To quantify this, in the supplementary material they created
a simple model for the effect of switching in the value of the spindown
$\dot{\nu}$ on timing residuals. We will now repeat this experiment to outline
the results. 

Firstly we model a pulsar as spinning down in the usual way except that it's
spindown has two distinct values $\dot{\nu}_{A}$ and $\dot{\nu}_{B}$. Then we
define the model the ratio of time spent state in state $A$ and $B$ as $R =
t_{B}/t_{A}$.  This is a purely deterministic model and having generated the
spindown values we can integrated twice to get the phase. Fitting and
subtracting a quadratic polynomial leaves the phase residual. In figure
\ref{fig: lyne example D=0} we show a typical result; in the top figure is the
spindown values which we define and in the bottom the resulting structure in
the phase residual.
\begin{figure}[htb]
    \centering
    \includegraphics[width=.5\textwidth]{{R_3.0_D_0}.pdf}
    \caption{A deterministic realisation of the Lyne switched spindown model. The
             resulting structure in the timing residuals are strictly periodic.}
    \label{fig: lyne example D=0}
\end{figure}
The phase residuals in figure \ref{fig: lyne example D=0} are strictly
periodic. \citet{Lyne2010} realised that in order to fit the observed
quasi-periodic residuals a random element must be introduced. This can be done
by the form of a 'dither' $D$ in the waiting time between switches. Now we have
periods $t_{A}^{i}$ and $t_{B}^{i}$ which are Gaussian distributed with a mean
of $t_{A}$ and $t_{B}$ and a standard deviation $D t_{A}$ and $D t_{B}$. The
result is illustrated in figure \ref{fig: lyne example D=0.3}.

\begin{figure}[htb]
    \centering
    \includegraphics[width=.5\textwidth]{{R_3.0_D_0.3}.pdf}
    \caption{A realisation of the Lyne model with a random element producing the
             observed quasi-period structure.}
    \label{fig: lyne example D=0.3}
\end{figure}


\citet{Lyne2010} argue that the fast state changes seem to rule out free
precession as the origin of oscillatory behaviour observed in timing residuals.
One of the pulsars which shows some evidence for two state switching is PSR
B1828-11; this pulsar was cited as evidence for free precession by
\citet{Akgun2006}. \citet{Lyne2010} argue that the fluctuations from this pulsar
should be reinterpreted as two-state switch due to the observed fast state
changes. 

\citet{Jones2012} argues that such dismissal of precession is premature
since the modulation period of the switching has yet to be explained.  Instead,
the idea is raised that precession and magnetospheric switching are not
mutually exclusive.  Pulsars are most probably born in a randomly distributed
magnetospheric state, at least some may therefore exist under a delicate
balance between two states. Precession may be capable of periodically varying
the statistical probability of existing in one state or the other, sharp
changes would be caused by an `avalanche effect' as the particle energies reach
a threshold.  This provides the timescale for switching along with the ability
for the switching to be quasi-periodic since the precession only biases the
probability.

A similar idea considered by \citet{Cordes2013} interpreted two state switching
as evidence for a system in a state of stochastic resonance.  This occurs in
systems in which, under certain conditions, a weak periodic forcing function is
amplified by stochastic noise.  To explain this phenomenon in appendix
\ref{App: Stochastic} we present a toy model of stochastic resonance for a
particle in a well. The switching could therefore be the result of any periodic
modulation, such as precession, coupled to random fluctuations. This would quite
naturally explain the stability of states, the timescales over which the occur,
and the fact that it is observed in only some pulsars.

\subsection{Evidence from anomalous braking indices}
\label{sec: evidence from anomalous braking indices}

An alternative motivation to study timing noise comes from the measurement of
anomalous braking indices. The pulsar braking index is defined by $n$ in
equation \ref{eqn: power law spindown}.  Rearranging this equation as in
\eqref{eqn: measured braking index} the braking index can be measured for
observed pulsars. Different types of braking exhibit different braking indices.
It is therefore a reasonable idea to measure the braking index and calculate
the type of braking. Pulsars spun down by an electromagnetic torque should
follow a braking index of $n=3$, while gravitational wave spindown has $n=5$.

Measuring these indices for the known pulsar population we do not find a consensus
on the type of braking. Values from from unity up to $10^{6}$ and even negative
braking indices have been measured. These are known as \emph{anomalous}
braking indices. 

Recent work by \citet{Biryukov2012} observed that
younger pulsars tend to have braking indices of the correct order of
magnitude. However, beyond~$\tau_{ch}\approx10^{5}$~years
the absolute value of the braking index rapidly grows,
reaching values as large as $10^{6}$ for the oldest pulsar. In addition an
almost equal number of pulsars have positive and negative values of the braking
index. The figure demonstrating this is plotted in
figure~\ref{fig: braking indices}.

\begin{figure}[ht]
\centering
	\includegraphics[width=0.5\textwidth,trim=0mm -10mm 0mm 0mm]
               {{Biryukov_2012_Figure_7}.png}
\caption{Pulsar population in the $n_{obs}-\tau_{ch}$ diagram image from
\citet{Biryukov2012}}
\label{fig: braking indices}
\end{figure}

\citet{Biryukov2012} proposed that the spindown $\dot{\nu}(t)$ may contain the
secular spin down $\dot{\nu}_{\textrm{sec}}(t)$ and a cyclic component
$\dot{\nu}_{\textrm{sec}}(t)\epsilon(t)\nu(t)$ oscillating the spindown about
a mean value. Taking a simple case where the cyclic term has the form $A
\cos\phi(t)$ where $A$ is the relative amplitude of the oscillations and
$\phi(t)$ is linear in $t$, the authors derive an equation for the observed braking
index

\begin{equation}
n_{obs}(t) =
\frac{n}{1+A\cos(\dot{\phi}t+\phi_0)}
+\frac{(n-1)(kt-c)}{(1+A\cos(\dot{\phi}t+\phi_{0}))^{2}}A\dot{\phi}\sin(\dot{\phi}t+\phi_{0}).
\label{eqn: nobs}
\end{equation}•

This observed braking index contains a constant positive term oscillating about
the true braking index and a term which grows linearly in time. The authors found 
that for $\tau_{ch}<10^{5}$ yrs the linear term is negligible and so we observe
approximately the real braking index $n$. At later times the linear term
drives the observed braking index to larger values while a sinusoidal term produces
positive and negative values. In figure \ref{fig: nobs} we plot the trajectory
of a single pulsar following equation \eqref{eqn: nobs}. The authors claim each
of the pulsars in \ref{fig: braking indices} is following a similar trajectory.

\begin{figure}[ht]
\centering
	\includegraphics[width=0.5\textwidth]
               {{Analytic_Monotonic_and_Cyclic}.png}
\caption{A sketch of the observed braking index according to
equation \eqref{eqn: nobs}, the values here are intended for a qualitative
overview rather than analysis. }
\label{fig: nobs}
\end{figure} 

This simplistic idea is able to explain some of the defining features of the
known pulsar population braking indices. This requires a mechanism to modulate
the spindown over long timescales. By fitting their model to data, they
estimate the timescale to be of the order $10^{3}-10^{4}$~years. At least one
physical model, precession, could produce variations on the required timescale.
However, this is significantly longer than the precession timescales invoked to
explain the fluctuations in timing residuals which were $1-10$~years.
\begin{subappendices}
\subsection{Toy model of stochastic resonance: particle in a potential}
\label{App: Stochastic}

Here we present a simple toy model of stochastic resonance. This is a
statistical phenomena occurring when a weak periodic forcing function is
amplified by noise (see \citet{Jung1991} for a full treatment).  For the
application to neutron stars, see \citet{Cordes2013}; here we simply aim to
describe the essential features of stochastic resonance (not its application to
NSs). 

We will consider a particle at a position $x$ which is subject to some
potential and acted upon by a forcing function $F(t)$. In general though, $x$
could be any state variable, thus stochastic resonance could be produced
in many systems.

First consider the static case of a particle in a potential $U(x)$  given by:
\begin{equation}
    U(x) = \frac{x^{4}}{4}-\frac{x^{2}}{2}. 
\end{equation}
This potential is characterised by two wells at $\pm1$, a maximum exists
between them at the origin. The particle in one of the wells sees a potential
barrier $\Delta U$ corresponding to the height of the maximum above its
position.

Assume the particle is acted upon by a random forcing function $F(t)$ which is
modelled as a Gaussian white noise with strength $D$. Depending on the
magnitude of $D$ with respect to the potential, the motion of the particle
admits two distinct cases:
\begin{enumerate}
\item $D \ll \Delta U \;\;$ in which case the particle remains inside whichever
    well it initially starts in and does not escape.
\item $D \gg \Delta U \;\;$ in this case the particle will not see the the
    individual wells only the larger one.
\end{enumerate}

The motion of the particle obeys the following equation of motion:
\begin{equation}
    \frac{dx}{dt} = -\frac{\partial V(x,t)}{\partial x} + F(t). 
\end{equation}
The motion of the particle has two components, the deterministic effect of the
potential and random fluctuations.

We now modify the potential to be acted on by a weak periodic function; this
introduced a third possible type of behaviour. Writing the time dependant
potential as
\begin{equation}
    V(x,t) = \frac{x^{4}}{4}-\frac{x^{2}}{2} + \epsilon x \cos(\omega_{0} t).
\end{equation}
Inserting this potential into the equations of motion:
\begin{equation}
    \frac{dx}{dt} =  x - x^{3} + F(t) + \epsilon \cos(\omega_{0} t).
\label{eqn:stochastic eom}
\end{equation}
Solving this numerically we fix $\epsilon=0.001$,
$\omega_{0}=\frac{2\pi}{10}$ and choose three values of $D$ which illustrate
typical behaviours of the solution
\begin{figure}[ht]
\centering
   \includegraphics[width=0.4\textwidth,trim=0mm -10mm 0mm 0mm]
   {{Stochastic_resonance}.png}

\caption{Three solutions to equation \eqref{eqn:stochastic eom} changing the
    random forcing functions strength $D$. The first and last panels show the
    deterministic solutions for the particle position: either the forcing
    function is weak compared to the potential, the particle remains in well in
    which it begins; or the forcing function is much stronger than the
    potential and so the particle freely moves about the two wells. The middle
    panel illustrates the special case of stochastic resonance whereby the
periodic fluctuations of the potential allow quasi-periodic variations in the
particles position between the two wells.}

\label{fig:stochastic resonance}
\end{figure}
The first and last runs replicate the behaviour expected  for a static well,
either the particle is confined to the well it starts in, or the random noise
is too strong and the individual wells are not observed. The middle case
displays strong stochastic resonance: the solution
displays a switching between bi-stable states but does not strictly follow the
period of the forcing function. The
important point here is that the forcing function may be weak, but provided it
is periodic or at least quasi-periodic the signal is amplified by the random
noise such that it may be visible in data sets where it would typically be
considered lost.

\end{subappendices}


\biblio


\end{document}
