Gravitational waves (GWs) were first predicted by Albert Einstein in 1916
\citep{einstein1916approximative} when he found that the linearised weak-field
equations of his General Theory of Relativity had transverse wave solutions.
Much like the generation of electromagnetic waves requires the acceleration of
electrical charges, GWs are generated by any source with a
time-varying mass quadrupole moment and can be understood as `ripples' or
spatial strains in the spacetime itself which travel at the speed of light.

Gravitational waves were first directly detected by the LIGO collaboration
\citep{abbott2016observation}. They observed a signal consistent with the
inspiral and merger event of two $\sim 30 \Msun$ black-holes over approximately
$0.2$~s. To detect such signals, LIGO uses a \emph{laser interferometer} to
measures the relative change in length between two orthogonal arms. In
particular, if $L$ is the length of either arm without a signal and a
gravitational wave passes through, the detector measures the strain
\begin{align}
h(t) = \frac{\delta L_x - \delta L_y}{L}
\end{align}
where $\delta L_x$ and $\delta L_y$ are the time-varying stretching and
squeezing of the two arms caused by the gravity wave. For the observed binary
black-hole merger the peak strain in the detector was $\sim 10^{-21}$.

Prior to this detection, indirect evidence for the existence of gravitational
waves was found by observing the orbital periods of compact binary systems.
Such systems have a time-varying quadrupole moment and emit gravitational
waves, which radiate energy away from the system causing, a decay of the orbital
period. In 1975, Hulse \& Taylor discovered a binary neutron star system where
one of the stars, PSR~1913+16, was visible as a pulsar \citep{Hulse1975}. Due to
the powerful techniques of pulsar timing, subsequent analysis by
\citet{Taylor1982} was able to verify that the orbital decay matched exactly
the predictions of General Relativity. Since this observation, more double
neutron star system have been discovered, including a system, PSR~J0737-3039A/B,
discovered by \citet{burgay2003increased}, where both neutron stars are seen as
pulsars. This so-called double pulsar system tests the agreement with General
Relativity at the 0.05\% level \citep{kramerstairs2006}.

Neutron stars observed as pulsars are often referred to as `cosmic clocks' for
the regularity of their pulsations. The most stable pulsars are the radio
millisecond pulsars (MSPs), which, due to their stability, many workers in the field
utilise in an attempt to search for GWs via a \emph{pulsar timing array}
\citep{hobbs2010international}: this searches for correlated signatures in the
TOAs from a network of well-timed MSPs. Such a detector is sensitive to a
stochastic background of gravitational waves by measuring the so-called
Hellings \& Downs curve \citep{hellings1983upper}, or to the mergers of
super-massive black hole binary systems \citep{lee2011gravitational}.

Isolated neutron stars themselves are potential sources of gravitational waves
through one of three mechanisms. If the star has a rotation axis misaligned
with its symmetry axis then it will undergo \emph{precession}: a `wobble` of
the star which has a time-varying quadrupole moment. This will produce GWs at
the rotation frequency and twice the rotation frequency, but the small
amplitudes of possible sources and questions over how long lived they might be
make this an unlikely candidate for LIGO \citep{Jones2002}. If the neutron star
is subject to \emph{non-axisymmetric instabilities}, such as the R-mode
instability in newborn and rapidly accreting neutron stars
\citep{andersson2001r}, then these too can produce GWs (for a review see
\citet{andersson2003gravitational}).  Finally, if the star possesses a
\emph{non-axisymmetric distortion}, $\epsilon$, also known as a `mountain', it
will produce a continuous gravitational wave at twice its rotation frequency
with a strain amplitude proportional to $\epsilon$. The LIGO detectors have
already been used to search for signals from known neutron stars and, by not
observing any radiation, are able to place upper limits on $\epsilon$ (see for
example \citet{ligo2008, ligo2011}).

All three of these detection mechanisms are potential sources of the first detection
of gravitational waves from neutron stars and realising this would provide a
unique opportunity to learn about neutron stars. But is it feasible? A statistical
argument can be made for the `loudest expected signal from unknown isolated
neutron stars'. This argument is given in \citet{abbott2007searches},
although the origin can be dated back to Blandford (1984) as attributed by Kip Thorne
in \citet{Hawking1989}. Essentially, one assumes that the population of $10^{5}$
neutron stars predicted to exist in our galaxy by stellar evolution models are
all born with a high spin-down rate and subsequently spin-down principally due to
the emission of gravitational waves. With additional assumptions that the stars
are born randomly throughout the Galactic disk with a constant birthrate the
populations are transformed into a population of neutron star strains. Then it
is shown that there is a 50\% chance a source exists with a strain amplitude
\begin{align}
h_0 \sim 4 \times 10^{-24},
\end{align}
which is close to `detectable' by LIGO, although the exact details depend on the
source frequency and duration. While this is a purely statistical argument, and
changing any of the assumptions tends to decrease this signal strain
\citep{Prix2009}, the rewards for detection in terms of astrophysics are
sufficient to motivate further research.

