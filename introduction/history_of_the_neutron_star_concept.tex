The story of a neutron star (NS) begins with the death of a main-sequence star
in a supernova event. This occurs when the radiation pressure provided by
fusion in the core can no longer support the stars structure against
gravitational collapse. The resulting stellar remnant can be one of three
compact objects depending on the mass of the original star. For low mass stars,
the gravitational collapse can be arrested by electron degeneracy pressure: a
result of the electrons (which are fermions) being subject to the Pauli
exclusion principle. This will produce a faintly luminous compact object known
as a white dwarf. \citet{Chandrasekhar1931} found that electron degeneracy
pressure limits the maximum size of a white dwarf to~$\sim 1.4 \Msun$ (where
$\Msun$ is one solar mass); larger mass stars will collapse further. For
intermediate mass stars they may collapse to form a neutron star.  These were
first postulated by Landau as `dense stars which look like giant atomic nuclei'
\citep{Yakovlev2013} even before the discovery of the neutron by
\citep{Chadwick1932}.  The first explicit prediction of a neutron star was made
by \citet{Baade1934} in trying to explain the energy released in supernova
explosions. For such massive stars the gravitational collapse is sufficient to
initiate inverse beta decay where the electrons are absorbed by the nuclei
converting the protons into neutrons
\begin{equation}
    e^{-} + p \rightarrow n + \nu.
\end{equation}
The resulting neutrons, being fermions, exhibit neutron degeneracy pressure
preventing further gravitational collapse and leaving a stable neutron star.
The maximum radius such a star can support is $\sim 10$~km but the mass 
compressed into this volume is $\sim M_{\odot}$.  Consequently, the density far
exceeds ordinary nuclear densities. For even more massive stars, no mechanism
exists which can support the structure against collapse, these will collapse to
a black hole. 

After the conception of neutron stars as stable compact objects there was
thought to be little chance of observation. They are many orders of magnitude
smaller than other celestial objects. Soon after their formation they are
rapidly cooled by the emitted neutrinos, typical temperature measurements have
been observed with $10^{5}$K to $10^{6}$K.  Nevertheless, their thermal
emission is difficult to observe. The early detection hopes were initially
focused on finding young neutron stars as discreet X-ray sources.

