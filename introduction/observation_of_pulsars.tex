A bright periodic EM signal was identified by \citet{Hewish1968} during a high
time-resolution survey for interplanetary scintillation. The shortness of
pulses and frequent and precise periodicity suggested that the source was
small. For a star to remain gravitationally bound, its rotation frequency is
constrained by the requirement that the centrifugal acceleration at the equator
be less than the gravitational acceleration. The observed pulses had a short
period $\sim 1.33$~s which ruled out white dwarfs since they could rotate this
fast without becoming gravitationally unbound. After the identification of
other similar objects these sources were named \emph{pulsars}. 

While multiple interpretation's were posed to explain the observation, the
identification of pulsars with neutron stars came from \citet{Pacini1967} and
independently \citet{Gold1968}. They suggested that a rapidly rotating neutron
star with a strong dipolar magnetic field would stream radiation out along the
magnetic axis. If this axis was misaligned from the rotation axis then the
beams would be swept out like a lighthouse. Beams passing over the earth are
observed as periodic pulses at the rotation frequency of the star.  This
identification was confirmed by the discovery of a pulsar at the centre of Crab
nebulae, a supernova remnant, agreeing with the prediction of \citet{Baade1934}.


