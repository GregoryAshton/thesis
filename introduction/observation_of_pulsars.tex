After the conception of neutron stars as stable compact objects there was
thought to be little chance of observing them. They are many orders of
magnitude smaller than other celestial objects and soon after their formation
in rare supernovae events (\citet{van1991} predicts just 3 supernovae per
century in the Milky Way Galaxy)
they are rapidly cooled by the emitted neutrinos
making their thermal emission difficult to detect.

In 1968 a bright periodic EM signal, now known as PSR~B1919+21, was identified
by \citet{Hewish1968} during a high time-resolution survey for interplanetary
scintillation. The source was measured with a radio frequency of 81.5~MHz
and pulsed with a period of $\sim 1.377$~s; this led to the name \emph{pulsars}
to refer to such sources.  Following this, several other similar objects where
discovered. A unifying feature of all pulsars is the clock-like stability of
the pulsations - something not rivalled by any other astrophysical phenomenon.
This stability indicates that the source must be a coliminated beam fixed to a
rotating body such that, as it rotates, the beam sweeps out like a lighthouse;
in this way the pulsation period is exactly the rotation period of the body.
Alternative models such as emission due to accretion from a binary companion
could never reach the stability's seen in pulsars.

For any rotating body to remain gravitationally bound, its rotation frequency is
constrained by the requirement that the centrifugal acceleration at the equator
be less than the gravitational acceleration. The rotation frequency of the observed
pulsars ruled out all known astrophysical bodies except the two most compact
objects, neutron stars and black holes, since all other bodies would not be
gravitationally bound at these frequencies. Isolated black holes are unable to
support an electromagnetic emission mechanism, this left only neutron stars
as candidates.

The identification of pulsars with neutron stars came independently from
\citet{Pacini1967} and \citet{Gold1968}. They suggested that a rapidly rotating
neutron star with a strong dipolar magnetic field would stream radiation out
along the magnetic axis. If this axis is misaligned from the rotation axis,
then the beams are swept out like a lighthouse. Beams passing over the earth
are observed as periodic pulses at the rotation frequency of the star.  Such a
model predicts that the electromagnetic radiation should exert a torque and
slow-down the rotation.  This slowdown was subsequently measured in the Crab
pulsar, which was discovered at the centre of the Crab nebulae which is a
supernova remnant, agreeing with the prediction of \citet{Baade1934}.

These early detection gave birth to a new field of astronomy: pulsar astronomy.
Since then, researches in the field have detected over 2000 radio pulsars, measured thermal
emission from a handful of nearby neutron stars with typical temperatures of
$10^{5}$K to $10^{6}$K \citep{pavlov2003thermal}, found neutron stars in
accreting binary systems with companions, and identified many other ways to
observe neutron stars. We will discuss some of these which are relevant to this
thesis in Section~\ref{sec: categorising neutron stars}, but first in
Section~\ref{sec: pulsar timing methods} we describe the techniques used by the
field to identify and `time' radio pulsars.
