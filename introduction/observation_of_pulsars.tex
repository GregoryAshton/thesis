After the conception of neutron stars as stable compact objects there was
thought to be little chance of observing them. They are many orders of magnitude
smaller than other celestial objects and soon after their formation they are
rapidly cooled by the emitted neutrinos. Nevertheless, thermal emission has
been observed from several nearby neutron stars with typical temperatures
of $10^{5}$K to $10^{6}$K.
%The early detection hopes were initially
%focused on finding young neutron stars as discreet X-ray sources.

In 1968 a bright periodic EM signal was identified by \citet{Hewish1968} during
a high time-resolution survey for interplanetary scintillation. The source was
measured with a radio radio frequency of 81.5~MHz and pulsed with a period of
$\sim 1.377$~s; this led to the name \emph{pulsars} to refer to such sources.
Folowing this, several other similar objects where discovered. A unifying
feature of all pulsars is the clock-like stability of the pulsations - something
not rivalled by any other astrophysicsl phenomenon.
The pulsar periods constrain the size of the source, indicating it must originate
from a compact astrophysical source, either a white dwarf, neutron star or
a black hole.

For a star to remain gravitationally bound, its rotation frequency is
constrained by the requirement that the centrifugal acceleration at the equator
be less than the gravitational acceleration. The observed pulsation periods
ruled out white dwarfs since they could not rotate this fast without becoming
gravitationally unbound. As for black holes, isolated black holes are unable to
support a periodic emission and emission from accretion can not achieve the
stablities observed in pulsars.

The identification of pulsars with neutron stars came from \citet{Pacini1967}
and independently \citet{Gold1968}. They suggested that a rapidly rotating
neutron star with a strong dipolar magnetic field would stream radiation out
along the magnetic axis. If this axis was misaligned from the rotation axis
then the beams would be swept out like a lighthouse. Beams passing over the
earth are observed as periodic pulses at the rotation frequency of the star.
Such a model predicts that the electromagnetic radiation should exert a torque
and slow-down the rotation.  This slowdown was subsequently measured in the
Crab pulsar, discovered at the centre of Crab nebulae, a supernova remnant,
agreeing with the prediction of \citet{Baade1934}.


