The story of a \emph{neutron star} (NS) begins with the death of a main-sequence star
in a supernova event. Prior to this, the nuclear fusion of hydrogen atoms into
helium provides pressure supporting the star in an equilibrium configuration
with the inward pressure of the stars self-gravity. Eventually the star depletes
its reserves of hydrogen and can no longer maintain the equilibrium. If the
star has an initial mass greater than $\sim 8 \Msun$, then it may undergo a
\emph{core-collapse supernova} during which the temperatures and pressure rapidly
increase. Under these conditions the electrons and protons undergo inverse
beta decay combining to form neutrons and neutrinos
\begin{equation}
    e^{-} + p \rightarrow n + \nu.
\end{equation}
Once the pressures reaches nuclear densities, neutron degeneracy pressure can
halt the collapse in a new equilibrium configuration; the resulting remnant is
a neutron star. However, if the remnant has a mass greater than $\sim 5 \Msun$,
neutron degeneracy pressure will be unable to support the gravitational
pressure and it will collapse to form a black hole.

The idea of neutrons stars was first postulated by Landau as `dense stars
which look like giant atomic nuclei' \citep{Yakovlev2013} even before the
discovery of the neutron by \citep{Chadwick1932}. However it was
\citet{Baade1934} who made the explicit prediction of a neutron star whilst
trying to explain the energy released in supernova explosions.

The maximum radius such a star can support is $\sim 10$~km but the mass
compressed into this volume is $\sim M_{\odot}$.  They can rotate with
frequencies up to 700~Hz and exist as either \emph{isolated} or as part of a
binary system.  In either of these systems their compactness means they are
potential sources of detectable gravitational waves. In fact the orbital decay
of the binary neutron star system PSR~1913+16 discovered by \citet{Hulse1975}
and subsequently analysed by \citet{Taylor1982} provided the first evidence for
gravitational waves.  NSs can have extreme magnetic fields with estimates
strengths up to $10^{15}$ gauss. 

Our knowledge of neutron stars is founded on observing them in the variety of
system detectable by astronomy. In this thesis we will investigate how we can
use current electromagnetic observations to further our understanding of neutron
star physics by investigating the causes of so-called \emph{timing-noise}.
Detecting gravitational waves from isolated neutron stars will provide a second
channel to investigate neutron star physics; in this thesis I will also
investigate the importance timing-noise and glitches may have on the ability to
detect such gravitational waves.

In this introductory chapter, we will describe current understanding of neutron
stars concentrating on those observed as 'pulsars'. The basic astrophysics will
be introduced along with the methods used to collect data. Finally we will describe
two phenomena, glitches and timing-noise, which provide an opportunity to
probe the neutron star physics.
