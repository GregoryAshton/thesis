For rotation powered pulsars, the normal population in Fig.~\ref{fig:
Period_PeriodDot}, we describe in this section how the neutron star properties
can be inferred from the stars timing properties. To do this, we will model the
star as described by \citet{Pacini1967} and \citet{Gold1968} and illustrated in
Fig.~\ref{fig: DipoleSpindownSimple}: a rapidly rotating
body with a magnetic dipole fixed in the crust at an angle $\alpha$ to the rotation
axis.
\begin{figure}[htb]
    \centering
    \includegraphics[width=.5\textwidth]{DipoleModelSimple}
    \caption{An illustration of the dipole spin-down model. The dipole and some 
    of the closed field lines are fixed at an angle $\alpha$ to the rotation 
    axis. As the body rotates, radiation is emitted along both ends of the dipole
    axis producing a torque on the body.}
    \label{fig: DipoleSpindownSimple}
\end{figure}

From \citet{Landau2013classical} the total radiation from a dipole rotated at
angular frequency $\Omega$ is given by
\begin{align}
I = \frac{2}{3}\frac{\Omega^{4}}{c^{3}} d_{0}^{2},
\end{align}
where $d_0$ is the projection of the dipole moment on the plane perpendicular
to the axis of rotation \citep{Pacini1967}. Following \citet{Shapiro83}, the
magnitude of the magnetic dipole moment for a star with radius $R$ and surface magnetic
field strength $B_0$ is $B_{0}R^{3}/2$. Including the projection onto the
plane perpendicular to the rotation axis, the total radiation is then
\begin{align}
I = \frac{1}{6}\frac{\Omega^{4}}{c^{3}} B_0^2 R^{6} \sin^{2}\alpha.
\end{align}

The rotational energy of a body spinning at $\Omega$ with a moment of inertia
$I_{0}$ is given by
\begin{equation}
    E = \frac{1}{2}I_{0}\Omega^{2}.
\end{equation}
Differentiating this expression with respect to time gives the loss of rotational
energy, $\dot{E}=I_0 \Omega\dot{\Omega}$. Assuming that all the energy is lost
to the rotation of the dipole, hence the name rotation powered pulsars, we can
equate $\dot{E} = -I$. We then rearrange
to give a power-law relation between the spin-down rate and the spin-frequency:
\begin{align}
\dot{\Omega} = -\frac{B_0^{2} R^{6} \sin^{2}\alpha}{6 c^{3} I_0} \Omega^{3}.
\end{align}

This power-law dependence is a model specific version of a more general
phenomenological power-law braking model
\begin{equation}
    \dot{\Omega} = -k \Omega^{n}.
    \label{eqn: power law spin-down}
\end{equation}
Generalising in this way suggests a powerful method to determine the type of
braking for a given pulsar. Specifically, differentiating Eqn.~\eqref{eqn: power law spin-down}
and rearranging it can be shown that
\begin{equation}
    n = \frac{\ddot{\Omega}\Omega}{\dot{\Omega}^{2}}.
    \label{eqn: measured braking index}
\end{equation}
Therefore, if $\ddot{\Omega}$ can be measured, then $n$ can be determined, and
hence used to infer the type of braking. For example, measuring $n=3$ would
indicate the pulsar braking is dominated by losses due to the magnetic dipole,
in contrast, it can be shown that gravitational wave braking would produce
$n=5$ \citep{Shapiro83}. Unfortunately, in reality, pulsars do not constrain this
value. Braking indexes have been measured with values as large as $\sim10^{6}$.
We will return to this issue in Sec.~ \ref{sec: evidence from anomalous braking
indices}.

To infer the age of the pulsar, Eqn.~\eqref{eqn: power law spin-down} can
be integrated between the initial values ($t=0, \Omega=\Omega_{i}$) and the
observed value ($\Omega_{\mathrm{o}}$) to give
\begin{equation}
    t = \frac{1}{(1-n)} \frac{\Omega_{\mathrm{o}}}{\dot\Omega_{\mathrm{o}}} 
        \left(1 - \frac{\Omega_{\mathrm{o}}^{n-1}}{\Omega_{i}^{n-1}}\right).
\label{eqn: characteristic age}
\end{equation}
Typically, we make the assumption that all
pulsars, regardless of their measure braking index, are dominated by EM braking
such that $N=3$. Then additionally assuming that $\Omega_{i} \gg
\Omega_{\mathrm{o}}$ we can approximate to a characteristic age
\begin{equation}
    \tau_c = \frac{-1}{2}\frac{\Omega_{\mathrm{o}}}{\dot\Omega_{\mathrm{o}}}
         = \frac{1}{2}\frac{P}{\Pdot},
\end{equation}
where $P=\frac{2\pi}{\Omega}$ is the pulse period and
$\dot{P}=-2\pi\frac{\dot{\Omega}}{\Omega^{2}}$ is the period derivative.

To infer the approximate surface magnetic field strength, we first note that
in the EM dipole braking model:
\begin{align}
k = \frac{B_0^{2} R^{6} \sin^{2}\alpha}{6c^{3}I_0}.
\end{align}
Then rearranging and substituting for $k$ in Eqn.~\eqref{eqn: power law
spin-down} we can estimate the surface magnetic field strength by
\begin{equation}
    B_{0} = \left(\frac{6 c^{3} I_{0}}{R^{6} \sin^{2}\alpha}\right)^{\frac{1}{2}} 
            \left(\frac{-\dot{\Omega}}{\Omega^{3}}\right)^{\frac{1}{2}}
          = \frac{1}{2\pi}\left(\frac{6 c^{3} I_{0}}{R^{6} \sin^{2}\alpha}\right)^{\frac{1}{2}}
           \sqrt{P \Pdot}
\label{eqn: surface magnetic field}
\end{equation}
In general we do not know the inclination angle $\alpha$, but we can evaluate a 
minimum magnetic field strength by setting $\alpha=\pi/2$. In CGS units, for a
canonical pulsar with $R=10^{6}$~cm, $I_{0}=10^{45}$~g~cm$^{2}$ \citep{Lyne2012book}, we can approximate
the magnetic field strength as
\begin{equation}
    B_{0} = 3.2 \times 10^{19} \sqrt{P \Pdot}.
\label{eqn: surface magnetic field canonical}
\end{equation}

In this section, we have introduced some of the simple results that can be
obtained by modelling the time evolution of pulsars with a power law. In
practise this model is consisent with most pulsar observations and provides a
useful way to categorise them via their spin-down age and magnetic field.
However, such a model does not capture all of the subtle variations which are
the focus of this thesis.  We will frequently refer back to this model as it
provides a useful platform from which to begin understanding neutron stars.

