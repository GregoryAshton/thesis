\documentclass[../full_thesis/full_thesis.tex]{subfiles}

% Default image directory
\graphicspath{{../introduction/img/}} 
\newcommand{\IntroductionDir}{../introduction}

\begin{document} 

\section{History of the neutron star concept}
The story of a neutron star (NS) begins with the death of a main-sequence star
in a supernova event. This occurs when the radiation pressure provided by
fusion in the core can no longer support the stars structure against
gravitational collapse. The resulting stellar remnant can be one of three
compact objects depending on the mass of the original star. For low mass stars,
the gravitational collapse can be arrested by electron degeneracy pressure: a
result of the electrons (which are fermions) being subject to the Pauli
exclusion principle. This will produce a faintly luminous compact object known
as a white dwarf. \citet{Chandrasekhar1931} found that electron degeneracy
pressure limits the maximum size of a white dwarf to~$\sim 1.4 \Msun$ (where
$\Msun$ is one solar mass); larger mass stars will collapse further. For
intermediate mass stars they may collapse to form a neutron star.  These were
first postulated by Landau as `dense stars which look like giant atomic nuclei'
\citep{Yakovlev2013} even before the discovery of the neutron by
\citep{Chadwick1932}.  The first explicit prediction of a neutron star was made
by \citet{Baade1934} in trying to explain the energy released in supernova
explosions. For such massive stars the gravitational collapse is sufficient to
initiate inverse beta decay where the electrons are absorbed by the nuclei
converting the protons into neutrons
\begin{equation}
    e^{-} + p \rightarrow n + \nu.
\end{equation}
The resulting neutrons, being fermions, exhibit neutron degeneracy pressure
preventing further gravitational collapse and leaving a stable neutron star.
The maximum radius such a star can support is $\sim 10$~km but the mass 
compressed into this volume is $\sim M_{\odot}$.  Consequently, the density far
exceeds ordinary nuclear densities. For even more massive stars, no mechanism
exists which can support the structure against collapse, these will collapse to
a black hole. 

After the conception of neutron stars as stable compact objects there was
thought to be little chance of observation. They are many orders of magnitude
smaller than other celestial objects. Soon after their formation they are
rapidly cooled by the emitted neutrinos, typical temperature measurements have
been observed with $10^{5}$K to $10^{6}$K.  Nevertheless, their thermal
emission is difficult to observe. The early detection hopes were initially
focused on finding young neutron stars as discreet X-ray sources.



\section{Observation of pulsars and their identification with neutron stars}
A bright periodic EM signal was identified by \citet{Hewish1968} during a high
time-resolution survey for interplanetary scintillation. The shortness of
pulses and frequent and precise periodicity suggested that the source was
small. For a star to remain gravitationally bound, its rotation frequency is
constrained by the requirement that the centrifugal acceleration at the equator
be less than the gravitational acceleration. The observed pulses had a short
period $\sim 1.33$~s which ruled out white dwarfs since they could rotate this
fast without becoming gravitationally unbound. After the identification of
other similar objects these sources were named \emph{pulsars}. 

While multiple interpretation's were posed to explain the observation, the
identification of pulsars with neutron stars came from \citet{Pacini1967} and
independently \citet{Gold1968}. They suggested that a rapidly rotating neutron
star with a strong dipolar magnetic field would stream radiation out along the
magnetic axis. If this axis was misaligned from the rotation axis then the
beams would be swept out like a lighthouse. Beams passing over the earth are
observed as periodic pulses at the rotation frequency of the star.  This
identification was confirmed by the discovery of a pulsar at the centre of Crab
nebulae, a supernova remnant, agreeing with the prediction of \citet{Baade1934}.




\section{Categorising neutron stars}
The timing properties, and other features measured by the timing model, for
over 2000 pulsars observed can be accessed via the Australia Telescope National
Facility (ATNF) pulsar catalogue \citep{ATNF}.  We can categorise the
population by their measured values of period $P$ and the period derivative
$\Pdot$. This is done by plotting them in a so-called $P - \Pdot$ diagram, as
shown in Figure~\ref{fig: Period_PeriodDot}.  The various categories to which
each pulsar can be assigned have been marked in this plot and we now discuss
their features.

\begin{figure}[htb]
\centering
\includegraphics[]{Period_PeriodDot} 
\caption{Period -
Period derivative diagram using data taken from the ATNF pulsar catalogue
\citep{ATNF}. Dashed lines show inferred characteristic ages and 
surface magnetic fields as given by Eqn.~\eqref{eqn: tauAge definition}
and Eqn.~\eqref{eqn: surface magnetic field canonical} respectively.}
\label{fig: Period_PeriodDot}
\end{figure}

The majority of pulsars (referred to as the `normal' pulsars) are found
\emph{isolated} without a binary companion and have typical periods of
$P=10^{-2}-10^{1}$~s. These can be described as \emph{rotation powered}
pulsars, since the electromagnetic (EM) radiation is powered by the loss of
rotational energy. As described later in Section~\ref{sec: rotation powered
pulsars}, estimates can be made of their characteristic age, $\tauAge$, and
surface magnetic field strength, $B_{0}$, based on a dipole spin-down model.
Constant lines of these quantities are plotted in Figure~\ref{fig:
Period_PeriodDot}. Of the normal pulsars, we can choose to identify the young
pulsars as those for which $\tauAge<10^{5}$~yrs. Some of these, such as the
Crab and Vela pulsars, can be directly associated with their supernova remnant
from which they were formed \citep{Kaspi1996}.

A second smaller population of isolated rotation powered pulsars exists with
$P<10^{-1}$~s. These are known as the \emph{millisecond pulsars} (MSPs). This
special class of pulsars are believed to start their life as normal pulsars,
but are then spun-up through accretion from a normal star. In support of this
hypothesis, the majority of MSPs in Figure~\ref{fig: Period_PeriodDot} have a
binary companion \citep{wijnands1998millisecond}. Additionally, we see so-called
low-mass X-ray binary systems (LMXBs) which are systems where a neutron star in
a binary accretes matter from its companion; the infalling matter releases
gravitational potential energy in the form of X-rays \citep{lewin1997x}. It is
thought that these LMXBs are the progenitors of the MSPs and recent results of
`transitional systems' (see for example \citep{archibald2009radio}) which
switch between the two seem to confirm this.

We include one final class of neutron stars, \emph{magnetars}, thought to have
large magnetic fields of $B\gtrsim 5\times10^{13}$~G. These are in fact
observed from two channels which we will now describe. Some pulsars are
observed to emit X-ray radiation; usually this is powered by the accretion of
matter from a binary companion, but this mechanism does not apply to isolated
stars. Subsequently, isolated stars observed in the X-ray band where named
\emph{anomalous X-ray pulsars} (AXPs). It was shown by
\citet{duncan1996magnetars} that AXPs are magnetars where the emission is
powered by the decay of the strong magnetic field.  At the same time,
astronomers found a class of objects emitting irregular bursts of
$\gamma$-rays or X-rays which they named the \emph{soft $\gamma$-ray repeaters}
(SGRs). As discussed in \citet{kouveliotou2003magnetars} these are now
understood to be magnetars which undergo rearrangement of their magnetic
fields. The two individual observations where unified by observation of X-ray
bursts from AXPs by \citet{Gavriil2002}. In Figure~\ref{fig: Period_PeriodDot} we
label observations from both these sources as magnetars.


\section{The physics of rotation powered pulsars} 
\label{sec: rotation powered pulsars}
Typically observers can measure only the period and a few derivatives from a
pulsar (the method to do this is described in section \ref{sec: pulsar timing
methods}).  We now consider what can be learnt from the observed $P$ and
$\Pdot$ if we assume a simple dipole spindown model. In figure \ref{fig:
DipoleSpindownSimple} we illustrate a rotating rigid body with a dipole fixed
at an angle $\alpha$ to the rotation axis; we will assume the body rotates in
vacuum. 

\begin{figure}[htb]
    \centering
    \includegraphics[width=.5\textwidth]{DipoleModelSimple}
    \caption{An illustration of the dipole spindown model. The dipole and some 
    of the closed field lines are fixed at an angle $\alpha$ to the rotation 
    axis. As the body rotates, radiation is emitted along both ends of the dipole
    axis producing a torque on the body.}
    \label{fig: DipoleSpindownSimple}
\end{figure}

The rotational energy of a body spinning at $\Omega$ with a moment of inertia
$I_{0}$ is given by
\begin{equation}
    E = \frac{1}{2}I_{0}\Omega^{2}.
\end{equation}
Differentiating this expression, we equate this loss of rotational energy
to the rate at which a magnetic dipole in free space radiates energy as found
by \citet{Pacini1967}:
\begin{equation}
    I_{0}\Omega\dot{\Omega} = 
                    -\frac{2\Omega^{4}}{3 c^{3}} R^{6} B_{0}^{2}\sin^{2}\alpha,
    \label{eqn: equate dKE to dEM} 
\end{equation}
where $R$ is the radius of the NS, $B_{0}$ is the surface magnetic field, and
$\alpha$ is the angle between the dipole and the rotation axis.  

In this simple model we can interpret the type of spindown by defining a power
law dependence for the braking
\begin{equation}
    \dot{\Omega} = -k \Omega^{n},
    \label{eqn: power law spindown}
\end{equation}
where $k$ is constant of proportionality, $\dot{\Omega}$ is the spindown, and
$n$ defines the \emph{braking index}. Rearranging equation 
\eqref{eqn: equate dKE to dEM} to compare with this power law, we find that for
magnetic dipole spindown:
\begin{align}
    n = 3 && \textrm{ and } && k = \frac{2B_{0}^{2}R^{6} \sin^{2}\alpha}{3 c^{3} I_{0}}.
    \label{eqn: n and k}
\end{align}
\textcolor{red}{This is different to Shapiro by a factor of 1/4}\\
For gravitational wave powered spindown, it can be shown that $n=5$
(see \citet{Shapiro83}). This suggests a powerful method to determine the spindown
mechanism if the braking index can be measured. This can be done if
$\ddot{\Omega}$ can be measured: rearranging equation \eqref{eqn: power law
spindown} to give
\begin{equation}
    n = \frac{\ddot{\Omega}\Omega}{\dot{\Omega}^{2}}.
    \label{eqn: measured braking index}
\end{equation}
The results of this are are discussed further in section \ref{sec: evidence from
    anomalous braking indices}.

For a pulsar powered by rotation, equation \eqref{eqn: power law spindown} can
be integrated between the initial values ($t=0, \Omega=\Omega_{i}$) and the
observed value ($\Omega_{o}$) to give an expression for the age of the pulsar
\begin{equation}
    t = \frac{1}{(1-n)} \frac{\Omega_{o}}{\dot\Omega_{o}} 
        \left(1 - \frac{\Omega_{o}^{n-1}}{\Omega_{i}^{n-1}}\right).
\label{eqn: characteristic age}
\end{equation}
Assuming magnetic dipole braking ($n=3$) and $\Omega_{i} \gg \Omega_{o}$ we can
approximate to a characteristic age 
\begin{equation}
    \tau = \frac{-1}{2}\frac{\Omega_{\textrm{o}}}{\dot\Omega_{o}}
         = \frac{1}{2}\frac{P}{\Pdot}.
\end{equation}
Where $P=\frac{2\pi}{\Omega}$ is the pulse period and
$\dot{P}=-2\pi\frac{\dot{\Omega}}{\Omega^{2}}$ is the period derivative.
Similarly
rearranging equations \eqref{eqn: power law spindown} and 
\eqref{eqn: n and k} we can estimate the surface magnetic field strength by
\begin{equation}
    B_{0} = \left(\frac{3 c^{3} I_{0}}{2R^{6} \sin^{2}\alpha}\right)^{\frac{1}{2}} 
            \left(\frac{-\dot{\Omega}}{\Omega^{3}}\right)^{\frac{1}{2}}
          = \frac{1}{2\pi}\left(\frac{3 c^{3} I_{0}}{2R^{6} \sin^{2}\alpha}\right)^{\frac{1}{2}}
           \sqrt{P \Pdot}
\label{eqn: surface magnetic field}
\end{equation}
In general we do not know the inclination angle $\alpha$, but we can evaluate a 
minimum magnetic field strength by setting $\alpha=\pi/2$. In CGS units for a
canonical pulsar with $R=10^{6}$~cm, $I_{0}=10^{45}$~g~cm$^{2}$ we can approximate
the magnetic field strength as
\begin{equation}
    B_{0} = 3.2 \times 10^{19} \sqrt{P \Pdot}.
\label{eqn: surface magnetic field canonical}
\end{equation}




\FloatBarrier
%\section{The importance of neutron stars}
%%Neutron stars fill an important role in fundamental physics. Their extreme
%%densities but low temperatures make then ideal laborotories to explore this
%%region of the QCD phase diagram. In particular the study of the equation of
%%state at nuclear densities. Their extreme compactness 
%
%\section{Neutron stars as a source of gravitational waves}

\section{Pulsar timing methods}
\label{sec: pulsar timing methods}
Pulsars can be observed by measuring the variation in amplitude of the radio
waves from a particular sky location. A single observation consists of measuring
the amplitude over a period of approximately $30$ minutes or so, which for pulsars
with periods $\sim 1$~s, means recording up to several thousand individual pulsations.

The shape of individual pulses can vary substantially during a single
observation; to show this, in Fig.~\ref{fig: CP1919 stacked}, successive pulses
from the first pulsar discovered, PSR B1919+21, are vertically stacked and
aligned illustrating typical variations.
\begin{figure}[htb]
\centering
\includegraphics[width=0.5\textwidth]{CP1919_stacked}
\caption{The radio amplitude of successive pulses from PSR~B1919+21 stacked
vertically, figure reproduced from \citet{mitton1977cambridge}, originally
produced by \citet{craft1970}.}
\label{fig: CP1919 stacked}
\end{figure}
Each pulsation lasts for a small fraction of the pulse period. As an example,
the pulses in PSR~B1919+21, shown in Fig.~\ref{fig: CP1919 stacked}, have
typical widths of $0.031$~s, but the pulse period is $1.337$~s; note that the
stacked plot truncates each pulsation to show only the pulsation itself. We will
demonstrate this is true for the whole pulsar population in
Sec.~\ref{sec:}.

In order to understand the gross features of a pulsar, astronomers average over
the hundreds to thousands of pulses observed during a single observation to
create a single integrated pulse profile. This is done by sampling the radio
signal at fixed time intervals then `folding' all the samples at the pulse
period (for a complete review see \citet{Lyne2012book}). The integrated pulse
profile of a pulsar looks similar to any of the individual pulses seen in
Fig.~\ref{fig: CP1919 stacked}, but in contrast to the individual pulses which
create it, it can be highly stable when measured independently over timescales
of years.

The integrated pulse profile not only gives a stable picture of what the
pulsations look like on average, but it also provides a highly accurate
measurement of the time of arrival (TOA) of a single pulse during the
observation; which pulse depends on how the folding is done. It is this TOA
which can be used to `time' a pulsar. To do this, regular observations of a
pulsar must be made every few months or so, each observation results in a
precise TOA measurement. Having obtained a series of TOAs, pulsar astronomers
generate a \emph{timing model} which attempts to exactly count each and every
pulse. Between any two observations there may be several million pulses so the
timing model needs to account for any mechanisms which may produce variations
in the TOAs.  The process is standardised by the software package
\texttt{TEMPO2} developed by \citet{Hobbs2006}, we will now describe the
essential features.

The TOA of a pulse at the detector on Earth depends on many factors such as the
time at which the beam was directed by source towards the Earth, the relative
motions of the source and detector, and any mechanisms effecting the signal during
its transit. The time at which pulses are generated (when the source beams towards
the earth) are governed by the \emph{timing properties} of the star itself.
As described by \citet{Edwards2006} these can be modelled by a Taylor expansion
in the phase at time $t$ given by
\begin{equation}
\phi(t) = \sum_{n\ge 1}\frac{\nu^{(n-1)}}{n!}(t - \tref)^{n} + \phi_{0}.
\label{eqn: Taylor compact}
\end{equation} 
where $\{\nu^{(n)}\}$ is the frequency $\nu$, spin-down rate $\dot{\nu}$, etc.
of the rotating body, $\phi_0$ is the initial phase, and $\tref$ is an
arbitrary reference time. This expansion is usually truncated at $n=3$, the
second order spin-down rate $\ddot{\nu}$. The timing model then includes
corrections to this to model the relative motion of the source and detector,
intergalactic transit, and other effects; these are described in full in
\citet{Edwards2006}. 

Between any two TOAs, if the timing model is correct, an integer number of
rotations must have occurred; this allows the use of the deviation of
$\phi(t_{TOA}^{j})$ from an integer as a test statistic. The timing model
minimises the RMS of these deviations with respect to the timing model
parameters, for example the frequency and frequency derivatives. The output of
applying a particular timing model (choice of corrections) and a set of data is
then the best-fit of these parameters and an estimate of their associated
errors.  The corrections applied in a timing model provide a method to
investigate pulsar physics: for example in some pulsar an orbital correction
must be applied which models the periodic motion of the star due to an orbital
companion. Using this technique, \citet{wolszczan1992planetary} discovered the
first \emph{exoplanet} orbiting the pulsar PSR~B1257+12.

A minimisation of the timing model parameters will converge regardless of
whether of not the model itself is appropriate. To qualitatively check this,
pulsar astronomers refer to the \emph{timing residual}, which is the difference
between the TOA, as given by the timing model, and the actual TOA. The timing
residual provides a mechanism to evaluate the timing model: a periodic
variation in the timing residual with period $365$~days, may indicate the correction
of the Earth's orbit about the Sun may be incorrect.

If the timing model is accurate enough to track the pulsar to within a single
rotation the resulting timing solution is described as \emph{phase-connected}.
For most pulsars this is the case and a single set of coefficients can track
the spindown over periods greater than a year. For well timed pulsars the
timing residual will display a `white` noise with a mean of zero. However, for
many pulsars the timing residual contains `structure' known as \emph{timing
variations} which cannot be associated with any known correction, this is the
focus of this work and we describe the phenomenon in Sec.~\ref{sec: timing
variations}. In the next section, we will study the variety of known pulsars
which have been timed using this method.




\section{Glitches}
\label{ref: glitches}
In addition to the regular spin-down of radio pulsars due to magnetic braking,
some pulsars undergo anomalies in their timing solutions known as
\emph{glitches}.  These are sudden rapid increases in the pulsation frequency
which were first observed in the Crab \citep{Boynton1969, Richards1969} and
Vela pulsars \citep{RadhakrishnanManchester1969, Reichley1969}. Pulsar
timing methods model this as a permanent increase in the phase, frequency, and
first frequency derivative in addition to a frequency increment that
subsequently decays exponentially to zero \citep{Edwards2006}.  To model this,
for each glitch pulsar astronomers add on an additional term to
Eqn.~\eqref{eqn: Taylor compact}:
\begin{align}
\begin{split}
\phi_{\textrm{g}} = H(\tTOA-\tg)& {}\left(
\Delta\phi + \Delta\nu(\tTOA - \tg) + \frac{\Delta\dot{\nu}}{2}(\tTOA - \tg)^{2} \right.\\
& \hspace{3mm}+ \left.\left[1-\exp\left(-\frac{\tTOA - \tg}{\tau}\right)\right]\Delta\nu_{\textrm{t}}(\tTOA - \tg)
\right),
\end{split}
\label{eqn: glitch timing model}
\end{align}
where $H(t)$ is the Heaviside step function. The first
three terms are the permanent increase in phase, frequency, and spin-down,
while the last term gives the transient increase in the frequency
$\Delta\nu_{\textrm{t}}$ which decays exponentially with a timescale $\tau$.
To illustrate this, in Fig.~\ref{fig: glitch sketch} we show the spin-frequency
model of a glitch including a permanent increase in frequency $\Delta\nu$ and
a component $\Delta\nu_{\textrm{t}}$ which is `recovered'.
\begin{figure}[htb]
\centering
\includegraphics[]{glitch_sketch}
\caption{Illustration of the glitch model fitted by pulsar astronomers.}
\label{glitch sketch}
\end{figure}

In effect, pulsar astronomers fit separate Taylor expansions either side of the
glitch. This is a good model when the rise-time of the glitch, during which the
frequency increases, is short compared to the duration between observations.
This evolution of the frequency during a glitch has yet to be observed, but
high time resolution monitoring of the Vela pulsar placed an upper limit of
40~s for the rise-time between the original and the new period
\citep{Dodson2001}. Since we cannot resolve the glitch itself,
Eqn.~\eqref{eqn: glitch timing model} is appropriate and used for all known
glitches.

A comprehensive review of glitches was carried out by
\citet{Espinoza2011}; to illustrate a typical glitch, in Fig.~\ref{fig: glitch}
we reproduce data from this review on a glitch in the Crab pulsar.
\begin{figure}[htb]
    \centering
    \includegraphics[width=0.5\textwidth]{GlitchExample_jbman}
    \caption{
A glitch in the PSR B0531+21, the Crab pulsar. It occurred around MJD
53067 and had a fractional frequency jump of $\Delta\nu/\nu = 5.33 \pm 0.05
\times 10^{−9}$. (a) The timing residuals relative to a
slowdown model with two frequency derivatives when fitting data only up to the
glitch date. (b) Timing residuals after fitting all data in the plot; note
that the glitch feature is still visible. Both these panels have the same
scale, covering 500 ms. (c) Frequency residuals, obtained by subtracting the
main slope given by an average $\dot\nu$. (d) The behaviour of $\dot\nu$
through the glitch. This figure and caption are taken from Fig.~1 of
\citet{Espinoza2011}}
    \label{fig: glitch}
\end{figure}

Over 165 of the 2000 observed pulsars have been seen to glitch, often multiple
times. Typical values of the instantaneous frequency change range form
$10^{-9}$~Hz to $10^{-4}$~Hz For some pulsars this is accompanied by a change,
with either sign, in the spin-down rate $\Delta\dot{\nu}$ with absolute
magnitudes between $10^{-19}$~Hz/s to $10^{-12}$~Hz/s.
In most pulsars, the fraction of the change in frequency
\begin{align}
Q = \frac{\Delta\nu_\textrm{t}}{\Delta\nu + \Delta\nu_\textrm{t}},
\end{align}
can not be measured. This may be due either to infrequent observations of the
pulsar, or simply that $Q\ll1$.  A review of pulsars with measured values of
$Q$ was conducted by \citet{Lyne2000}; they found that in glitches from 18
pulsars, $Q$ correlates with $|\dot{\nu}|$ reaching values as large as
$\sim0.9$ for the youngest pulsar with the highest absolute spin-down rate,
the Crab pulsar.

Many pulsars have been observed to glitch several times, \citet{Melatos2008}
considered the waiting times between glitches and concluded that in most
glitching pulsars the glitches happen randomly with waiting times consistent
with a Poisson process, except in PSR J0537-6910 and PSR J0835-4510 which
displayed quasi-periodicity in the waiting times.

In Chapter~\ref{sec: glitches in cgw}, we perform our own investigation into the
population statistics of glitches with an aim to understand their implication
for gravitational wave searches. We find, in agreement with
\citet{Espinoza2011} and references therein, that the distribution of glitch
magnitudes has multiple modes which suggests that glitches may come from more
than one mechanism. We go on to apply a statistical model and determine, in an
empirical fashion, the properties of the underlying source populations.

Glitches provide a unique opportunity to investigate the physics of neutron
stars and many of the leading insights have been gained by their study. Two
leading models exist known as the \emph{superfluid unpinning} model and the
\emph{starquake} model.

In the superfluid unpinning model proposed by \citet{Anderson1975}, the star
contains a superfluid component in which the angular momentum is stored in an
array of vortices which are `pinned' to the crust. The magnetic dipole, rigidly
fixed to the crust, exerts a torque on the crust gradually spinning it down.
The superfluid component cannot decrease its angular momentum without
destroying the vortices into which it stores angular momentum and so does not
spin-down at the same rate.  A lag in frequency between the superfluid
component and rest of the crust develops until the forces are sufficiently
large to cause an avalanche of unpinning events rapidly transferring the stored
angular momentum in the superfluid component to the crust. The observed
pulsations measure the rotation rate of the crust into which the dipole is
frozen, so when this unpinning occurs, we see a rapid increase in the
frequency.

The second model, starquakes, follows from the observation that a rapidly
spinning fluid body has an oblate `rest shape' with a bulge about its equator
due to the centrifugal force. The crust of a star spinning at a frequency $\nu_0$
will solidify with a corresponding oblateness. Subsequently, as the star spins down,
it will have a different rest shape due to its decreased frequency, but the crust
will retain a memory of the earlier shape at which it solidified. This will cause
strains in the crust which eventually cause a starquake relieving the strain,
resetting the reference rest shape, and producing glitch like features. This
model was first proposed by \citet{Ruderman1969} and later built upon by
\citet{Baym1971}.

Both of these models have support in the literature and have been developed
significantly to explain the variety of observed glitches. However, there are
observations which cause difficulties for both models: glitches seen in the
Vela pulsar are too large and too often to be consistent with a starquakes
model, while the unpinning model requires a superfluid component which is at
odds with observation of precession (we discuss this further in
Chapter~\ref{sec: testing models}). In this thesis, we will not use glitches
as a tool for inferring neutron star physics, but any predictions we do make
must be compatible with what has already been learnt from glitches.




\FloatBarrier
\section{Timing noise}
\label{sec: timing noise}
Timing noise refers to small, low level structure in the timing residual which
cannot be attributed to any other source, an example is given in figure
\ref{fig: timing noise example}. The amount of timing noise will depend on the
order of truncation of the Taylor expansion.  In most studies, and so in this
work, we will truncate at $n=3$, the second order spindown; the timing noise is
then the residual left after a $n=3$ fit. Increasing this truncation level can
fit out these variations, but we are interested in understanding the origins
and implications of timing noise.

\begin{figure}[htb] 
    \centering
    \includegraphics[width=0.75\textwidth]{PhaseResidual_45000_47000.pdf}
    \caption{A phase residual demonstrating structure which is named timing
        noise.  This is generated from data on the Crab ephemeris (see section
    \ref{sec: timing noise as described by the crab ephemeris})}
    \label{fig: timing noise example}
\end{figure} 

There have been various attempts to characterise timing noise
over the history of 
pulsar astronomy. Most observations are tied to a particular interpretation
which we will discuss in part \ref{sec: interpretations of timing noise}.
It is however worth first listing some of the model independent observations.
These are summarised by the most comprehensive study of timing noise from 
\citet{Hobbs2010} who considered 366 pulsars over time scales $\gtrsim10$~years.
They found that:
\begin{enumerate}

    \item Timing noise is widespread in pulsars

    \item Timing noise is inversely correlated the characteristic age
        defined in equation \ref{eqn: characteristic age}

    \item The structures seen in the timing residual vary with data span. As
        more data is collected, more quasi-periodic features are observed.

    \item The dominant contribution to timing noise for young pulsars with
        $\tau_{c}<10^{5}$~years can be explained as being caused by the
        recovery from previous glitches.

    \item A handful of pulsars exhibit significant periodicity's while
        quasi-periodicity's are observed in many pulsars

\end{enumerate}




\biblio


\end{document}
