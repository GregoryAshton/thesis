\documentclass[../full_thesis/full_thesis.tex]{subfiles}

% Default image directory
\newcommand{\thisdir}{../introduction}
\graphicspath{{\thisdir/img/}} 

\begin{document} 

The story of a \emph{neutron star} (NS) begins with the death of a main-sequence star
in a supernova event. Prior to this, the nuclear fusion of hydrogen atoms into
helium provides pressure supporting the star in an equilibrium configuration
with the inward pressure of the stars self-gravity. Eventually the star depletes
its reserves of hydrogen and can no longer maintain the equilibrium. If the
star has an initial mass greater than $\sim 8 \Msun$, then it may undergo a
\emph{core-collapse supernova} during which the temperatures and pressure rapidly
increase. Under these conditions the electrons and protons undergo inverse
beta decay combining to form neutrons and neutrinos
\begin{equation}
    e^{-} + p \rightarrow n + \nu.
\end{equation}
Once the pressures reaches nuclear densities, neutron degeneracy pressure can
halt the collapse in a new equilibrium configuration; the resulting remnant is
a neutron star. However, if the remnant has a mass greater than $\sim 5 \Msun$,
neutron degeneracy pressure will be unable to support the gravitational
pressure and it will collapse to form a black hole.

The idea of neutrons stars was first postulated by Landau as `dense stars
which look like giant atomic nuclei' \citep{Yakovlev2013} even before the
discovery of the neutron by \citep{Chadwick1932}. However it was
\citet{Baade1934} who made the explicit prediction of a neutron star whilst
trying to explain the energy released in supernova explosions.

The maximum radius such a star can support is $\sim 10$~km but the mass
compressed into this volume is $\sim M_{\odot}$.  They can rotate with
frequencies up to 700~Hz and exist as either \emph{isolated} or as part of a
binary system.  In either of these systems their compactness means they are
potential sources of detectable gravitational waves. In fact the orbital decay
of the binary neutron star system PSR~1913+16 discovered by \citet{Hulse1975}
and subsequently analysed by \citet{Taylor1982} provided the first evidence for
gravitational waves.  NSs can have extreme magnetic fields with estimates
strengths up to $10^{15}$ gauss. 

Our knowledge of neutron stars is founded on observing them in the variety of
system detectable by astronomy. In this thesis we will investigate how we can
use current electromagnetic observations to further our understanding of neutron
star physics by investigating the causes of so-called \emph{timing-noise}.
Detecting gravitational waves from isolated neutron stars will provide a second
channel to investigate neutron star physics; in this thesis I will also
investigate the importance timing-noise and glitches may have on the ability to
detect such gravitational waves.

In this introductory chapter, we will describe current understanding of neutron
stars concentrating on those observed as 'pulsars'. The basic astrophysics will
be introduced along with the methods used to collect data. Finally we will describe
two phenomena, glitches and timing-noise, which provide an opportunity to
probe the neutron star physics.


\section{Observation of pulsars and their identification with neutron stars}
A bright periodic EM signal was identified by \citet{Hewish1968} during a high
time-resolution survey for interplanetary scintillation. The shortness of
pulses and frequent and precise periodicity suggested that the source was
small. For a star to remain gravitationally bound, its rotation frequency is
constrained by the requirement that the centrifugal acceleration at the equator
be less than the gravitational acceleration. The observed pulses had a short
period $\sim 1.33$~s which ruled out white dwarfs since they could rotate this
fast without becoming gravitationally unbound. After the identification of
other similar objects these sources were named \emph{pulsars}. 

While multiple interpretation's were posed to explain the observation, the
identification of pulsars with neutron stars came from \citet{Pacini1967} and
independently \citet{Gold1968}. They suggested that a rapidly rotating neutron
star with a strong dipolar magnetic field would stream radiation out along the
magnetic axis. If this axis was misaligned from the rotation axis then the
beams would be swept out like a lighthouse. Beams passing over the earth are
observed as periodic pulses at the rotation frequency of the star.  This
identification was confirmed by the discovery of a pulsar at the centre of Crab
nebulae, a supernova remnant, agreeing with the prediction of \citet{Baade1934}.




\section{Categorising neutron stars}
The timing properties, and other features measured by the timing model, for
over 2000 pulsars observed can be accessed via the Australia Telescope National
Facility (ATNF) pulsar catalogue \citep{ATNF}.  We can categorise the
population by their measured values of period $P$ and the period derivative
$\Pdot$. This is done by plotting them in a so-called $P - \Pdot$ diagram, as
shown in Figure~\ref{fig: Period_PeriodDot}.  The various categories to which
each pulsar can be assigned have been marked in this plot and we now discuss
their features.

\begin{figure}[htb]
\centering
\includegraphics[]{Period_PeriodDot} 
\caption{Period -
Period derivative diagram using data taken from the ATNF pulsar catalogue
\citep{ATNF}. Dashed lines show inferred characteristic ages and 
surface magnetic fields as given by Eqn.~\eqref{eqn: tauAge definition}
and Eqn.~\eqref{eqn: surface magnetic field canonical} respectively.}
\label{fig: Period_PeriodDot}
\end{figure}

The majority of pulsars (referred to as the `normal' pulsars) are found
\emph{isolated} without a binary companion and have typical periods of
$P=10^{-2}-10^{1}$~s. These can be described as \emph{rotation powered}
pulsars, since the electromagnetic (EM) radiation is powered by the loss of
rotational energy. As described later in Section~\ref{sec: rotation powered
pulsars}, estimates can be made of their characteristic age, $\tauAge$, and
surface magnetic field strength, $B_{0}$, based on a dipole spin-down model.
Constant lines of these quantities are plotted in Figure~\ref{fig:
Period_PeriodDot}. Of the normal pulsars, we can choose to identify the young
pulsars as those for which $\tauAge<10^{5}$~yrs. Some of these, such as the
Crab and Vela pulsars, can be directly associated with their supernova remnant
from which they were formed \citep{Kaspi1996}.

A second smaller population of isolated rotation powered pulsars exists with
$P<10^{-1}$~s. These are known as the \emph{millisecond pulsars} (MSPs). This
special class of pulsars are believed to start their life as normal pulsars,
but are then spun-up through accretion from a normal star. In support of this
hypothesis, the majority of MSPs in Figure~\ref{fig: Period_PeriodDot} have a
binary companion \citep{wijnands1998millisecond}. Additionally, we see so-called
low-mass X-ray binary systems (LMXBs) which are systems where a neutron star in
a binary accretes matter from its companion; the infalling matter releases
gravitational potential energy in the form of X-rays \citep{lewin1997x}. It is
thought that these LMXBs are the progenitors of the MSPs and recent results of
`transitional systems' (see for example \citep{archibald2009radio}) which
switch between the two seem to confirm this.

We include one final class of neutron stars, \emph{magnetars}, thought to have
large magnetic fields of $B\gtrsim 5\times10^{13}$~G. These are in fact
observed from two channels which we will now describe. Some pulsars are
observed to emit X-ray radiation; usually this is powered by the accretion of
matter from a binary companion, but this mechanism does not apply to isolated
stars. Subsequently, isolated stars observed in the X-ray band where named
\emph{anomalous X-ray pulsars} (AXPs). It was shown by
\citet{duncan1996magnetars} that AXPs are magnetars where the emission is
powered by the decay of the strong magnetic field.  At the same time,
astronomers found a class of objects emitting irregular bursts of
$\gamma$-rays or X-rays which they named the \emph{soft $\gamma$-ray repeaters}
(SGRs). As discussed in \citet{kouveliotou2003magnetars} these are now
understood to be magnetars which undergo rearrangement of their magnetic
fields. The two individual observations where unified by observation of X-ray
bursts from AXPs by \citet{Gavriil2002}. In Figure~\ref{fig: Period_PeriodDot} we
label observations from both these sources as magnetars.


\section{The physics of rotation powered pulsars} 
\label{sec: rotation powered pulsars}
Typically observers can measure only the period and a few derivatives from a
pulsar (the method to do this is described in section \ref{sec: pulsar timing
methods}).  We now consider what can be learnt from the observed $P$ and
$\Pdot$ if we assume a simple dipole spindown model. In figure \ref{fig:
DipoleSpindownSimple} we illustrate a rotating rigid body with a dipole fixed
at an angle $\alpha$ to the rotation axis; we will assume the body rotates in
vacuum. 

\begin{figure}[htb]
    \centering
    \includegraphics[width=.5\textwidth]{DipoleModelSimple}
    \caption{An illustration of the dipole spindown model. The dipole and some 
    of the closed field lines are fixed at an angle $\alpha$ to the rotation 
    axis. As the body rotates, radiation is emitted along both ends of the dipole
    axis producing a torque on the body.}
    \label{fig: DipoleSpindownSimple}
\end{figure}

The rotational energy of a body spinning at $\Omega$ with a moment of inertia
$I_{0}$ is given by
\begin{equation}
    E = \frac{1}{2}I_{0}\Omega^{2}.
\end{equation}
Differentiating this expression, we equate this loss of rotational energy
to the rate at which a magnetic dipole in free space radiates energy as found
by \citet{Pacini1967}:
\begin{equation}
    I_{0}\Omega\dot{\Omega} = 
                    -\frac{2\Omega^{4}}{3 c^{3}} R^{6} B_{0}^{2}\sin^{2}\alpha,
    \label{eqn: equate dKE to dEM} 
\end{equation}
where $R$ is the radius of the NS, $B_{0}$ is the surface magnetic field, and
$\alpha$ is the angle between the dipole and the rotation axis.  

In this simple model we can interpret the type of spindown by defining a power
law dependence for the braking
\begin{equation}
    \dot{\Omega} = -k \Omega^{n},
    \label{eqn: power law spindown}
\end{equation}
where $k$ is constant of proportionality, $\dot{\Omega}$ is the spindown, and
$n$ defines the \emph{braking index}. Rearranging equation 
\eqref{eqn: equate dKE to dEM} to compare with this power law, we find that for
magnetic dipole spindown:
\begin{align}
    n = 3 && \textrm{ and } && k = \frac{2B_{0}^{2}R^{6} \sin^{2}\alpha}{3 c^{3} I_{0}}.
    \label{eqn: n and k}
\end{align}
\textcolor{red}{This is different to Shapiro by a factor of 1/4}\\
For gravitational wave powered spindown, it can be shown that $n=5$
(see \citet{Shapiro83}). This suggests a powerful method to determine the spindown
mechanism if the braking index can be measured. This can be done if
$\ddot{\Omega}$ can be measured: rearranging equation \eqref{eqn: power law
spindown} to give
\begin{equation}
    n = \frac{\ddot{\Omega}\Omega}{\dot{\Omega}^{2}}.
    \label{eqn: measured braking index}
\end{equation}
The results of this are are discussed further in section \ref{sec: evidence from
    anomalous braking indices}.

For a pulsar powered by rotation, equation \eqref{eqn: power law spindown} can
be integrated between the initial values ($t=0, \Omega=\Omega_{i}$) and the
observed value ($\Omega_{o}$) to give an expression for the age of the pulsar
\begin{equation}
    t = \frac{1}{(1-n)} \frac{\Omega_{o}}{\dot\Omega_{o}} 
        \left(1 - \frac{\Omega_{o}^{n-1}}{\Omega_{i}^{n-1}}\right).
\label{eqn: characteristic age}
\end{equation}
Assuming magnetic dipole braking ($n=3$) and $\Omega_{i} \gg \Omega_{o}$ we can
approximate to a characteristic age 
\begin{equation}
    \tau = \frac{-1}{2}\frac{\Omega_{\textrm{o}}}{\dot\Omega_{o}}
         = \frac{1}{2}\frac{P}{\Pdot}.
\end{equation}
Where $P=\frac{2\pi}{\Omega}$ is the pulse period and
$\dot{P}=-2\pi\frac{\dot{\Omega}}{\Omega^{2}}$ is the period derivative.
Similarly
rearranging equations \eqref{eqn: power law spindown} and 
\eqref{eqn: n and k} we can estimate the surface magnetic field strength by
\begin{equation}
    B_{0} = \left(\frac{3 c^{3} I_{0}}{2R^{6} \sin^{2}\alpha}\right)^{\frac{1}{2}} 
            \left(\frac{-\dot{\Omega}}{\Omega^{3}}\right)^{\frac{1}{2}}
          = \frac{1}{2\pi}\left(\frac{3 c^{3} I_{0}}{2R^{6} \sin^{2}\alpha}\right)^{\frac{1}{2}}
           \sqrt{P \Pdot}
\label{eqn: surface magnetic field}
\end{equation}
In general we do not know the inclination angle $\alpha$, but we can evaluate a 
minimum magnetic field strength by setting $\alpha=\pi/2$. In CGS units for a
canonical pulsar with $R=10^{6}$~cm, $I_{0}=10^{45}$~g~cm$^{2}$ we can approximate
the magnetic field strength as
\begin{equation}
    B_{0} = 3.2 \times 10^{19} \sqrt{P \Pdot}.
\label{eqn: surface magnetic field canonical}
\end{equation}




\FloatBarrier
%\section{The importance of neutron stars}
%%Neutron stars fill an important role in fundamental physics. Their extreme
%%densities but low temperatures make then ideal laborotories to explore this
%%region of the QCD phase diagram. In particular the study of the equation of
%%state at nuclear densities. Their extreme compactness 
%
%\section{Neutron stars as a source of gravitational waves}

\section{Pulsar timing methods}
\label{sec: pulsar timing methods}
Pulsars can be observed by measuring the variation in amplitude of the radio
waves from a particular sky location. A single observation consists of measuring
the amplitude over a period of approximately $30$ minutes or so, which for pulsars
with periods $\sim 1$~s, means recording up to several thousand individual pulsations.

The shape of individual pulses can vary substantially during a single
observation; to show this, in Fig.~\ref{fig: CP1919 stacked}, successive pulses
from the first pulsar discovered, PSR B1919+21, are vertically stacked and
aligned illustrating typical variations.
\begin{figure}[htb]
\centering
\includegraphics[width=0.5\textwidth]{CP1919_stacked}
\caption{The radio amplitude of successive pulses from PSR~B1919+21 stacked
vertically, figure reproduced from \citet{mitton1977cambridge}, originally
produced by \citet{craft1970}.}
\label{fig: CP1919 stacked}
\end{figure}
Each pulsation lasts for a small fraction of the pulse period. As an example,
the pulses in PSR~B1919+21, shown in Fig.~\ref{fig: CP1919 stacked}, have
typical widths of $0.031$~s, but the pulse period is $1.337$~s; note that the
stacked plot truncates each pulsation to show only the pulsation itself. We will
demonstrate this is true for the whole pulsar population in
Sec.~\ref{sec:}.

In order to understand the gross features of a pulsar, astronomers average over
the hundreds to thousands of pulses observed during a single observation to
create a single integrated pulse profile. This is done by sampling the radio
signal at fixed time intervals then `folding' all the samples at the pulse
period (for a complete review see \citet{Lyne2012book}). The integrated pulse
profile of a pulsar looks similar to any of the individual pulses seen in
Fig.~\ref{fig: CP1919 stacked}, but in contrast to the individual pulses which
create it, it can be highly stable when measured independently over timescales
of years.

The integrated pulse profile not only gives a stable picture of what the
pulsations look like on average, but it also provides a highly accurate
measurement of the time of arrival (TOA) of a single pulse during the
observation; which pulse depends on how the folding is done. It is this TOA
which can be used to `time' a pulsar. To do this, regular observations of a
pulsar must be made every few months or so, each observation results in a
precise TOA measurement. Having obtained a series of TOAs, pulsar astronomers
generate a \emph{timing model} which attempts to exactly count each and every
pulse. Between any two observations there may be several million pulses so the
timing model needs to account for any mechanisms which may produce variations
in the TOAs.  The process is standardised by the software package
\texttt{TEMPO2} developed by \citet{Hobbs2006}, we will now describe the
essential features.

The TOA of a pulse at the detector on Earth depends on many factors such as the
time at which the beam was directed by source towards the Earth, the relative
motions of the source and detector, and any mechanisms effecting the signal during
its transit. The time at which pulses are generated (when the source beams towards
the earth) are governed by the \emph{timing properties} of the star itself.
As described by \citet{Edwards2006} these can be modelled by a Taylor expansion
in the phase at time $t$ given by
\begin{equation}
\phi(t) = \sum_{n\ge 1}\frac{\nu^{(n-1)}}{n!}(t - \tref)^{n} + \phi_{0}.
\label{eqn: Taylor compact}
\end{equation} 
where $\{\nu^{(n)}\}$ is the frequency $\nu$, spin-down rate $\dot{\nu}$, etc.
of the rotating body, $\phi_0$ is the initial phase, and $\tref$ is an
arbitrary reference time. This expansion is usually truncated at $n=3$, the
second order spin-down rate $\ddot{\nu}$. The timing model then includes
corrections to this to model the relative motion of the source and detector,
intergalactic transit, and other effects; these are described in full in
\citet{Edwards2006}. 

Between any two TOAs, if the timing model is correct, an integer number of
rotations must have occurred; this allows the use of the deviation of
$\phi(t_{TOA}^{j})$ from an integer as a test statistic. The timing model
minimises the RMS of these deviations with respect to the timing model
parameters, for example the frequency and frequency derivatives. The output of
applying a particular timing model (choice of corrections) and a set of data is
then the best-fit of these parameters and an estimate of their associated
errors.  The corrections applied in a timing model provide a method to
investigate pulsar physics: for example in some pulsar an orbital correction
must be applied which models the periodic motion of the star due to an orbital
companion. Using this technique, \citet{wolszczan1992planetary} discovered the
first \emph{exoplanet} orbiting the pulsar PSR~B1257+12.

A minimisation of the timing model parameters will converge regardless of
whether of not the model itself is appropriate. To qualitatively check this,
pulsar astronomers refer to the \emph{timing residual}, which is the difference
between the TOA, as given by the timing model, and the actual TOA. The timing
residual provides a mechanism to evaluate the timing model: a periodic
variation in the timing residual with period $365$~days, may indicate the correction
of the Earth's orbit about the Sun may be incorrect.

If the timing model is accurate enough to track the pulsar to within a single
rotation the resulting timing solution is described as \emph{phase-connected}.
For most pulsars this is the case and a single set of coefficients can track
the spindown over periods greater than a year. For well timed pulsars the
timing residual will display a `white` noise with a mean of zero. However, for
many pulsars the timing residual contains `structure' known as \emph{timing
variations} which cannot be associated with any known correction, this is the
focus of this work and we describe the phenomenon in Sec.~\ref{sec: timing
variations}. In the next section, we will study the variety of known pulsars
which have been timed using this method.




\section{Glitches}
\label{ref: glitches}
In addition to the regular spin-down of radio pulsars due to magnetic braking,
some pulsars undergo anomalies in their timing solutions known as
\emph{glitches}.  These are sudden rapid increases in the pulsation frequency
which were first observed in the Crab \citep{Boynton1969, Richards1969} and
Vela pulsars \citep{RadhakrishnanManchester1969, Reichley1969}. Pulsar
timing methods model this as a permanent increase in the phase, frequency, and
first frequency derivative in addition to a frequency increment that
subsequently decays exponentially to zero \citep{Edwards2006}.  To model this,
for each glitch pulsar astronomers add on an additional term to
Eqn.~\eqref{eqn: Taylor compact}:
\begin{align}
\begin{split}
\phi_{\textrm{g}} = H(\tTOA-\tg)& {}\left(
\Delta\phi + \Delta\nu(\tTOA - \tg) + \frac{\Delta\dot{\nu}}{2}(\tTOA - \tg)^{2} \right.\\
& \hspace{3mm}+ \left.\left[1-\exp\left(-\frac{\tTOA - \tg}{\tau}\right)\right]\Delta\nu_{\textrm{t}}(\tTOA - \tg)
\right),
\end{split}
\label{eqn: glitch timing model}
\end{align}
where $H(t)$ is the Heaviside step function. The first
three terms are the permanent increase in phase, frequency, and spin-down,
while the last term gives the transient increase in the frequency
$\Delta\nu_{\textrm{t}}$ which decays exponentially with a timescale $\tau$.
To illustrate this, in Fig.~\ref{fig: glitch sketch} we show the spin-frequency
model of a glitch including a permanent increase in frequency $\Delta\nu$ and
a component $\Delta\nu_{\textrm{t}}$ which is `recovered'.
\begin{figure}[htb]
\centering
\includegraphics[]{glitch_sketch}
\caption{Illustration of the glitch model fitted by pulsar astronomers.}
\label{glitch sketch}
\end{figure}

In effect, pulsar astronomers fit separate Taylor expansions either side of the
glitch. This is a good model when the rise-time of the glitch, during which the
frequency increases, is short compared to the duration between observations.
This evolution of the frequency during a glitch has yet to be observed, but
high time resolution monitoring of the Vela pulsar placed an upper limit of
40~s for the rise-time between the original and the new period
\citep{Dodson2001}. Since we cannot resolve the glitch itself,
Eqn.~\eqref{eqn: glitch timing model} is appropriate and used for all known
glitches.

A comprehensive review of glitches was carried out by
\citet{Espinoza2011}; to illustrate a typical glitch, in Fig.~\ref{fig: glitch}
we reproduce data from this review on a glitch in the Crab pulsar.
\begin{figure}[htb]
    \centering
    \includegraphics[width=0.5\textwidth]{GlitchExample_jbman}
    \caption{
A glitch in the PSR B0531+21, the Crab pulsar. It occurred around MJD
53067 and had a fractional frequency jump of $\Delta\nu/\nu = 5.33 \pm 0.05
\times 10^{−9}$. (a) The timing residuals relative to a
slowdown model with two frequency derivatives when fitting data only up to the
glitch date. (b) Timing residuals after fitting all data in the plot; note
that the glitch feature is still visible. Both these panels have the same
scale, covering 500 ms. (c) Frequency residuals, obtained by subtracting the
main slope given by an average $\dot\nu$. (d) The behaviour of $\dot\nu$
through the glitch. This figure and caption are taken from Fig.~1 of
\citet{Espinoza2011}}
    \label{fig: glitch}
\end{figure}

Over 165 of the 2000 observed pulsars have been seen to glitch, often multiple
times. Typical values of the instantaneous frequency change range form
$10^{-9}$~Hz to $10^{-4}$~Hz For some pulsars this is accompanied by a change,
with either sign, in the spin-down rate $\Delta\dot{\nu}$ with absolute
magnitudes between $10^{-19}$~Hz/s to $10^{-12}$~Hz/s.
In most pulsars, the fraction of the change in frequency
\begin{align}
Q = \frac{\Delta\nu_\textrm{t}}{\Delta\nu + \Delta\nu_\textrm{t}},
\end{align}
can not be measured. This may be due either to infrequent observations of the
pulsar, or simply that $Q\ll1$.  A review of pulsars with measured values of
$Q$ was conducted by \citet{Lyne2000}; they found that in glitches from 18
pulsars, $Q$ correlates with $|\dot{\nu}|$ reaching values as large as
$\sim0.9$ for the youngest pulsar with the highest absolute spin-down rate,
the Crab pulsar.

Many pulsars have been observed to glitch several times, \citet{Melatos2008}
considered the waiting times between glitches and concluded that in most
glitching pulsars the glitches happen randomly with waiting times consistent
with a Poisson process, except in PSR J0537-6910 and PSR J0835-4510 which
displayed quasi-periodicity in the waiting times.

In Chapter~\ref{sec: glitches in cgw}, we perform our own investigation into the
population statistics of glitches with an aim to understand their implication
for gravitational wave searches. We find, in agreement with
\citet{Espinoza2011} and references therein, that the distribution of glitch
magnitudes has multiple modes which suggests that glitches may come from more
than one mechanism. We go on to apply a statistical model and determine, in an
empirical fashion, the properties of the underlying source populations.

Glitches provide a unique opportunity to investigate the physics of neutron
stars and many of the leading insights have been gained by their study. Two
leading models exist known as the \emph{superfluid unpinning} model and the
\emph{starquake} model.

In the superfluid unpinning model proposed by \citet{Anderson1975}, the star
contains a superfluid component in which the angular momentum is stored in an
array of vortices which are `pinned' to the crust. The magnetic dipole, rigidly
fixed to the crust, exerts a torque on the crust gradually spinning it down.
The superfluid component cannot decrease its angular momentum without
destroying the vortices into which it stores angular momentum and so does not
spin-down at the same rate.  A lag in frequency between the superfluid
component and rest of the crust develops until the forces are sufficiently
large to cause an avalanche of unpinning events rapidly transferring the stored
angular momentum in the superfluid component to the crust. The observed
pulsations measure the rotation rate of the crust into which the dipole is
frozen, so when this unpinning occurs, we see a rapid increase in the
frequency.

The second model, starquakes, follows from the observation that a rapidly
spinning fluid body has an oblate `rest shape' with a bulge about its equator
due to the centrifugal force. The crust of a star spinning at a frequency $\nu_0$
will solidify with a corresponding oblateness. Subsequently, as the star spins down,
it will have a different rest shape due to its decreased frequency, but the crust
will retain a memory of the earlier shape at which it solidified. This will cause
strains in the crust which eventually cause a starquake relieving the strain,
resetting the reference rest shape, and producing glitch like features. This
model was first proposed by \citet{Ruderman1969} and later built upon by
\citet{Baym1971}.

Both of these models have support in the literature and have been developed
significantly to explain the variety of observed glitches. However, there are
observations which cause difficulties for both models: glitches seen in the
Vela pulsar are too large and too often to be consistent with a starquakes
model, while the unpinning model requires a superfluid component which is at
odds with observation of precession (we discuss this further in
Chapter~\ref{sec: testing models}). In this thesis, we will not use glitches
as a tool for inferring neutron star physics, but any predictions we do make
must be compatible with what has already been learnt from glitches.




\section{Timing noise: observations}
\label{sec: timing noise observations}
Timing noise refers to small, low level structure in the timing residual which
cannot be attributed to any other source, an example is given in figure
\ref{fig: timing noise example}. The amount of timing noise will depend on the
order of truncation of the Taylor expansion.  In most studies, and so in this
work, we will truncate at $n=3$, the second order spindown; the timing noise is
then the residual left after a $n=3$ fit. Increasing this truncation level can
fit out these variations, but we are interested in understanding the origins
and implications of timing noise.

\begin{figure}[htb] 
    \centering
    \includegraphics[width=0.75\textwidth]{PhaseResidual_45000_47000.pdf}
    \caption{A phase residual demonstrating structure which is named timing
        noise.  This is generated from data on the Crab ephemeris (see section
    \ref{sec: timing noise as described by the crab ephemeris})}
    \label{fig: timing noise example}
\end{figure} 

There have been various attempts to characterise timing noise
over the history of 
pulsar astronomy. Most observations are tied to a particular interpretation
which we will discuss in part \ref{sec: interpretations of timing noise}.
It is however worth first listing some of the model independent observations.
These are summarised by the most comprehensive study of timing noise from 
\citet{Hobbs2010} who considered 366 pulsars over time scales $\gtrsim10$~years.
They found that:
\begin{enumerate}

    \item Timing noise is widespread in pulsars

    \item Timing noise is inversely correlated the characteristic age
        defined in equation \ref{eqn: characteristic age}

    \item The structures seen in the timing residual vary with data span. As
        more data is collected, more quasi-periodic features are observed.

    \item The dominant contribution to timing noise for young pulsars with
        $\tau_{c}<10^{5}$~years can be explained as being caused by the
        recovery from previous glitches.

    \item A handful of pulsars exhibit significant periodicity's while
        quasi-periodicity's are observed in many pulsars

\end{enumerate}




\section{Timing noise: interpretations}
\label{sec: timing noise interpretations}
The underlying mechanism which causes timing noise is not understood. Since the
first discussions in \citet{Boynton1972} multiple models have been proposed
which are able to describe some of the features. However, the variety of ways
timing noise manifests and the uncertainty in the mechanisms at work have made
it difficult for any conclusive statements to be made about the models. A
complete undestanding of timing noise must not only explain the observed
variations, but also remain consistent with our understanding of neutron stars
derived from other observations such as glitches. A complicating factor in
understanding timing noise is that the observed features have timescales
similar to the duration that we have been able to observe pulsars; it is
therefore possible that observations which look like a random walk over a short
timescale, may in fact be periodic or something else entirely over longer
timescales.  Observations of new features prompt new models for timing noise;
as a result the timing noise interpretations have evolved with the
observations. In this section we will present an overview of these concepts and
the existing evidence which supports them.

\subsection{Random walk models}
\label{sec: TN interpretations random walk models}

Timing noise was first quantified and interpreted by \citet{Boynton1972} as a
Poisson like random walk in one of the phase, frequency, or spindown. The
pulsar spins down according to the power law spindown of equation \eqref{eqn:
power law spindown} except that at random times the pulse phase, frequency or
spindown jumps, or changes discontinuously. The waiting times between events
are Poisson distributed with a rate R such that over a period $T$ the number of
events follows a Poisson distribution with mean RT. All the jumps are
independent and the magnitudes are randomly distributed with means given by
$\langle\dP\rangle$, $\langle\dF\rangle$, and $\langle\dS\rangle$ for the phase
frequency and spindown. We can investigate the statistical properties of such
random walks by splitting the phase into contributions from the secular
spindown $\phi_{\mathrm{S}}$ (as given by equation \eqref{eqn: Taylor compact})
and contributions from the random walks
\begin{equation}
    \phi = \phi_{\mathrm{S}} + \Delta\phi_{\mathrm{R}}
\end{equation}
The three random walks can be written as the sum of $N$ individual events
occurring at times $t_{i}$
\begin{align}
    \Delta\phi_{\mathrm{R}} & = \s{i=1}{N}\Delta\phi_{i} H(t - t_{i}) 
     && \mathrm{(phase),} \\
    \Delta\phi_{\mathrm{R}} & = \s{i=1}{N}\Delta\f_{i} (t-t_{i})H(t - t_{i}) 
     && \mathrm{(frequency),} \\
    \Delta\phi_{\mathrm{R}} & = \s{i=1}{N}\frac{1}{2}\Delta\fdot_{i} (t-t_{i})^{2}H(t - t_{i}) 
     && \mathrm{(spindown),}
\end{align}
where $H(t)$ is unit step function at $t=0$. It should be noted that here we are
treating the three types of noise separately such that timing noise residuals 
from either a random walk in phase, frequency, or spindown; work by \citet{Cordes1980}
extended this model to handle mixing between the types of noise.

Timing noise is the remainder having
fitted and subtracted a second order Taylor expansion. Provided the perturbations
of $\phi_{\mathrm{s}}$ are small then the remainder will be exactly given by 
$\phi_{\mathrm{s}}$. Then, as described by \citet{Boynton1972} we can then 
average over: the $\dP_{i}, \dF_{i}$ or $\dS_{i}$
distributions, the $t_{i}$ distribution, and the $N$ distributions to give
\begin{align}
    \langle \Delta\phi_{R} \rangle & = \langle \dP \rangle R T 
    = S_{\mathrm{PN}}T && \mathrm{(phase),} \\
    \langle \Delta\phi_{R} \rangle & = \frac{1}{2}\langle \dF \rangle R T^{2} 
    = \frac{1}{2}S_{\mathrm{FN}}T^{2} && \mathrm{(frequency),} \\
    \langle \Delta\phi_{R} \rangle & = \frac{1}{6}\langle \dS \rangle R T^{3} 
    = \frac{1}{6}S_{\mathrm{SN}}T^{3} && \mathrm{(spindown).} \\
\end{align}
Here we have implicitly defined the strength parameters which combine the rate 
and averaged magnitude of jumps into a single quantity. Measurements of timing 
noise, if they are discreet events, will observe the accumulation of many 
individual events. As a result only the strength can be
measured, not the rate and average magnitude.

The three types of noise are distinguishable by their dependence on the
observation time $T$. The type of noise can be measured by
translating this into the dispersion measure of
$\Delta\fddot$ after fitting a cubic, or by inspection of the power spectrum.
\citet{Boynton1972} was able to categorise the Crab pulsar as frequency like noise.
They found that over a $5$~year period the noise process was stationary
and consistent with the frequency noise hypothesis. No deterministic process
could account for the timing residuals strengthening their conviction that some
random process was taking place.  However, they suggested that over longer periods
the random walk will be non-stationary due to either mixing with other types of
walks, or decay of the strength parameter with time.

The interpretation of timing noise as a Poisson random walk is a purely
statistical model. It is however backed up for a rich variety of physical
models.  A key feature of any physical
random walk model is that it must be able to produce both increases and
decreases in the relevant parameter. For this reason it is felt unlikely that
the timing noise mechanism is the same as the glitch mechanism, although they
must be related.  In addition it is unclear if timing noise is a continuous or
discreet process, certainly if it is discreet the waiting time between events
must be shorter than the shortest observation periods $\sim days$. 

The first physical model was proposed by \citet{Boynton1972}, the noise process
consisted of the as accretion of small lumps of matter onto the NS from the
interstellar medium. Lumps of matter fall randomly onto the surface of the star
causing either a spin-up or slowdown through the transfer of angular momentum.
After this many models were proposed such as starquakes and the random pinning
and unpinning of vortex lines; these were reviewed by \citet{Cordes1981} and
evaluated against observational constraints. Of these only three mechanisms
where found to be consistent with observations: crust breaking by vortex pinning, a
response to heat pulses, and luminosity related torque fluctuations. Since this
review, new random walk mechanisms have been proposed such as: variations in
the magnetospheric gap size \citep{Cheng1987}; the interference by debris entering
the magnetosphere \citep{Cordes2008}; and the accumulation of multiple micro-glitches
\citep{Janssen2006}. It would be a useful exercise to review both the new and 
old mechanisms against the current observational catalogue.

The first measurement of individual events was made by \citet{Cordes1985} who
identified $\sim20$ events in both frequency and spindown which could not be
explained by a glitch. In the same work, considering 24 pulsars over a period
of~$\sim13$~years, the authors concluded that: the timing noise seen in the
data could not be explained solely by an idealised random walk processes in the
phase, or its derivatives. They suggested that most of the activity is due to a
mixture of events in the phase, frequency and/or frequency derivative.

A recent review, and the most comprehensive by far was performed by
\citet{Hobbs2010} for 366 pulsars. They found that, timing residuals tend to
admit quasi-periodic features when observed on sufficiently long time scales
$\gtrsim 2$~years. As such, the method of measuring the type and strength of
timing noise depends on the length and epoch of observation. This suggests a
pure random walk hypothesis is not entirely consistent with observations.
Nevertheless, it is unclear what repercussions the conclusions of
\citet{Hobbs2010} has for the physical origin of timing noise.

\subsection{Free precession}
\label{sec: free precession}

A mechanism which could quite naturally produce strictly periodic variations in the
observable features of a pulsar is \emph{free precession}. This occurs in any
non-spherical body for which the angular momentum is not aligned with a principle
axis of the moment of inertia. Such a circumstance could arise given the
chaotic birth of NSs. However, we must be clear that the timing noise induced by 
precession alone would be strictly deterministic; this is something which we do
not observe.
It is instructive however to consider the mechanics of precession since it will
be visited later on.

In the simplest case imagine an biaxial body, 
rotating about an axis $\Omega$. with a moment of inertia given by 
\begin{equation}
    I = \left[\begin{array}{ccc}
            I_{0} & 0 & 0 \\
            0 & I_{0} & 0 \\
            0 & 0 & I_{0}(1 + \epsilon)
            \end{array}\right],
\end{equation}
where $\epsilon \ll 1$ is the measure of oblateness or prolateness.  If the
body is free from torques, then in the rotating frame of the body, Euler's
equations of motion (see section \ref{sec: neutron star dynamics in the
rotating frame}) are given by
\begin{equation}
    I\dot{\bm{\Omega}} + \bm{\Omega} \times \left(I\bm{\Omega}\right)=0.
\end{equation}
This is a system of three coupled ODEs. Writing the components of the spin
vector as $\bm{\Omega} = [\Omega_{x}, \Omega_{y}, \Omega{z}]$, we have the
set of equations:
\begin{align}
\dot{\Omega}_x = -\Omega_y\Omega_z, &&
\dot{\Omega}_y = \Omega_x \Omega_z, &&
\dot{\Omega}_z = 0
\end{align}
We can find a solution by first realising that $\Omega_{z}=\mathrm{const}$.
We are then left with a set of
two coupled ODEs, solving these with appropriate intial conditions 
the solutions take the form
\begin{align}
    \Omega_{x} & = \Omega_{0}\sin(a_0)\sin\left(\Omega_{0}\cos(a_0)\epsilon t\right), \\
    \Omega_{y} & = \Omega_{0}\sin(a_0)\cos\left(\Omega_{0}\cos(a_0)\epsilon t\right),\\
    \Omega_{z} & = \Omega_0 \cos(a_0),
\end{align}
where $a_0$ is the angle between the spin-vector and the body frame $z$ axis and
$\Omega_0$ is the magnitude of the spin-vector. 

We observe that the spin axis of the body will trace out a cone about the $z$
principle axis of the moment of inertia with a period of
$\frac{1}{\Omega_{z}\epsilon}$.  The half-angle of the cone is set by the
initial conditions and will not evolve. This is the motion of free precession
and is illustrated in figure \ref{fig: precession}. 
\begin{figure}[htb]
\centering
\includegraphics[scale=0.2]{Precession.png}
\caption{Illustration of free precession for a simple biaxial body. The spin
    axis $\spin$ traces out a cone about the angular momentum vector $\mathbf{J}$.}
\label{fig: precession}
\end{figure}•

Neutron stars are assume to have a rigid crust, as such they may be non-axially
symmetric. Precession as a candidate to explain timing noise fluctuations was
first discussed by \citet{Ruderman1970}. He found that the free precession
period was, for reasonable values of the  ellipticity $\epsilon$, able to
explain periodic fluctuations in the Crab pulsar. Of the known Neutron star 
physics, one of the few mechanism that could operate over the time-scales observed
in timing residuals is free precession. 

The favoured interpretation for glitches poses a problem for sustained free
precession as an interpretation of timing noise. Theoretical models suggest the
interior of a neutron star is a superfluid; most of the moment of inertia is
contained in an array of vortices which are pinned to the crust.  Glitch events
correspond to the sudden unpinning of these vortices. It was shown by
\citet{Shaham1977} that for perfect pinning the free precession frequency and
geometry were modified resulting in no slowly oscillating long-lived modes. In
the case of imperfect pinning \citet{Sedrakian1999} found that long-lived modes
existed but where damped.

Despite the inconsistency with the superfluid pinning model for glitches,
evidence was presented by \citet{Stairs2000} of free precession in pulsar
B1828-11. They found the phase residuals and variations in the pulse profile
could be accounted for by precession.  This was followed up by detailed
modelling of the effects by \citet{Akgun2006}.

Including the spindown from an applied torque \citet{Cordes1993} noted that
free precession may be driven by fluctuations that counter the damping process;
in turn, the precession can drive torque fluctuations. The effect is most
noticeable in young pulsars. 

Work by \citet{Jones2001} compared a model of free precession against the handful
of proposed observations of free precession. Their model included the feedback
between the torque and precession and required only the crust to undergo precession.
In all but one case such a model was found to be consistent with the observations.


\subsection{Two state switching}
\label{sec: two state switching}

Recently a new model has been proposed by \citet{Lyne2010} to explain the
observation that, over long time periods, the timing noise structure is
quasi-periodic. This began with the observation by \citet{Kramer2006} that the
pulses from PSR B1931+24 where intermittent. The pulsar acts as a normal pulsar
for $\sim10$~days and then switches off, being undetectable for $\sim25$~days,
and then switching on again. Analysing the spindown rate between the on and off
states, they determined the spindown rate $\dot{\f}$ was $\sim50\%$ faster in
the on state. This figure illustrating this is reproduced in figure \ref{fig:
kramer 2006 fig2}.
\begin{figure}
    \centering
    \includegraphics[width=.5\textwidth]{Kramer_2006_fig2}
    \caption{Figure taken from \citet{Kramer2006} showing the switched spindown
             of pulsar PSR B1931+24}
    \label{fig: kramer 2006 fig2}
\end{figure}
In the  upper panel (\textbf{A}) the authors show the evolution of the
rotational frequency over a 160 day period encompassing several switching
events. The line shows the long-term spindown of the pulsar while the dots show
individual measurements made during the on state. During these on states the
gradient of the reduction in frequency is increased, that is the spindown has
increased. It is thought that measurements of the frequency in the off state
would produce a line with decreased spindown connecting the dots. This is
accompanied by the timing residual measured over the same period in the lower
panel (\textbf{B}). This shows significant quasi-periodic modulations in sync
with the switching. This work proposed that the switching was a magnetospheric
phenomenon. The sharpness of the switches certainly requires something acting
on a short timescale and it intuitively makes sense that the greater spindown
results from greater torque produced by the emissions observed during the on
state.

The authors of \citet{Lyne2010} then tested a range of other pulsars and
presented a study of 17 pulsars for which they claim evidence for two-state
switching. Unlike B1931+24 these pulsars are not intermittent but 
continuously pulse. Measuring the spindown as a function of time over a
$\sim20$~year period they demonstrated fluctuations as reproduced in figure
\ref{fig: lyne 2010 fig2}

\begin{figure}
    \centering
    \includegraphics[width=.5\textwidth]{Lyne_2010_fig2}
    \caption{Figure taken from \citet{Lyne2010} showing the spindown rate
             of 17 pulsars over a $\sim20$~year period.}
    \label{fig: lyne 2010 fig2}
\end{figure}

This plot shows smooth variations in the spindown with some pulsars better
behaved than others. The method used to calculate the spindown required
averaging over a $\sim100$~day period; the authors argue that, if the spindown
undergoes a sharp switch between two values, this will be smoothed out by the
averaging.  Therefore it is argued in \citet{Lyne2010} that figure \ref{fig:
lyne 2010 fig2} show the $\dot{nu}$ moving between a few (typically 2) well
defined values.  The authors noted a gradual long-term change in the spindown
for all pulsars across the data set, indicating a non-zero second order
spindown. 

In order to quantify the claim of \citet{Kramer2006} that the switching was
magnetospheric (e.g. the results of enhanced particle flow), \citet{Lyne2010}
looked for correlations between the pulse width, measuring the amount of
emission, with the spindown rate.  They found that for 6 pulsars the pulse
width was indeed wider during the higher spindown. However, these variations
are also smooth and subject to the same averaging process. To improve the
resolution they then show the individual measurements of pulse width for two of
the pulsars; these appear to demonstrate the switching happening
instantaneously between the two values. The authors argue this confirms that
the two-state switching is magnetospheric since this is the only mechanism able
to act on such short timescales. Intuitively it makes sense that changes in
the pulse shape, will be correlated with changed in the amount of emission and 
hence the spindown rate. 

It has not been established how these magnetospheric will manifest themselves.
In particular this interpretation lacks an explanation of how the magnetosphere is
regulated to stay in a stable state for long periods ($1-10$~years) but then
switch over $\lesssim100$~days.

\citet{Lyne2010} proposes that the quasi-periodic structure observed in many
timing residuals (e.g. see \citet{Hobbs2010}) could be explain by this
switching process. To quantify this, in the supplementary material they created
a simple model for the effect of switching in the value of the spindown
$\dot{\nu}$ on timing residuals. We will now repeat this experiment to outline
the results. 

Firstly we model a pulsar as spinning down in the usual way except that it's
spindown has two distinct values $\dot{\nu}_{A}$ and $\dot{\nu}_{B}$. Then we
define the model the ratio of time spent state in state $A$ and $B$ as $R =
t_{B}/t_{A}$.  This is a purely deterministic model and having generated the
spindown values we can integrated twice to get the phase. Fitting and
subtracting a quadratic polynomial leaves the phase residual. In figure
\ref{fig: lyne example D=0} we show a typical result; in the top figure is the
spindown values which we define and in the bottom the resulting structure in
the phase residual.
\begin{figure}[htb]
    \centering
    \includegraphics[width=.5\textwidth]{{R_3.0_D_0}.pdf}
    \caption{A deterministic realisation of the Lyne switched spindown model. The
             resulting structure in the timing residuals are strictly periodic.}
    \label{fig: lyne example D=0}
\end{figure}
The phase residuals in figure \ref{fig: lyne example D=0} are strictly
periodic. \citet{Lyne2010} realised that in order to fit the observed
quasi-periodic residuals a random element must be introduced. This can be done
by the form of a 'dither' $D$ in the waiting time between switches. Now we have
periods $t_{A}^{i}$ and $t_{B}^{i}$ which are Gaussian distributed with a mean
of $t_{A}$ and $t_{B}$ and a standard deviation $D t_{A}$ and $D t_{B}$. The
result is illustrated in figure \ref{fig: lyne example D=0.3}.

\begin{figure}[htb]
    \centering
    \includegraphics[width=.5\textwidth]{{R_3.0_D_0.3}.pdf}
    \caption{A realisation of the Lyne model with a random element producing the
             observed quasi-period structure.}
    \label{fig: lyne example D=0.3}
\end{figure}


\citet{Lyne2010} argue that the fast state changes seem to rule out free
precession as the origin of oscillatory behaviour observed in timing residuals.
One of the pulsars which shows some evidence for two state switching is PSR
B1828-11; this pulsar was cited as evidence for free precession by
\citet{Akgun2006}. \citet{Lyne2010} argue that the fluctuations from this pulsar
should be reinterpreted as two-state switch due to the observed fast state
changes. 

\citet{Jones2012} argues that such dismissal of precession is premature
since the modulation period of the switching has yet to be explained.  Instead,
the idea is raised that precession and magnetospheric switching are not
mutually exclusive.  Pulsars are most probably born in a randomly distributed
magnetospheric state, at least some may therefore exist under a delicate
balance between two states. Precession may be capable of periodically varying
the statistical probability of existing in one state or the other, sharp
changes would be caused by an `avalanche effect' as the particle energies reach
a threshold.  This provides the timescale for switching along with the ability
for the switching to be quasi-periodic since the precession only biases the
probability.

A similar idea considered by \citet{Cordes2013} interpreted two state switching
as evidence for a system in a state of stochastic resonance.  This occurs in
systems in which, under certain conditions, a weak periodic forcing function is
amplified by stochastic noise.  To explain this phenomenon in appendix
\ref{App: Stochastic} we present a toy model of stochastic resonance for a
particle in a well. The switching could therefore be the result of any periodic
modulation, such as precession, coupled to random fluctuations. This would quite
naturally explain the stability of states, the timescales over which the occur,
and the fact that it is observed in only some pulsars.

\subsection{Evidence from anomalous braking indices}
\label{sec: evidence from anomalous braking indices}

An alternative motivation to study timing noise comes from the measurement of
anomalous braking indices. The pulsar braking index is defined by $n$ in
equation \ref{eqn: power law spindown}.  Rearranging this equation as in
\eqref{eqn: measured braking index} the braking index can be measured for
observed pulsars. Different types of braking exhibit different braking indices.
It is therefore a reasonable idea to measure the braking index and calculate
the type of braking. Pulsars spun down by an electromagnetic torque should
follow a braking index of $n=3$, while gravitational wave spindown has $n=5$.

Measuring these indices for the known pulsar population we do not find a consensus
on the type of braking. Values from from unity up to $10^{6}$ and even negative
braking indices have been measured. These are known as \emph{anomalous}
braking indices. 

Recent work by \citet{Biryukov2012} observed that
younger pulsars tend to have braking indices of the correct order of
magnitude. However, beyond~$\tau_{ch}\approx10^{5}$~years
the absolute value of the braking index rapidly grows,
reaching values as large as $10^{6}$ for the oldest pulsar. In addition an
almost equal number of pulsars have positive and negative values of the braking
index. The figure demonstrating this is plotted in
figure~\ref{fig: braking indices}.

\begin{figure}[ht]
\centering
	\includegraphics[width=0.5\textwidth,trim=0mm -10mm 0mm 0mm]
               {{Biryukov_2012_Figure_7}.png}
\caption{Pulsar population in the $n_{obs}-\tau_{ch}$ diagram image from
\citet{Biryukov2012}}
\label{fig: braking indices}
\end{figure}

\citet{Biryukov2012} proposed that the spindown $\dot{\nu}(t)$ may contain the
secular spin down $\dot{\nu}_{\textrm{sec}}(t)$ and a cyclic component
$\dot{\nu}_{\textrm{sec}}(t)\epsilon(t)\nu(t)$ oscillating the spindown about
a mean value. Taking a simple case where the cyclic term has the form $A
\cos\phi(t)$ where $A$ is the relative amplitude of the oscillations and
$\phi(t)$ is linear in $t$, the authors derive an equation for the observed braking
index

\begin{equation}
n_{obs}(t) =
\frac{n}{1+A\cos(\dot{\phi}t+\phi_0)}
+\frac{(n-1)(kt-c)}{(1+A\cos(\dot{\phi}t+\phi_{0}))^{2}}A\dot{\phi}\sin(\dot{\phi}t+\phi_{0}).
\label{eqn: nobs}
\end{equation}•

This observed braking index contains a constant positive term oscillating about
the true braking index and a term which grows linearly in time. The authors found 
that for $\tau_{ch}<10^{5}$ yrs the linear term is negligible and so we observe
approximately the real braking index $n$. At later times the linear term
drives the observed braking index to larger values while a sinusoidal term produces
positive and negative values. In figure \ref{fig: nobs} we plot the trajectory
of a single pulsar following equation \eqref{eqn: nobs}. The authors claim each
of the pulsars in \ref{fig: braking indices} is following a similar trajectory.

\begin{figure}[ht]
\centering
	\includegraphics[width=0.5\textwidth]
               {{Analytic_Monotonic_and_Cyclic}.png}
\caption{A sketch of the observed braking index according to
equation \eqref{eqn: nobs}, the values here are intended for a qualitative
overview rather than analysis. }
\label{fig: nobs}
\end{figure} 

This simplistic idea is able to explain some of the defining features of the
known pulsar population braking indices. This requires a mechanism to modulate
the spindown over long timescales. By fitting their model to data, they
estimate the timescale to be of the order $10^{3}-10^{4}$~years. At least one
physical model, precession, could produce variations on the required timescale.
However, this is significantly longer than the precession timescales invoked to
explain the fluctuations in timing residuals which were $1-10$~years.
\begin{subappendices}
\subsection{Toy model of stochastic resonance: particle in a potential}
\label{App: Stochastic}

Here we present a simple toy model of stochastic resonance. This is a
statistical phenomena occurring when a weak periodic forcing function is
amplified by noise (see \citet{Jung1991} for a full treatment).  For the
application to neutron stars, see \citet{Cordes2013}; here we simply aim to
describe the essential features of stochastic resonance (not its application to
NSs). 

We will consider a particle at a position $x$ which is subject to some
potential and acted upon by a forcing function $F(t)$. In general though, $x$
could be any state variable, thus stochastic resonance could be produced
in many systems.

First consider the static case of a particle in a potential $U(x)$  given by:
\begin{equation}
    U(x) = \frac{x^{4}}{4}-\frac{x^{2}}{2}. 
\end{equation}
This potential is characterised by two wells at $\pm1$, a maximum exists
between them at the origin. The particle in one of the wells sees a potential
barrier $\Delta U$ corresponding to the height of the maximum above its
position.

Assume the particle is acted upon by a random forcing function $F(t)$ which is
modelled as a Gaussian white noise with strength $D$. Depending on the
magnitude of $D$ with respect to the potential, the motion of the particle
admits two distinct cases:
\begin{enumerate}
\item $D \ll \Delta U \;\;$ in which case the particle remains inside whichever
    well it initially starts in and does not escape.
\item $D \gg \Delta U \;\;$ in this case the particle will not see the the
    individual wells only the larger one.
\end{enumerate}

The motion of the particle obeys the following equation of motion:
\begin{equation}
    \frac{dx}{dt} = -\frac{\partial V(x,t)}{\partial x} + F(t). 
\end{equation}
The motion of the particle has two components, the deterministic effect of the
potential and random fluctuations.

We now modify the potential to be acted on by a weak periodic function; this
introduced a third possible type of behaviour. Writing the time dependant
potential as
\begin{equation}
    V(x,t) = \frac{x^{4}}{4}-\frac{x^{2}}{2} + \epsilon x \cos(\omega_{0} t).
\end{equation}
Inserting this potential into the equations of motion:
\begin{equation}
    \frac{dx}{dt} =  x - x^{3} + F(t) + \epsilon \cos(\omega_{0} t).
\label{eqn:stochastic eom}
\end{equation}
Solving this numerically we fix $\epsilon=0.001$,
$\omega_{0}=\frac{2\pi}{10}$ and choose three values of $D$ which illustrate
typical behaviours of the solution
\begin{figure}[ht]
\centering
   \includegraphics[width=0.4\textwidth,trim=0mm -10mm 0mm 0mm]
   {{Stochastic_resonance}.png}

\caption{Three solutions to equation \eqref{eqn:stochastic eom} changing the
    random forcing functions strength $D$. The first and last panels show the
    deterministic solutions for the particle position: either the forcing
    function is weak compared to the potential, the particle remains in well in
    which it begins; or the forcing function is much stronger than the
    potential and so the particle freely moves about the two wells. The middle
    panel illustrates the special case of stochastic resonance whereby the
periodic fluctuations of the potential allow quasi-periodic variations in the
particles position between the two wells.}

\label{fig:stochastic resonance}
\end{figure}
The first and last runs replicate the behaviour expected  for a static well,
either the particle is confined to the well it starts in, or the random noise
is too strong and the individual wells are not observed. The middle case
displays strong stochastic resonance: the solution
displays a switching between bi-stable states but does not strictly follow the
period of the forcing function. The
important point here is that the forcing function may be weak, but provided it
is periodic or at least quasi-periodic the signal is amplified by the random
noise such that it may be visible in data sets where it would typically be
considered lost.

\end{subappendices}


\biblio


\end{document}
