\documentclass[../full_thesis/full_thesis.tex]{subfiles}

% Default image directory
\newcommand{\thisdir}{../introduction}
\graphicspath{{\thisdir/img/}} 

\begin{document} 

Neutron stars were first postulated by Landau as `dense stars
which look like giant atomic nuclei' \citep{Yakovlev2013}, even before the
discovery of the neutron by \citet{Chadwick1932}. However it was
\citet{Baade1934} who made the explicit prediction of a neutron star whilst
trying to explain the energy released in observed supernova.

In a main-sequence star, the nuclear fusion of
hydrogen atoms into helium provides outward pressure balancing the star in an
equilibrium configuration with the inward pressure of the stars self-gravity.
Eventually the star depletes its reserves of hydrogen and can no longer
maintain equilibrium. If the star has an initial mass greater than $\sim 8
\Msun$, then it may undergo a \emph{core-collapse supernova} during which some
of the mass is ejected, but the rest falls in creating a new compact object.
In this object, temperatures and pressure rapidly sore and the
electrons and protons undergo inverse beta decay combining to form neutrons and
neutrinos:
\begin{equation}
    e^{-} + p \rightarrow n + \nu.
\end{equation}
If the compact objects mass is less than $\sim 5 \Msun$ then, once the pressures
reach nuclear densities of $\sim 2.3 \times10^{14}$~g/cm$^{3}$, neutron
degeneracy pressure can halt the collapse in a new compact stable equilibrium
configuration which we call a neutron star. For remnants with larger masses,
this is not possible and the object will collapse to form a black-hole; the
detail of exactly what the critical mass a neutron star can sustain is
sensitive to the equation of state of matter under these conditions.

Our knowledge of neutron stars is founded on observations made by
electromagnetic astronomy. This has revealed a wealth of different neutron
stars and their phenomena which we will introduce in the following sections.
Many of these observed phenomena have well defined models which allow us to
infer properties of neutron stars themselves.
However, our knowledge of neutron stars is far from complete: current
observations can be contradictory or have features not explained by any known
physical models. Improvements in electromagnetic astronomy will bring to light
a greater number of new neutron stars and improve the resolution of those
currently observed; it is hoped that this will help us to better understand them.

There are two other methods we can utilise to learn more about neutron stars: improved
modelling of current observations and by observing them from their
gravitational wave emissions. In this thesis, we will study how one of the
observed phenomena, so-called `timing variations', can help us to learn more
from current observations and also test whether it may hinder the current
search for gravitational waves from neutron stars.

In this introduction, we will acquaint the reader with the current observations
of neutron star and introduce some of the basic physics.

\section{Observation of pulsars and their identification with neutron stars}
A bright periodic EM signal was identified by \citet{Hewish1968} during a high
time-resolution survey for interplanetary scintillation. The shortness of
pulses and frequent and precise periodicity suggested that the source was
small. For a star to remain gravitationally bound, its rotation frequency is
constrained by the requirement that the centrifugal acceleration at the equator
be less than the gravitational acceleration. The observed pulses had a short
period $\sim 1.33$~s which ruled out white dwarfs since they could rotate this
fast without becoming gravitationally unbound. After the identification of
other similar objects these sources were named \emph{pulsars}. 

While multiple interpretation's were posed to explain the observation, the
identification of pulsars with neutron stars came from \citet{Pacini1967} and
independently \citet{Gold1968}. They suggested that a rapidly rotating neutron
star with a strong dipolar magnetic field would stream radiation out along the
magnetic axis. If this axis was misaligned from the rotation axis then the
beams would be swept out like a lighthouse. Beams passing over the earth are
observed as periodic pulses at the rotation frequency of the star.  This
identification was confirmed by the discovery of a pulsar at the centre of Crab
nebulae, a supernova remnant, agreeing with the prediction of \citet{Baade1934}.




\section{Pulsar timing}
\label{sec: pulsar timing methods}
Pulsars can be observed by measuring the variation in amplitude of the radio
waves from a particular sky location. A single observation consists of measuring
the amplitude over a period of approximately $30$ minutes or so, which for pulsars
with periods $\sim 1$~s, means recording up to several thousand individual pulsations.

The shape of individual pulses can vary substantially during a single
observation; to show this, in Fig.~\ref{fig: CP1919 stacked}, successive pulses
from the first pulsar discovered, PSR B1919+21, are vertically stacked and
aligned illustrating typical variations.
\begin{figure}[htb]
\centering
\includegraphics[width=0.5\textwidth]{CP1919_stacked}
\caption{The radio amplitude of successive pulses from PSR~B1919+21 stacked
vertically, figure reproduced from \citet{mitton1977cambridge}, originally
produced by \citet{craft1970}.}
\label{fig: CP1919 stacked}
\end{figure}
Each pulsation lasts for a small fraction of the pulse period. As an example,
the pulses in PSR~B1919+21, shown in Fig.~\ref{fig: CP1919 stacked}, have
typical widths of $0.031$~s, but the pulse period is $1.337$~s; note that the
stacked plot truncates each pulsation to show only the pulsation itself. We will
demonstrate this is true for the whole pulsar population in
Sec.~\ref{sec:}.

In order to understand the gross features of a pulsar, astronomers average over
the hundreds to thousands of pulses observed during a single observation to
create a single integrated pulse profile. This is done by sampling the radio
signal at fixed time intervals then `folding' all the samples at the pulse
period (for a complete review see \citet{Lyne2012book}). The integrated pulse
profile of a pulsar looks similar to any of the individual pulses seen in
Fig.~\ref{fig: CP1919 stacked}, but in contrast to the individual pulses which
create it, it can be highly stable when measured independently over timescales
of years.

The integrated pulse profile not only gives a stable picture of what the
pulsations look like on average, but it also provides a highly accurate
measurement of the time of arrival (TOA) of a single pulse during the
observation; which pulse depends on how the folding is done. It is this TOA
which can be used to `time' a pulsar. To do this, regular observations of a
pulsar must be made every few months or so, each observation results in a
precise TOA measurement. Having obtained a series of TOAs, pulsar astronomers
generate a \emph{timing model} which attempts to exactly count each and every
pulse. Between any two observations there may be several million pulses so the
timing model needs to account for any mechanisms which may produce variations
in the TOAs.  The process is standardised by the software package
\texttt{TEMPO2} developed by \citet{Hobbs2006}, we will now describe the
essential features.

The TOA of a pulse at the detector on Earth depends on many factors such as the
time at which the beam was directed by source towards the Earth, the relative
motions of the source and detector, and any mechanisms effecting the signal during
its transit. The time at which pulses are generated (when the source beams towards
the earth) are governed by the \emph{timing properties} of the star itself.
As described by \citet{Edwards2006} these can be modelled by a Taylor expansion
in the phase at time $t$ given by
\begin{equation}
\phi(t) = \sum_{n\ge 1}\frac{\nu^{(n-1)}}{n!}(t - \tref)^{n} + \phi_{0}.
\label{eqn: Taylor compact}
\end{equation} 
where $\{\nu^{(n)}\}$ is the frequency $\nu$, spin-down rate $\dot{\nu}$, etc.
of the rotating body, $\phi_0$ is the initial phase, and $\tref$ is an
arbitrary reference time. This expansion is usually truncated at $n=3$, the
second order spin-down rate $\ddot{\nu}$. The timing model then includes
corrections to this to model the relative motion of the source and detector,
intergalactic transit, and other effects; these are described in full in
\citet{Edwards2006}. 

Between any two TOAs, if the timing model is correct, an integer number of
rotations must have occurred; this allows the use of the deviation of
$\phi(t_{TOA}^{j})$ from an integer as a test statistic. The timing model
minimises the RMS of these deviations with respect to the timing model
parameters, for example the frequency and frequency derivatives. The output of
applying a particular timing model (choice of corrections) and a set of data is
then the best-fit of these parameters and an estimate of their associated
errors.  The corrections applied in a timing model provide a method to
investigate pulsar physics: for example in some pulsar an orbital correction
must be applied which models the periodic motion of the star due to an orbital
companion. Using this technique, \citet{wolszczan1992planetary} discovered the
first \emph{exoplanet} orbiting the pulsar PSR~B1257+12.

A minimisation of the timing model parameters will converge regardless of
whether of not the model itself is appropriate. To qualitatively check this,
pulsar astronomers refer to the \emph{timing residual}, which is the difference
between the TOA, as given by the timing model, and the actual TOA. The timing
residual provides a mechanism to evaluate the timing model: a periodic
variation in the timing residual with period $365$~days, may indicate the correction
of the Earth's orbit about the Sun may be incorrect.

If the timing model is accurate enough to track the pulsar to within a single
rotation the resulting timing solution is described as \emph{phase-connected}.
For most pulsars this is the case and a single set of coefficients can track
the spindown over periods greater than a year. For well timed pulsars the
timing residual will display a `white` noise with a mean of zero. However, for
many pulsars the timing residual contains `structure' known as \emph{timing
variations} which cannot be associated with any known correction, this is the
focus of this work and we describe the phenomenon in Sec.~\ref{sec: timing
variations}. In the next section, we will study the variety of known pulsars
which have been timed using this method.




\section{Categorising neutron stars}
\label{sec: categorising neutron stars}
The timing properties, and other features measured by the timing model, for
over 2000 pulsars observed can be accessed via the Australia Telescope National
Facility (ATNF) pulsar catalogue \citep{ATNF}.  We can categorise the
population by their measured values of period $P$ and the period derivative
$\Pdot$. This is done by plotting them in a so-called $P - \Pdot$ diagram, as
shown in Figure~\ref{fig: Period_PeriodDot}.  The various categories to which
each pulsar can be assigned have been marked in this plot and we now discuss
their features.

\begin{figure}[htb]
\centering
\includegraphics[]{Period_PeriodDot} 
\caption{Period -
Period derivative diagram using data taken from the ATNF pulsar catalogue
\citep{ATNF}. Dashed lines show inferred characteristic ages and 
surface magnetic fields as given by Eqn.~\eqref{eqn: tauAge definition}
and Eqn.~\eqref{eqn: surface magnetic field canonical} respectively.}
\label{fig: Period_PeriodDot}
\end{figure}

The majority of pulsars (referred to as the `normal' pulsars) are found
\emph{isolated} without a binary companion and have typical periods of
$P=10^{-2}-10^{1}$~s. These can be described as \emph{rotation powered}
pulsars, since the electromagnetic (EM) radiation is powered by the loss of
rotational energy. As described later in Section~\ref{sec: rotation powered
pulsars}, estimates can be made of their characteristic age, $\tauAge$, and
surface magnetic field strength, $B_{0}$, based on a dipole spin-down model.
Constant lines of these quantities are plotted in Figure~\ref{fig:
Period_PeriodDot}. Of the normal pulsars, we can choose to identify the young
pulsars as those for which $\tauAge<10^{5}$~yrs. Some of these, such as the
Crab and Vela pulsars, can be directly associated with their supernova remnant
from which they were formed \citep{Kaspi1996}.

A second smaller population of isolated rotation powered pulsars exists with
$P<10^{-1}$~s. These are known as the \emph{millisecond pulsars} (MSPs). This
special class of pulsars are believed to start their life as normal pulsars,
but are then spun-up through accretion from a normal star. In support of this
hypothesis, the majority of MSPs in Figure~\ref{fig: Period_PeriodDot} have a
binary companion \citep{wijnands1998millisecond}. Additionally, we see so-called
low-mass X-ray binary systems (LMXBs) which are systems where a neutron star in
a binary accretes matter from its companion; the infalling matter releases
gravitational potential energy in the form of X-rays \citep{lewin1997x}. It is
thought that these LMXBs are the progenitors of the MSPs and recent results of
`transitional systems' (see for example \citep{archibald2009radio}) which
switch between the two seem to confirm this.

We include one final class of neutron stars, \emph{magnetars}, thought to have
large magnetic fields of $B\gtrsim 5\times10^{13}$~G. These are in fact
observed from two channels which we will now describe. Some pulsars are
observed to emit X-ray radiation; usually this is powered by the accretion of
matter from a binary companion, but this mechanism does not apply to isolated
stars. Subsequently, isolated stars observed in the X-ray band where named
\emph{anomalous X-ray pulsars} (AXPs). It was shown by
\citet{duncan1996magnetars} that AXPs are magnetars where the emission is
powered by the decay of the strong magnetic field.  At the same time,
astronomers found a class of objects emitting irregular bursts of
$\gamma$-rays or X-rays which they named the \emph{soft $\gamma$-ray repeaters}
(SGRs). As discussed in \citet{kouveliotou2003magnetars} these are now
understood to be magnetars which undergo rearrangement of their magnetic
fields. The two individual observations where unified by observation of X-ray
bursts from AXPs by \citet{Gavriil2002}. In Figure~\ref{fig: Period_PeriodDot} we
label observations from both these sources as magnetars.


\section{Radio pulsar population statistics}
\label{sec: population stats}
Radio pulsars make up the majority of the observed neutron star population and will be
the focus of discussion in this thesis. In this section, we will provide some
population statistics for the normal radio pulsar population: we ignore the
millisecond population since they are disjoint from the normal population and
have a distinct history, but include the young pulsars. All data
in this section is taken from the ATNF pulsar catalogue \citet{ATNF} and it
should be stated that in each case the observed property is an average
over all observations made for each pulsar.

For each observable property of the population of neutron stars (such as the
frequency), we will present the data as a histogram choosing an appropriate
binning size in each instance. In order to make simple inferences about the
population, we will also calculate the mean and standard-deviation.
We will test normality using the \citet{Scipy} implementation of the
\citet{d1971omnibus} test. This results in a $p$-value, which, if less than
0.05, rejects the hypothesis that the data is normal with $95\%$ confidence. We
will give this $p$-value in the legend for each observable property and show
the normal distribution with the calculated mean and standard-deviation.


In Figure~\ref{fig: pop stats timing} we present the data for the three timing
properties measured directly from the pulsar timing models. For normal radio
pulsars the pulsation frequency, $\nu$, can always be accurately measured
provided at least
one observation has been made.  Several precise observations of a pulsar must
be made in order to measure the higher order derivatives of the frequency. As a
result, the pulsar catalogue contains missing information and the number of
data points for $\dot{\nu}$ and $\ddot{\nu}$ is smaller than the total observed
number of pulsars: the exact numbers are given in the caption.
\begin{figure}[htb]
\centering
\includegraphics[]{timing_distribution}
\caption{The distribution in log-space of the frequency $\nu$ and the first two
frequency derivatives $\dot{\nu}$ and $\ddot{\nu}$ for normal radio pulsars in the
ATNF pulsar catalogue. Appropriate bin sizes were selected for each quantity.
The population sizes are 1942, 1686, 339 for $\nu$,
$\dot{\nu}$, $\ddot{\nu}$ respectively.}
\label{fig: pop stats timing}
\end{figure}
By eye, the histograms are clustered and appear to be approximately normal.
However, for all three properties, the normal hypothesis is rejected; we note
that the level of rejection is dependent on the number of data points.

In Figure~\ref{fig: pop stats others} we present some other interesting quantities
held in the ATNF catalogue. Firstly, in the left-hand panel we plot the characteristic
age as defined in Eqn.~\eqref{eqn: characteristic age}. Then, in the middle panel
we give a measure of the pulsars beam-width $W_{10}$. Specifically, $W_{10}$ is
the width of the integrated pulse profile (in seconds) at 10\% of the integrated
pulse profile maximum. In the right-hand panel we  plot $W_{10}\nu$, i.e. the
product of the beam-width and frequency for each pulsar. This gives information
about the effective duty-cycle: the ratio between the pulse duration and period.
Notably, the majority of pulsars have duty-cycles substantially less than a
$0.5$ indicating that the pulses are short compared to the period.
\begin{figure}[htb]
\centering
\includegraphics[]{W10_and_age_distribution}
\caption{The distribution in log-space of the characteristic age
$\tau_{\textrm{Age}}$, the $W_{10}$ measure of the beam-width, and the
effective duty-cycle $W_{10} \nu$ for normal radio pulsars in the
ATNF pulsar catalogue. Appropriate bin sizes were selected for each quantity.
The population sizes are 1942, 915, and 915 for
$\tau_{\textrm{Age}}$, $W_{10}$, and $W_{10}\nu$.}
\label{fig: pop stats others}
\end{figure}
In this instance, the normal hypothesis is rejected for the beam-width and
duty-cycle, but accepted for the spin-down age. It would be an interesting
exercise to investigate this further.

These results discussed in this section provide an overview
of the observed radio pulsar population. We have shown that, while the populations
are not normally distributed, they nevertheless have well defined distributions
that are not too distinct from normality. We will use this overview to provide
context for other results in this thesis.



\section{The physics of rotation powered pulsars} 
\label{sec: rotation powered pulsars}
Typically observers can measure only the period and a few derivatives from a
pulsar (the method to do this is described in section \ref{sec: pulsar timing
methods}).  We now consider what can be learnt from the observed $P$ and
$\Pdot$ if we assume a simple dipole spindown model. In figure \ref{fig:
DipoleSpindownSimple} we illustrate a rotating rigid body with a dipole fixed
at an angle $\alpha$ to the rotation axis; we will assume the body rotates in
vacuum. 

\begin{figure}[htb]
    \centering
    \includegraphics[width=.5\textwidth]{DipoleModelSimple}
    \caption{An illustration of the dipole spindown model. The dipole and some 
    of the closed field lines are fixed at an angle $\alpha$ to the rotation 
    axis. As the body rotates, radiation is emitted along both ends of the dipole
    axis producing a torque on the body.}
    \label{fig: DipoleSpindownSimple}
\end{figure}

The rotational energy of a body spinning at $\Omega$ with a moment of inertia
$I_{0}$ is given by
\begin{equation}
    E = \frac{1}{2}I_{0}\Omega^{2}.
\end{equation}
Differentiating this expression, we equate this loss of rotational energy
to the rate at which a magnetic dipole in free space radiates energy as found
by \citet{Pacini1967}:
\begin{equation}
    I_{0}\Omega\dot{\Omega} = 
                    -\frac{2\Omega^{4}}{3 c^{3}} R^{6} B_{0}^{2}\sin^{2}\alpha,
    \label{eqn: equate dKE to dEM} 
\end{equation}
where $R$ is the radius of the NS, $B_{0}$ is the surface magnetic field, and
$\alpha$ is the angle between the dipole and the rotation axis.  

In this simple model we can interpret the type of spindown by defining a power
law dependence for the braking
\begin{equation}
    \dot{\Omega} = -k \Omega^{n},
    \label{eqn: power law spindown}
\end{equation}
where $k$ is constant of proportionality, $\dot{\Omega}$ is the spindown, and
$n$ defines the \emph{braking index}. Rearranging equation 
\eqref{eqn: equate dKE to dEM} to compare with this power law, we find that for
magnetic dipole spindown:
\begin{align}
    n = 3 && \textrm{ and } && k = \frac{2B_{0}^{2}R^{6} \sin^{2}\alpha}{3 c^{3} I_{0}}.
    \label{eqn: n and k}
\end{align}
\textcolor{red}{This is different to Shapiro by a factor of 1/4}\\
For gravitational wave powered spindown, it can be shown that $n=5$
(see \citet{Shapiro83}). This suggests a powerful method to determine the spindown
mechanism if the braking index can be measured. This can be done if
$\ddot{\Omega}$ can be measured: rearranging equation \eqref{eqn: power law
spindown} to give
\begin{equation}
    n = \frac{\ddot{\Omega}\Omega}{\dot{\Omega}^{2}}.
    \label{eqn: measured braking index}
\end{equation}
The results of this are are discussed further in section \ref{sec: evidence from
    anomalous braking indices}.

For a pulsar powered by rotation, equation \eqref{eqn: power law spindown} can
be integrated between the initial values ($t=0, \Omega=\Omega_{i}$) and the
observed value ($\Omega_{o}$) to give an expression for the age of the pulsar
\begin{equation}
    t = \frac{1}{(1-n)} \frac{\Omega_{o}}{\dot\Omega_{o}} 
        \left(1 - \frac{\Omega_{o}^{n-1}}{\Omega_{i}^{n-1}}\right).
\label{eqn: characteristic age}
\end{equation}
Assuming magnetic dipole braking ($n=3$) and $\Omega_{i} \gg \Omega_{o}$ we can
approximate to a characteristic age 
\begin{equation}
    \tau = \frac{-1}{2}\frac{\Omega_{\textrm{o}}}{\dot\Omega_{o}}
         = \frac{1}{2}\frac{P}{\Pdot}.
\end{equation}
Where $P=\frac{2\pi}{\Omega}$ is the pulse period and
$\dot{P}=-2\pi\frac{\dot{\Omega}}{\Omega^{2}}$ is the period derivative.
Similarly
rearranging equations \eqref{eqn: power law spindown} and 
\eqref{eqn: n and k} we can estimate the surface magnetic field strength by
\begin{equation}
    B_{0} = \left(\frac{3 c^{3} I_{0}}{2R^{6} \sin^{2}\alpha}\right)^{\frac{1}{2}} 
            \left(\frac{-\dot{\Omega}}{\Omega^{3}}\right)^{\frac{1}{2}}
          = \frac{1}{2\pi}\left(\frac{3 c^{3} I_{0}}{2R^{6} \sin^{2}\alpha}\right)^{\frac{1}{2}}
           \sqrt{P \Pdot}
\label{eqn: surface magnetic field}
\end{equation}
In general we do not know the inclination angle $\alpha$, but we can evaluate a 
minimum magnetic field strength by setting $\alpha=\pi/2$. In CGS units for a
canonical pulsar with $R=10^{6}$~cm, $I_{0}=10^{45}$~g~cm$^{2}$ we can approximate
the magnetic field strength as
\begin{equation}
    B_{0} = 3.2 \times 10^{19} \sqrt{P \Pdot}.
\label{eqn: surface magnetic field canonical}
\end{equation}




\section{Neutron stars and gravitational waves}
\label{sec: gravitational waves}
Gravitational waves (GWs) were first predicted by Albert Einstein in 1916
\citep{einstein1916approximative} when he found that the linearised weak-field
equations of his General Theory of Relativity had transverse wave solutions.
Much like the generation of electromagnetic waves requires the acceleration of
electrical charges, GWs are generated by any source with a
time-varying mass quadrupole moment and can be understood as `ripples' or
spatial strains in the spacetime itself which travel at the speed of light.

Gravitational waves were first directly detected by the LIGO collaboration
\citep{abbott2016observation}. They observed a signal consistent with the
inspiral and merger event of two $\sim 30 \Msun$ black-holes over approximately
$0.2$~s. To detect such signals, LIGO uses a \emph{laser interferometer} to
measures the relative change in length between two orthogonal arms. In
particular, if $L$ is the length of either arm without a signal and a
gravitational wave passes through, the detector measures the strain
\begin{align}
h(t) = \frac{\delta L_x - \delta L_y}{L}
\end{align}
where $\delta L_x$ and $\delta L_y$ are the time-varying stretching and
squeezing of the two arms caused by the gravity wave. For the observed binary
black-hole merger the peak strain in the detector was $\sim 10^{-21}$.

Prior to this detection, indirect evidence for the existence of gravitational
waves was found by observing the orbital periods of compact binary systems.
Such systems have a time-varying quadrupole moment and emit gravitational
waves, which radiate energy away from the system causing, a decay of the orbital
period. In 1975, Hulse \& Taylor discovered a binary neutron star system where
one of the stars, PSR~1913+16, was visible as a pulsar \citep{Hulse1975}. Due to
the powerful techniques of pulsar timing, subsequent analysis by
\citet{Taylor1982} was able to verify that the orbital decay matched exactly
the predictions of General Relativity. Since this observation, more double
neutron star system have been discovered, including a system, PSR~J0737-3039A/B,
discovered by \citet{burgay2003increased}, where both neutron stars are seen as
pulsars. This so-called double pulsar system tests the agreement with General
Relativity at the 0.05\% level \citep{kramerstairs2006}.

Neutron stars observed as pulsars are often referred to as `cosmic clocks' for
the regularity of their pulsations. The most stable pulsars are the radio
millisecond pulsars (MSPs), which, due to their stability, many workers in the field
utilise in an attempt to search for GWs via a \emph{pulsar timing array}
\citep{hobbs2010international}: this searches for correlated signatures in the
TOAs from a network of well-timed MSPs. Such a detector is sensitive to a
stochastic background of gravitational waves by measuring the so-called
Hellings \& Downs curve \citep{hellings1983upper}, or to the mergers of
super-massive black hole binary systems \citep{lee2011gravitational}.

Isolated neutron stars themselves are potential sources of gravitational waves
through one of three mechanisms. If the star has a rotation axis misaligned
with its symmetry axis then it will undergo \emph{precession}: a `wobble` of
the star which has a time-varying quadrupole moment. This will produce GWs at
the rotation frequency and twice the rotation frequency, but the small
amplitudes of possible sources and questions over how long lived they might be
make this an unlikely candidate for LIGO \citep{Jones2002}. If the neutron star
is subject to \emph{non-axisymmetric instabilities}, such as the R-mode
instability in newborn and rapidly accreting neutron stars
\citep{andersson2001r}, then these too can produce GWs (for a review see
\citet{andersson2003gravitational}).  Finally, if the star possesses a
\emph{non-axisymmetric distortion}, $\epsilon$, also known as a `mountain', it
will produce a continuous gravitational wave at twice its rotation frequency
with a strain amplitude proportional to $\epsilon$. The LIGO detectors have
already been used to search for signals from known neutron stars and, by not
observing any radiation, are able to place upper limits on $\epsilon$ (see for
example \citet{ligo2008, ligo2011}).

All three of these detection mechanisms are potential sources of the first detection
of gravitational waves from neutron stars and realising this would provide a
unique opportunity to learn about neutron stars. But is it feasible? A statistical
argument can be made for the `loudest expected signal from unknown isolated
neutron stars'. This argument is given in \citet{abbott2007searches},
although the origin can be dated back to Blandford (1984) as attributed by Kip Thorne
in \citet{Hawking1989}. Essentially, one assumes that the population of $10^{5}$
neutron stars predicted to exist in our galaxy by stellar evolution models are
all born with a high spin-down rate and subsequently spin-down principally due to
the emission of gravitational waves. With additional assumptions that the stars
are born randomly throughout the Galactic disk with a constant birthrate the
populations are transformed into a population of neutron star strains. Then it
is shown that there is a 50\% chance a source exists with a strain amplitude
\begin{align}
h_0 \sim 4 \times 10^{-24},
\end{align}
which is close to `detectable' by LIGO, although the exact details depend on the
source frequency and duration. While this is a purely statistical argument, and
changing any of the assumptions tends to decrease this signal strain
\citep{Prix2009}, the rewards for detection in terms of astrophysics are
sufficient to motivate further research.




\section{Plan of the thesis}

Following this introductory chapter, we will have a chapter introducing
so-called timing variations in pulsars: glitches and timing-noise. This will
introduce the observed phenomena and describe the current state of modelling.
We will provide some original work on simple ways which the models could be
tested.

In the next four chapters we evaluate models of timing noise in the face of
current observations and attempt to constrain the models.  In Chapter~\ref{sec:
rotating frame} we explore how \emph{precession}, a potential ingredient to
explain timing-noise, operates in the inertial frame of the star.  Following
this, Chapter~\ref{sec: rotating frame} looks at how precession will manifest
in the observations made by pulsar astronomers. In Chapter~\ref{sec: testing
models} we perform a rigorous quantitative model comparison between precession and the
leading alternative, \emph{switching}, for PSR~B1828-11. This is then extended
in Chapter.~\ref{sec: extending precession models} where we use the model
comparison tools to discover previously unknown fundamental physics of the
pulsar.

In the final three chapters we approach another important aspect of timing
variations for neutron stars: the effect they may have on our ability to
detection gravitational waves from neutron stars. In Chapter~\ref{sec: intro to
cgw} we introduce the methods and formalisms used by gravitational wave
astronomy before analysing the effect of glitches in Chapter~\ref{sec: glitches
in cgw} and the effect of timing-noise in Chapter~\ref{sec: timing noise in
cgw}.

\biblio


\end{document}
